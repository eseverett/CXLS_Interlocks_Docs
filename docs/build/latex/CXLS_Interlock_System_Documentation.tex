%% Generated by Sphinx.
\def\sphinxdocclass{report}
\documentclass[letterpaper,10pt,english]{sphinxmanual}
\ifdefined\pdfpxdimen
   \let\sphinxpxdimen\pdfpxdimen\else\newdimen\sphinxpxdimen
\fi \sphinxpxdimen=.75bp\relax
\ifdefined\pdfimageresolution
    \pdfimageresolution= \numexpr \dimexpr1in\relax/\sphinxpxdimen\relax
\fi
%% let collapsible pdf bookmarks panel have high depth per default
\PassOptionsToPackage{bookmarksdepth=5}{hyperref}

\PassOptionsToPackage{warn}{textcomp}
\usepackage[utf8]{inputenc}
\ifdefined\DeclareUnicodeCharacter
% support both utf8 and utf8x syntaxes
  \ifdefined\DeclareUnicodeCharacterAsOptional
    \def\sphinxDUC#1{\DeclareUnicodeCharacter{"#1}}
  \else
    \let\sphinxDUC\DeclareUnicodeCharacter
  \fi
  \sphinxDUC{00A0}{\nobreakspace}
  \sphinxDUC{2500}{\sphinxunichar{2500}}
  \sphinxDUC{2502}{\sphinxunichar{2502}}
  \sphinxDUC{2514}{\sphinxunichar{2514}}
  \sphinxDUC{251C}{\sphinxunichar{251C}}
  \sphinxDUC{2572}{\textbackslash}
\fi
\usepackage{cmap}
\usepackage[T1]{fontenc}
\usepackage{amsmath,amssymb,amstext}
\usepackage{babel}



\usepackage{tgtermes}
\usepackage{tgheros}
\renewcommand{\ttdefault}{txtt}



\usepackage[Bjarne]{fncychap}
\usepackage{sphinx}

\fvset{fontsize=auto}
\usepackage{geometry}


% Include hyperref last.
\usepackage{hyperref}
% Fix anchor placement for figures with captions.
\usepackage{hypcap}% it must be loaded after hyperref.
% Set up styles of URL: it should be placed after hyperref.
\urlstyle{same}

\addto\captionsenglish{\renewcommand{\contentsname}{Contents:}}

\usepackage{sphinxmessages}
\setcounter{tocdepth}{1}



\title{CXLS_Interlock_System_Documentation}
\date{Mar 05, 2024}
\release{2024}
\author{Eric Everett}
\newcommand{\sphinxlogo}{\vbox{}}
\renewcommand{\releasename}{Release}
\makeindex
\begin{document}

\ifdefined\shorthandoff
  \ifnum\catcode`\=\string=\active\shorthandoff{=}\fi
  \ifnum\catcode`\"=\active\shorthandoff{"}\fi
\fi

\pagestyle{empty}
\sphinxmaketitle
\pagestyle{plain}
\sphinxtableofcontents
\pagestyle{normal}
\phantomsection\label{\detokenize{index::doc}}



\chapter{About This Documentation}
\label{\detokenize{index:about-this-documentation}}
\sphinxAtStartPar
The purpose of this documentation is to provide a complete guide to the CXLS interlock system.
This document will outline user guides to all interlock subsystems, as well as provide documentation on testing and troubleshooting the system.


\chapter{CXLS Facility Overview}
\label{\detokenize{index:cxls-facility-overview}}
\noindent{\hspace*{\fill}\sphinxincludegraphics[scale=0.5]{{CXLS_FACILITIES}.png}\hspace*{\fill}}

\sphinxAtStartPar



\chapter{Ionizing Radiation Hazards}
\label{\detokenize{index:ionizing-radiation-hazards}}
\sphinxAtStartPar
In the CXLS linear accelerator, relativistic electrons can interact with materials of beam pipes and LINACs.
These Coulomb interactions between the relativistic electrons and atomic nuclei within these materials cause the electrons to experience acceleration and release a high energy photon.
This process is known as Bremsstrahlung. These emitted gamma rays can then interact with nuclei and through the process of photodisintegration neutron radiation is produced.
These processes cause elevated radiation felids in Vault\sphinxhyphen{}1 and Hutch\sphinxhyphen{}1.


\chapter{Laser Hazards}
\label{\detokenize{index:laser-hazards}}
\sphinxAtStartPar
There are three class 4 lasers used throughout the CXLS system.
A UV laser is used at the photocathode to eject electrons in bunches via the photoelectric effect.
Once these electrons are at relativistic speeds, they collide with an IR laser and through inverse Compton scattering produce hard x\sphinxhyphen{}ray pulses.
These x\sphinxhyphen{}rays will interact with a test sample in pump\sphinxhyphen{}prove configuration where the pump laser can produce light in the THz spectrum.
Due to these high power lasers throughout the facility, laser enclosures have been designed to create laser safe areas while lasers are operational.


\chapter{Interlocks System}
\label{\detokenize{index:interlocks-system}}
\sphinxAtStartPar
The CXLS interlocks system is comprised of sets of sensors and components that send data to predefined digital logic systems.
This digital logic decides to change the state of actuators and safety displays.
These interlocks are designed to enforce proper sequential operations and control access to hazards.
Redundancy is built into the system to increase reliability, as well as fail safe designs to ensure systems are put into a safe state incase of a user of system failure.

\sphinxstepscope


\section{Vault\sphinxhyphen{}1 Ionizing Radiation Interlock System User Manual}
\label{\detokenize{user_documentation/Vault-1_ionizing_radiation:vault-1-ionizing-radiation-interlock-system-user-manual}}\label{\detokenize{user_documentation/Vault-1_ionizing_radiation::doc}}
\sphinxAtStartPar
This document provides a user overview for the Vault\sphinxhyphen{}1 Ionizing Radiation Interlock System.
It covers the hazard indicators, control protocase, VIEWMARQ displays, and beacons in Vault\sphinxhyphen{}1 Control and Accelerator Lab.
The control protocase allows users to view and change the secure state of Vault\sphinxhyphen{}1, arm the accelerator and transmitters, and view area monitor statuses.
The VIEWMARQ displays provide information on the ionizing radiation hazard status, laser hazards, and the secure state of Vault\sphinxhyphen{}1.
The beacons indicate the state of RF and ionizing radiation emergency stop buttons.


\subsection{Vault\sphinxhyphen{}1 Ionizing Radiation Hazard Indicators}
\label{\detokenize{user_documentation/Vault-1_ionizing_radiation:vault-1-ionizing-radiation-hazard-indicators}}
\sphinxAtStartPar
This section will cover the ionizing radiation hazard indicators in Vault\sphinxhyphen{}1 Control and Accelerator Lab.
These indicators will correspond to hazards present in Vault\sphinxhyphen{}1 and RF\sphinxhyphen{}1.


\subsubsection{Vault\sphinxhyphen{}1 Control Protocase}
\label{\detokenize{user_documentation/Vault-1_ionizing_radiation:vault-1-control-protocase}}
\sphinxAtStartPar
The Vault\sphinxhyphen{}1 Control IONIZING RADIATION INTERLOCK protocase is an interface to view if Vault\sphinxhyphen{}1 is secure, arm the accelerator and transmitters, and view the status of the area monitors.
This panel is located on the east wall in Vault\sphinxhyphen{}1 Control next to the Vault\sphinxhyphen{}1 door.

\sphinxAtStartPar
The SECURE PERIMETER section of the protocase shows the status of search buttons being pressed during a search and if the shield door is open.
If all the lamps are \DUrole{green}{green}, then Vault\sphinxhyphen{}1 is in a secure state. When Vault\sphinxhyphen{}1 is secure it may not be entered until accelerator operations are concluded and Vault\sphinxhyphen{}1 is surveyed.
If the shield door is opened with the transmitters armed, then the interlock system will trip and put the transmitters into a safe state.

\sphinxAtStartPar
The AREA MONITORS section of the protocase shows the status of RADIATION, OXYGEN, and MICROWAVE area monitors.
The corresponding hazard lamps will turn \DUrole{red}{red} if any of the ionizing radiation, \(O_{2}\), or microwave monitors in the CXLS suite are tripped.
In the situation where any of these monitors are alarming the interlock system will trip and put the transmitters into a safe state.

\sphinxAtStartPar
The ACCELERATOR section of the protocase is for viewing and changing the arming status of the accelerator and both transmitters.
The accelerator cannot be armed until Vault\sphinxhyphen{}1 is secured, and the transmitters cannot be armed until the accelerator is armed.
This section of the protocase can also be used to override the interlocks on the transmitters for maintenance.

\begin{figure}[htbp]
\centering
\capstart

\noindent\sphinxincludegraphics[scale=0.2]{{Vault-1_protocase}.jpg}
\caption{\sphinxstylestrong{Figure 1:} This is the Vault\sphinxhyphen{}1 Control Ionizing Radiation Interlock Protocase. In this state Vault\sphinxhyphen{}1 is not secured, and neither the accelerator or transmitters are armed.
As well, there are no area monitors alarming or failing.}\label{\detokenize{user_documentation/Vault-1_ionizing_radiation:id1}}\end{figure}


\subsubsection{VIEWMARQ displays}
\label{\detokenize{user_documentation/Vault-1_ionizing_radiation:viewmarq-displays}}
\sphinxAtStartPar
There are two VIEWMARQ displays that share information on Vault\sphinxhyphen{}1 ionizing radiation hazard status.
One in the Accelerator Lab to the right of the RF\sphinxhyphen{}1 door, and the other in Vault\sphinxhyphen{}1 Control above the Vault\sphinxhyphen{}1 door.
See Figure 2 and 3 for the Vault\sphinxhyphen{}1 Control and Accelerator Lab VIEWMARQ displays respectively.


\begin{savenotes}\sphinxattablestart
\centering
\begin{tabular}[t]{|*{2}{\X{1}{2}|}}
\hline
\sphinxstyletheadfamily 
\sphinxAtStartPar
VIEWMARQ Display Notes
&\sphinxstyletheadfamily 
\sphinxAtStartPar
VIEWMARQ Display Text
\\
\hline
\begin{DUlineblock}{0em}
\item[] The VIEWMARQ display in Vault\sphinxhyphen{}1 Control shows \DUrole{green}{LASER SAFE}
\item[] because this display also shows laser hazards present in Vault\sphinxhyphen{}1.
\item[] However, in this case \DUrole{green}{LASER SAFE} means that it is \DUrole{green}{RF SAFE} as well.
\item[] The display in Accelerator Lab displays \DUrole{green}{RF SAFE} when neither
\item[] transmitter is armed.
\end{DUlineblock}
&
\sphinxAtStartPar
\DUrole{green}{Laser Safe} / \DUrole{green}{RF Safe}
\\
\hline
\begin{DUlineblock}{0em}
\item[] Both VIEWMARQ displays show \DUrole{red}{VAULT SECURE \sphinxhyphen{} RF ARMED} once
\item[] Vault\sphinxhyphen{}1 is searched, secured, the accelerator is armed, and either one
\item[] or both transmitters are armed. Additional laser hazards will appear
\item[] here as well. See Vault\sphinxhyphen{}1 laser system manual for hazards.
\end{DUlineblock}
&
\sphinxAtStartPar
\DUrole{red}{Vault\sphinxhyphen{}1 Secure \sphinxhyphen{} RF Armed}
\\
\hline
\begin{DUlineblock}{0em}
\item[] Both VIEWMARQ displays show \DUrole{red}{IONIZING RADIATION E\sphinxhyphen{}STOP}
\item[] \DUrole{red}{ACTIVE} when any ionizing radiation e\sphinxhyphen{}stop in the CXLS suite is pressed.
\end{DUlineblock}
&
\begin{DUlineblock}{0em}
\item[] \DUrole{red}{Ionizing Radiation}
\item[] \DUrole{red}{E\sphinxhyphen{}Stop Activated}
\end{DUlineblock}
\\
\hline
\end{tabular}
\par
\sphinxattableend\end{savenotes}


\begin{savenotes}\sphinxattablestart
\centering
\begin{tabulary}{\linewidth}[t]{|T|T|T|}
\hline

\noindent{\hspace*{\fill}\sphinxincludegraphics[scale=0.28]{{Vault-1_Control_VIEWMARQ_safe}.jpg}\hspace*{\fill}}
&
\noindent{\hspace*{\fill}\sphinxincludegraphics[scale=0.28]{{Vault-1_Control_VIEWMARQ_armed}.jpg}\hspace*{\fill}}
&
\noindent{\hspace*{\fill}\sphinxincludegraphics[scale=0.28]{{Vault-1_Control_VIEWMARQ_e-stop}.jpg}\hspace*{\fill}}
\\
\hline
\sphinxAtStartPar
Vault\sphinxhyphen{}1 RF safe condition. \DUrole{white-cell}{============================================================}
&
\sphinxAtStartPar
Vault\sphinxhyphen{}1 RF hazard condition. \DUrole{white-cell}{==========================================================}
&
\sphinxAtStartPar
Vault\sphinxhyphen{}1 ionizing radiation e\sphinxhyphen{}stop. \DUrole{white-cell}{====================================================}
\\
\hline
\end{tabulary}
\par
\sphinxattableend\end{savenotes}

\begin{sphinxuseclass}{tight-table-caption-container}
\sphinxAtStartPar
\sphinxstylestrong{Figure 2:} This is the Vault\sphinxhyphen{}1 Control VIEWMARQ display under all 3 RF conditions.

\end{sphinxuseclass}

\begin{savenotes}\sphinxattablestart
\centering
\begin{tabulary}{\linewidth}[t]{|T|T|T|}
\hline

\noindent{\hspace*{\fill}\sphinxincludegraphics[scale=0.2]{{Accelerator_lab_VIEWMARQ_safe}.jpg}\hspace*{\fill}}
&
\noindent{\hspace*{\fill}\sphinxincludegraphics[scale=0.2]{{Accelerator_lab_VIEWMARQ_armed}.jpg}\hspace*{\fill}}
&
\noindent{\hspace*{\fill}\sphinxincludegraphics[scale=0.2]{{Accelerator_lab_VIEWMARQ_e-stop}.jpg}\hspace*{\fill}}
\\
\hline
\sphinxAtStartPar
Accelerator Lab RF safe condition. \DUrole{white-cell}{======================================================}
&
\sphinxAtStartPar
Accelerator Lab RF hazard condition. \DUrole{white-cell}{====================================================}
&
\sphinxAtStartPar
Accelerator Lab ionizing radiation e\sphinxhyphen{}stop. \DUrole{white-cell}{==============================================}
\\
\hline
\end{tabulary}
\par
\sphinxattableend\end{savenotes}

\begin{sphinxuseclass}{tight-table-caption-container}
\sphinxAtStartPar
\sphinxstylestrong{Figure 3:} This is the Accelerator Lab VIEWMARQ display under all 3 RF conditions.

\end{sphinxuseclass}

\subsubsection{Beacons}
\label{\detokenize{user_documentation/Vault-1_ionizing_radiation:beacons}}
\sphinxAtStartPar
There are blue, red, and orange beacons in Vault\sphinxhyphen{}1 Control and Accelerator Lab to the left of the VIEWMARQ displays.
Specifically, they are the individual beacon modules, not the stacked units, which can be seen in figure 4.
The stacked units correspond to the state of the Vault\sphinxhyphen{}1 laser interlock system.


\begin{savenotes}\sphinxattablestart
\centering
\begin{tabulary}{\linewidth}[t]{|T|T|T|}
\hline

\noindent{\hspace*{\fill}\sphinxincludegraphics[scale=1.1]{{Vault-1_Control_beacons1}.jpg}\hspace*{\fill}}
&
\noindent{\hspace*{\fill}\sphinxincludegraphics[scale=1.12]{{Accelerator_lab_beacons}.jpg}\hspace*{\fill}}
&
\noindent{\hspace*{\fill}\sphinxincludegraphics[scale=0.82]{{Protocase_beacon}.jpg}\hspace*{\fill}}
\\
\hline
\sphinxAtStartPar
Vault\sphinxhyphen{}1 Control beacons. \DUrole{white-cell}{==============================================================}
&
\sphinxAtStartPar
Accelerator Lab beacons. \DUrole{white-cell}{==============================================================}
&
\sphinxAtStartPar
Vault\sphinxhyphen{}1 Control protocase beacon. \DUrole{white-cell}{=====================================================}
\\
\hline
\end{tabulary}
\par
\sphinxattableend\end{savenotes}

\begin{sphinxuseclass}{tight-table-caption-container}
\sphinxAtStartPar
\sphinxstylestrong{Figure 4:} These are the Vault\sphinxhyphen{}1 Control and Accelerator Lab beacons.

\end{sphinxuseclass}

\begin{savenotes}\sphinxattablestart
\centering
\begin{tabular}[t]{|*{2}{\X{1}{2}|}}
\hline
\sphinxstyletheadfamily 
\sphinxAtStartPar
Status
&\sphinxstyletheadfamily 
\sphinxAtStartPar
Beacon Color
\\
\hline
\sphinxAtStartPar
The \DUrole{blue}{blue} beacon indicates that RF has been enabled into the Vault\sphinxhyphen{}1 structures.
&
\sphinxAtStartPar
\DUrole{blue-cell}{Beacon Color}
\\
\hline
\begin{DUlineblock}{0em}
\item[] The \DUrole{red}{red} beacon indicated that an ionizing radiation emergency stop button had been
\item[] pressed. This beacon is on the wall and on the protocase.
\end{DUlineblock}
&
\sphinxAtStartPar
\DUrole{red-cell}{Beacon Color}
\\
\hline
\begin{DUlineblock}{0em}
\item[] The \DUrole{orange}{orange} beacon indicates that one of the O2 meters is reading below 19\% \(O_{2}\)
\item[] levels.
\end{DUlineblock}
&
\sphinxAtStartPar
\DUrole{orange-cell}{Beacon Color}
\\
\hline
\end{tabular}
\par
\sphinxattableend\end{savenotes}


\subsubsection{O2 Main and Remote Units}
\label{\detokenize{user_documentation/Vault-1_ionizing_radiation:o2-main-and-remote-units}}
\sphinxAtStartPar
There are two O2 sensors in the Vault\sphinxhyphen{}1 ionizing radiation interlock system.
One is located in Vault\sphinxhyphen{}1 and the other is located in RF\sphinxhyphen{}1.
If alarming, these units will sound an alarm and flash one of the AL\# LEDs depending on the alarm set point it passed.
Any \(O_{2}\) reading below 19\% will cause the sensors to alarm, passing the AL1 set point.
Each O2 sensor has a remote unit that has controls and displays information from the main unit, but does not have its own dedicated sensor.
The Vault\sphinxhyphen{}1 remote unit is in Vault\sphinxhyphen{}1 Control and the RF\sphinxhyphen{}1 remote unit is in the Accelerator Lab.


\begin{savenotes}\sphinxattablestart
\centering
\begin{tabulary}{\linewidth}[t]{|T|T|}
\hline

\noindent{\hspace*{\fill}\sphinxincludegraphics[scale=0.2]{{Vault-1_O2_main}.jpg}\hspace*{\fill}}
&
\noindent{\hspace*{\fill}\sphinxincludegraphics[scale=0.2]{{Vault-1_O2_remote}.jpg}\hspace*{\fill}}
\\
\hline
\sphinxAtStartPar
O2 main unit. \DUrole{white-cell}{=====================================================================}
&
\sphinxAtStartPar
O2 remote unit. \DUrole{white-cell}{===================================================================}
\\
\hline
\end{tabulary}
\par
\sphinxattableend\end{savenotes}

\begin{sphinxuseclass}{tight-table-caption-container}
\sphinxAtStartPar
\sphinxstylestrong{Figure 5:} This is the O2 sensor pair.

\end{sphinxuseclass}

\subsubsection{Ionizing Radiation Monitor}
\label{\detokenize{user_documentation/Vault-1_ionizing_radiation:ionizing-radiation-monitor}}
\begin{sphinxadmonition}{note}{Note:}
\sphinxAtStartPar
The ionizing radiation monitor may go through changes in the near future.
This section will be updated when those changes are made.
\end{sphinxadmonition}


\subsection{Ionizing Radiation Emergency Stop Buttons}
\label{\detokenize{user_documentation/Vault-1_ionizing_radiation:ionizing-radiation-emergency-stop-buttons}}
\sphinxAtStartPar
Throughout the CXLS suite there are ionizing radiation emergency stop buttons.
These e\sphinxhyphen{}stop buttons will cut power to the transmitters, putting the accelerator in a safe state.
Once the transmitters are crashed, there will not longer be a source of ionizing radiation.
When an ionizing radiation e\sphinxhyphen{}stop button is pressed, the LED on the unit will turn on, all red beacons will turn on, and the VIEWMARQ displays will show \DUrole{red}{IONIZING RADIATION E\sphinxhyphen{}STOP ACTIVATED}.
To disengage the e\sphinxhyphen{}stop, rotate the button clockwise.

\sphinxAtStartPar
It is important to note that only the ionizing radiation emergency stop buttons will put the accelerator into a safe state.
There is also laser emergency stop buttons that will only cut power to their specific laser if armed and do not affect the transmitters.


\begin{savenotes}\sphinxattablestart
\centering
\begin{tabulary}{\linewidth}[t]{|T|T|}
\hline

\noindent{\hspace*{\fill}\sphinxincludegraphics[scale=0.2]{{Vault-1_estop_off}.jpg}\hspace*{\fill}}
&
\noindent{\hspace*{\fill}\sphinxincludegraphics[scale=0.2]{{Vault-1_estop_on}.jpg}\hspace*{\fill}}
\\
\hline
\sphinxAtStartPar
Ionizing radiation emergency stop button off. \DUrole{white-cell}{==============================================}
&
\sphinxAtStartPar
Ionizing radiation emergency stop button on. \DUrole{white-cell}{===============================================}
\\
\hline
\end{tabulary}
\par
\sphinxattableend\end{savenotes}

\begin{sphinxuseclass}{tight-table-caption-container}
\sphinxAtStartPar
\sphinxstylestrong{Figure 6:} This is the ionizing radiation emergency stop button in both states.

\end{sphinxuseclass}

\subsection{Search Procedure for Securing Vault\sphinxhyphen{}1}
\label{\detokenize{user_documentation/Vault-1_ionizing_radiation:search-procedure-for-securing-vault-1}}
\sphinxAtStartPar
To arm the accelerator and transmitters, Vault\sphinxhyphen{}1 must be searched and secured.
Starting at the west end of Vault\sphinxhyphen{}1 (down steam of the accelerator), while verifying the vault is empty, press the search button labeled 1.
As you continue to search and clear press 2 then 3 as you’re working your way towards the vault entrance.
Once the 3rd search button is pressed, a chime will sound, a timer will start, and all the SECURE PERIMETER SEARCH lamps on the Vault\sphinxhyphen{}1 Control IONIZING RADIATION INTERLOCK protocase will be \DUrole{green}{green}.
If the search buttons are pressed out of order, or the search takes too long, the search will need to be restarted.

\begin{figure}[htbp]
\centering
\capstart

\noindent\sphinxincludegraphics[scale=0.35]{{Vault1_Search_Buttons}.png}
\caption{\sphinxstylestrong{Figure 7:} This is a diagram of the Vault\sphinxhyphen{}1 search buttons. The numbers indicate the order in which they need to be pressed.}\label{\detokenize{user_documentation/Vault-1_ionizing_radiation:id2}}\end{figure}


\begin{savenotes}\sphinxattablestart
\centering
\begin{tabulary}{\linewidth}[t]{|T|T|}
\hline

\noindent{\hspace*{\fill}\sphinxincludegraphics[scale=0.2]{{Vault-1_search_off}.jpg}\hspace*{\fill}}
&
\noindent{\hspace*{\fill}\sphinxincludegraphics[scale=0.2]{{Vault-1_search_on}.jpg}\hspace*{\fill}}
\\
\hline
\sphinxAtStartPar
Vault\sphinxhyphen{}1 search button off. \DUrole{white-cell}{============================================================}
&
\sphinxAtStartPar
Vault\sphinxhyphen{}1 search button on. \DUrole{white-cell}{=============================================================}
\\
\hline
\end{tabulary}
\par
\sphinxattableend\end{savenotes}

\begin{sphinxuseclass}{tight-table-caption-container}
\sphinxAtStartPar
\sphinxstylestrong{Figure 8:} This is the Vault\sphinxhyphen{}1 search button in both states.

\end{sphinxuseclass}
\begin{figure}[htbp]
\centering
\capstart

\noindent\sphinxincludegraphics[scale=0.2]{{Vault-1_searched}.jpg}
\caption{\sphinxstylestrong{Figure 9:} This is the Vault\sphinxhyphen{}1 Control Ionizing Radiation Protocase when all searched buttons have been pressed in the correct order.}\label{\detokenize{user_documentation/Vault-1_ionizing_radiation:id3}}\end{figure}

\sphinxAtStartPar
Holding down the CLOSE button to the right of the protocase, close the shield door up to the yellow and black tape but not covering it.
Once the door is fully closed and actuating the door switches the SHIELD DOOR lamp on the Vault\sphinxhyphen{}1 Control IONIZING RADIATION INTERLOCK protocase will be \DUrole{green}{green}.

\begin{figure}[htbp]
\centering
\capstart

\noindent\sphinxincludegraphics[scale=0.2]{{Vault-1_door_buttons}.jpg}
\caption{\sphinxstylestrong{Figure 10:} These are the Vault\sphinxhyphen{}1 shield door control buttons.}\label{\detokenize{user_documentation/Vault-1_ionizing_radiation:id4}}\end{figure}

\begin{figure}[htbp]
\centering
\capstart

\noindent\sphinxincludegraphics[scale=0.2]{{Vault-1_door}.jpg}
\caption{\sphinxstylestrong{Figure 11:} This is the Vault\sphinxhyphen{}1 Control Ionizing Radiation Protocase when the shield door is closed.}\label{\detokenize{user_documentation/Vault-1_ionizing_radiation:id5}}\end{figure}


\subsection{Arming the Accelerator and Transmitters}
\label{\detokenize{user_documentation/Vault-1_ionizing_radiation:arming-the-accelerator-and-transmitters}}

\subsubsection{Non\sphinxhyphen{}Armable States}
\label{\detokenize{user_documentation/Vault-1_ionizing_radiation:non-armable-states}}
\sphinxAtStartPar
Besides Vault\sphinxhyphen{}1 not being secured, if any of the area monitors are alarming or failing the respective AREA MONITOR lamp will turn \DUrole{red}{red} and the accelerator will not arm.
If the accelerator is already armed and either of these states change, the accelerator will disarm.


\subsubsection{Arming Procedure}
\label{\detokenize{user_documentation/Vault-1_ionizing_radiation:arming-procedure}}
\sphinxAtStartPar
Once Vault\sphinxhyphen{}1 is secured the accelerator can be armed.
To arm the accelerator, turn the ACCELERATOR ENABLE key on the Vault\sphinxhyphen{}1 IONIZING RADIATION INTERLOCK protocase.
The STATUS lamp will turn \DUrole{green}{green}. Now that the accelerator is armed, the transmitters can be armed.

\begin{figure}[htbp]
\centering
\capstart

\noindent\sphinxincludegraphics[scale=0.2]{{Vault-1_protocase_accelerator_armed}.jpg}
\caption{\sphinxstylestrong{Figure 12:} This is the Vault\sphinxhyphen{}1 Control Ionizing Radiation Protocase when the accelerator is armed.}\label{\detokenize{user_documentation/Vault-1_ionizing_radiation:id6}}\end{figure}

\sphinxAtStartPar
Like the accelerator, to arm the individual transmitters turn the TRANSMITTER ENABLE key on the Vault\sphinxhyphen{}1 Control IONIZING RADIATION INTERLOCK protocase.
The STATUS lamp will turn \DUrole{green}{green} for the transmitter you armed.
Once either of the transmitters are armed the VIEWMARQ displays in Vault\sphinxhyphen{}1 Control and Accelerator Lab will display \DUrole{red}{VAULT SECURE \sphinxhyphen{} RF ARMED} and the \DUrole{blue}{blue} beacons next to the displays will be on.
At this state the transmitters can be set to trig and power can be enabled into the RF structures.

\sphinxAtStartPar
The accelerator and transmitters can be disarmed by pressing the ACCELERATOR RESET button on the Vault\sphinxhyphen{}1 Control IONIZING RADIATION INTERLOCK protocase.


\begin{savenotes}\sphinxattablestart
\centering
\begin{tabulary}{\linewidth}[t]{|T|T|T|}
\hline

\noindent{\hspace*{\fill}\sphinxincludegraphics[scale=0.2]{{Vault-1_protocase_transmitter_armed_1}.jpg}\hspace*{\fill}}
&
\noindent{\hspace*{\fill}\sphinxincludegraphics[scale=0.2]{{Vault-1_protocase_transmitter_armed_2}.jpg}\hspace*{\fill}}
&
\noindent{\hspace*{\fill}\sphinxincludegraphics[scale=0.2]{{Vault-1_protocase_transmitter_armed_both}.jpg}\hspace*{\fill}}
\\
\hline
\sphinxAtStartPar
Vault\sphinxhyphen{}1 Control Ionizing Radiation Protocase when transmitter 1 is armed. \DUrole{white-cell}{========================}
&
\sphinxAtStartPar
Vault\sphinxhyphen{}1 Control Ionizing Radiation Protocase when transmitter 2 is armed. \DUrole{white-cell}{========================}
&
\sphinxAtStartPar
Vault\sphinxhyphen{}1 Control Ionizing Radiation Protocase when both transmitters are armed. \DUrole{white-cell}{===================}
\\
\hline
\end{tabulary}
\par
\sphinxattableend\end{savenotes}

\begin{sphinxuseclass}{tight-table-caption-container}
\sphinxAtStartPar
\sphinxstylestrong{Figure 13:} This is the Vault\sphinxhyphen{}1 Control IONIZING RADIATION INTERLOCK protocase when transmitter 1 is armed.

\end{sphinxuseclass}

\subsection{Putting Vault\sphinxhyphen{}1 into a Non\sphinxhyphen{}Secure State}
\label{\detokenize{user_documentation/Vault-1_ionizing_radiation:putting-vault-1-into-a-non-secure-state}}
\sphinxAtStartPar
Once the transmitters are no longer triggering, the accelerator and transmitters can be disarmed.
This can be done by pressing the ACCELERATOR RESET button on the Vault\sphinxhyphen{}1 Control IONIZING RADIATION INTERLOCK protocase, where all \DUrole{green}{green} STATUS lamps will turn \DUrole{red}{red}.
This will keep Vault\sphinxhyphen{}1 in a secure state while disarming the accelerator and transmitters.
To put Vault\sphinxhyphen{}1 into a non\sphinxhyphen{}secure state, simply opening the shield door will disarm the system and turn all \DUrole{green}{green} STATUS and PERIMETER lamps will turn \DUrole{red}{red}.

\begin{sphinxadmonition}{note}{Note:}
\sphinxAtStartPar
2 minutes must pass from the transmitters being brought to a safe state and the accelerator being disarmed before the Vault\sphinxhyphen{}1 door can be opened.
\end{sphinxadmonition}


\subsection{Vault\sphinxhyphen{}1 Radiation Survey Procedure}
\label{\detokenize{user_documentation/Vault-1_ionizing_radiation:vault-1-radiation-survey-procedure}}
\sphinxAtStartPar
For Vault\sphinxhyphen{}1 to be cleared for open entry, it must first be surveyed for ionizing radiation.


\begin{savenotes}\sphinxattablestart
\centering
\begin{tabulary}{\linewidth}[t]{|T|T|}
\hline

\noindent{\hspace*{\fill}\sphinxincludegraphics[scale=1.2]{{dosimeter}.png}\hspace*{\fill}}
&
\noindent{\hspace*{\fill}\sphinxincludegraphics[scale=1.2]{{dosimeter_board}.jpg}\hspace*{\fill}}
\\
\hline
\sphinxAtStartPar
Personal dosimeter. \DUrole{white-cell}{================================================================}
&
\sphinxAtStartPar
Dosimeter storage board. \DUrole{white-cell}{===========================================================}
\\
\hline
\end{tabulary}
\par
\sphinxattableend\end{savenotes}

\begin{sphinxuseclass}{tight-table-caption-container}
\sphinxAtStartPar
\sphinxstylestrong{Figure 14:} This is the personal dosimeter and the dosimeter storage board. Your personal dosimeter should be worn at all time during the operation of the CXLS electron beam. If your dosimeter is not on your person, it should be on the dosimeter storage board, located in the corridor outside of Hutch Control / Experiment Prep entrance.

\end{sphinxuseclass}
\sphinxAtStartPar
Once the two minutes have elapsed, the Vault\sphinxhyphen{}1 can be opened, and the survey can be performed.
The surveyor, along with his personal dosimeter, must also wear a electronic personal dosimeter.
This unit will alarm if the surveyor is exposed to more than 5 mrem/hr.


\begin{savenotes}\sphinxattablestart
\centering
\begin{tabulary}{\linewidth}[t]{|T|T|T|}
\hline

\noindent{\hspace*{\fill}\sphinxincludegraphics[scale=1.2]{{Ludlum_23}.png}\hspace*{\fill}}
&
\noindent{\hspace*{\fill}\sphinxincludegraphics[scale=1.2]{{wearing_epd}.png}\hspace*{\fill}}
&
\noindent{\hspace*{\fill}\sphinxincludegraphics[scale=1.2]{{draw_holding_ludlum}.png}\hspace*{\fill}}
\\
\hline
\sphinxAtStartPar
Ludlum 23 electronic personal dosimeter. \DUrole{white-cell}{================================================}
&
\sphinxAtStartPar
Wearing the electronic personal dosimeter. \DUrole{white-cell}{==============================================}
&
\sphinxAtStartPar
Draw holding the Ludlum 23. \DUrole{white-cell}{=============================================================}
\\
\hline
\end{tabulary}
\par
\sphinxattableend\end{savenotes}

\begin{sphinxuseclass}{tight-table-caption-container}
\sphinxAtStartPar
\sphinxstylestrong{Figure 15:} This is the Ludlum 23 electronic personal dosimeter, how it is to be worn, and the draw holding the Ludlum 23. This draw holding the units is located at the desk to teh left when entering the Accelerator Lab.

\end{sphinxuseclass}
\sphinxAtStartPar
To perform the survey, the Ludlum 9DP is used to measure the gamma dose rate.
Once Vault\sphinxhyphen{}1 shield door is opened, they surveyor should slowly enter, watching the readings.
Go down the beam line, slowly scanning around inch away from the beam line and fill in the survey sheet.
If any element reads above 20 µR/hr, scan from 30 cm away to verify the general area is not above background from normal viewing distance.


\begin{savenotes}\sphinxattablestart
\centering
\begin{tabulary}{\linewidth}[t]{|T|T|}
\hline

\noindent{\hspace*{\fill}\sphinxincludegraphics[scale=1.2]{{Ludlum_9DP}.png}\hspace*{\fill}}
&
\noindent{\hspace*{\fill}\sphinxincludegraphics[scale=1.2]{{cabinet_holding_ludlum}.jpg}\hspace*{\fill}}
\\
\hline
\sphinxAtStartPar
Ludlum 9DP pressurized ionization chamber. \DUrole{white-cell}{==============================================}
&
\sphinxAtStartPar
Cabinet holding the Ludlum 9DP. \DUrole{white-cell}{=========================================================}
\\
\hline
\end{tabulary}
\par
\sphinxattableend\end{savenotes}

\begin{sphinxuseclass}{tight-table-caption-container}
\sphinxAtStartPar
\sphinxstylestrong{Figure 16:} This is the Ludlum 9DP pressurized ionization chamber and the cabinet holding the Ludlum 9DP.

\end{sphinxuseclass}
\sphinxAtStartPar
Once the Vault\sphinxhyphen{}1 radiation survey is completed, and it is verified that there are no elevated levels of ionizing radiation, Vault\sphinxhyphen{}1 can be entered by anyone.
Enter the readings into the designated spread sheet and sign the survey sheet.


\subsection{Overriding the Transmitters to Work in an Armed State}
\label{\detokenize{user_documentation/Vault-1_ionizing_radiation:overriding-the-transmitters-to-work-in-an-armed-state}}
\sphinxAtStartPar
When the transmitters are armed, attempting to remove the side panels for maintenance will cause the transmitters to lose power.
If work needs to be done on the transmitters in an armed state, you must override the interlocks on the transmitters.
To do this turn the OVERRIDE key on the Vault\sphinxhyphen{}1 Control IONIZING RADIATION INTERLOCK protocase.
The STATUS lamp for the transmitter in override will turn \DUrole{orange}{orange}.
In this state, working on the armed transmitters will not cause the interlocks to trip.


\begin{savenotes}\sphinxattablestart
\centering
\begin{tabulary}{\linewidth}[t]{|T|T|}
\hline

\noindent{\hspace*{\fill}\sphinxincludegraphics[scale=0.2]{{Vault-1_protocase_transmitter_override_2}.jpg}\hspace*{\fill}}
&
\noindent{\hspace*{\fill}\sphinxincludegraphics[scale=0.2]{{Vault-1_protocase_transmitter_override_both}.jpg}\hspace*{\fill}}
\\
\hline
\sphinxAtStartPar
Vault\sphinxhyphen{}1 Control Ionizing Radiation Protocase when a transmitter is in override. \DUrole{white-cell}{======================}
&
\sphinxAtStartPar
Vault\sphinxhyphen{}1 Control Ionizing Radiation Protocase when both transmitters are in override. \DUrole{white-cell}{=================}
\\
\hline
\end{tabulary}
\par
\sphinxattableend\end{savenotes}

\begin{sphinxuseclass}{tight-table-caption-container}
\sphinxAtStartPar
\sphinxstylestrong{Figure 17:} This is the Vault\sphinxhyphen{}1 Control IONIZING RADIATION INTERLOCK protocase in an override state.

\end{sphinxuseclass}
\sphinxstepscope


\section{Vault\sphinxhyphen{}1 Laser Interlock System User Manual}
\label{\detokenize{user_documentation/Vault-1_laser:vault-1-laser-interlock-system-user-manual}}\label{\detokenize{user_documentation/Vault-1_laser::doc}}
\sphinxAtStartPar
This document provides a user overview for the Vault\sphinxhyphen{}1 Laser Interlock System.
It covers the laser modules, hazard indicators, control protocases, VIEWMARQ display, and the beacons in Vault\sphinxhyphen{}1 and Vault\sphinxhyphen{}1 Control.
The laser modules are used for indicating the status of the laser interlock system and arming the laser systems.
The control protocase allows the users to view the override state of the laser enclosure and control the manual shutter control.
The VIEWMARQ display provides a quick overview of the laser hazard status in Vault\sphinxhyphen{}1.
The beacons are used to indicate the state of laser hazards in Vault\sphinxhyphen{}1 and the laser enclosures.


\subsection{Vault\sphinxhyphen{}1 Laser Hazard Indicators}
\label{\detokenize{user_documentation/Vault-1_laser:vault-1-laser-hazard-indicators}}
\sphinxAtStartPar
This section will cover the laser hazard indicators in Vault\sphinxhyphen{}1 Control and Vault\sphinxhyphen{}1.


\subsubsection{Pharos and Dira Enclosure Protocases}
\label{\detokenize{user_documentation/Vault-1_laser:pharos-and-dira-enclosure-protocases}}
\sphinxAtStartPar
The Pharos and Dira LASER ENCLOSURE INTERLOCK protocases are a interface to view and change is the enclosures are in an override state, view the state of the enclosure doors and laser hazard, and arm the local interlock modules for manual shutter control.

\sphinxAtStartPar
The PERIMETER section of the protocases has a laser warning module and door monitor module.
The laser warning module displays if the respective laser is forced into a safe state.
This means it will always show \DUrole{red}{DANGER LASER HAZARD} unless the interlocks are tripped.
The Dira enclosure laser warning module will not show \DUrole{red}{DANGER LASER HAZARD} until Laser\sphinxhyphen{}1 is armed.
The door monitor shows if the enclosure doors are opened or closed, showing \DUrole{green}{CLOSED} or nothing.
If the enclosure is put into an override state, then the monitor will always show \DUrole{green}{CLOSED} because the interlocks are bypassed.

\sphinxAtStartPar
The LOCAL INTERLOCK CONTACT CONTROL section of the protocases has two local interlock modules.
These modules are for arming the UV and IV shutter controllers for manual operation.
These can only be armed when the enclosure is armed.
However, if the enclosure is not set to override, then these will disarm when opening the enclosure.
The labeling of the Pharos and Dira protocases differs here.
On the Pharos protocase the local interlock modules are labeled CONTACT \#1 AND CONTACT \#2, while on the Dira enclosure they are labeled AUX \#1 and AUX \#2.
Though they are labeled differently, they function the same.

\sphinxAtStartPar
The INTERLOCK OVERRIDE section of the protocases shows the status of the enclosure interlocks.
If the key is set to OVERRIDE and the STATUS LED is \DUrole{red}{red}, that means that the enclosure interlocks are bypass, and there could be a laser hazard if the enclosure is opened.

\begin{figure}[htbp]
\centering
\capstart

\noindent\sphinxincludegraphics[scale=0.2]{{Pharos_protocase1}.jpg}
\caption{\sphinxstylestrong{Figure 1:} Vault\sphinxhyphen{}1 Pharos enclosure protocase.
It is located on the east wall of the Pharos enclosure.}\label{\detokenize{user_documentation/Vault-1_laser:id1}}\end{figure}

\begin{figure}[htbp]
\centering
\capstart

\noindent\sphinxincludegraphics[scale=0.2]{{Dira_protocase1}.jpg}
\caption{\sphinxstylestrong{Figure 2:} Vault\sphinxhyphen{}1 Dira enclosure protocase.
It is located on the east wall of the Dira enclosure.}\label{\detokenize{user_documentation/Vault-1_laser:id2}}\end{figure}


\subsubsection{Beacon Stacks}
\label{\detokenize{user_documentation/Vault-1_laser:beacon-stacks}}
\sphinxAtStartPar
There are beacon stacks in Vault\sphinxhyphen{}1 Control, on the Vault\sphinxhyphen{}1 east wall, on the Pharos LASER ENCLOSURE INTERLOCK protocase, and on the Dira LASER ENCLOSURE INTERLOCK protocase.
The beacons stacks can notify you of the arming status for Vault\sphinxhyphen{}1, the Pharos enclosure, and the Dira enclosure.
Also, the beacon stacks can notify you if there is an enclosure whose interlocks are in administrative override.


\begin{savenotes}\sphinxattablestart
\centering
\begin{tabulary}{\linewidth}[t]{|T|T|T|T|}
\hline

\noindent{\hspace*{\fill}\sphinxincludegraphics[scale=0.76]{{Vault-1_Control_beacons}.jpg}\hspace*{\fill}}
&
\noindent{\hspace*{\fill}\sphinxincludegraphics[scale=0.2]{{Vault-1_beacons}.jpg}\hspace*{\fill}}
&
\noindent{\hspace*{\fill}\sphinxincludegraphics[scale=0.43]{{Pharos_beacons}.jpg}\hspace*{\fill}}
&
\noindent{\hspace*{\fill}\sphinxincludegraphics[scale=0.53]{{Dira_beacons}.jpg}\hspace*{\fill}}
\\
\hline
\sphinxAtStartPar
Vault\sphinxhyphen{}1 Control beacon stack. \DUrole{white-cell}{=========================================================}
&
\sphinxAtStartPar
Vault\sphinxhyphen{}1 beacon stack. \DUrole{white-cell}{=================================================================}
&
\sphinxAtStartPar
Pharos LASER ENCLOSURE INTERLOCK protocase beacon stack. \DUrole{white-cell}{==============================}
&
\sphinxAtStartPar
Dira LASER ENCLOSURE INTERLOCK protocase beacon stack. \DUrole{white-cell}{================================}
\\
\hline
\end{tabulary}
\par
\sphinxattableend\end{savenotes}

\begin{sphinxuseclass}{tight-table-caption-container}
\sphinxAtStartPar
\sphinxstylestrong{Table 3:} These are the Vault\sphinxhyphen{}1 laser interlock system beacon stacks.

\end{sphinxuseclass}

\begin{savenotes}\sphinxattablestart
\centering
\begin{tabular}[t]{|*{2}{\X{1}{2}|}}
\hline
\sphinxstyletheadfamily 
\sphinxAtStartPar
Status
&\sphinxstyletheadfamily 
\sphinxAtStartPar
Beacon Color
\\
\hline
\begin{DUlineblock}{0em}
\item[] Vault\sphinxhyphen{}1 is not armed as a laser lab. Vault\sphinxhyphen{}1 is laser safe.
\end{DUlineblock}
&
\sphinxAtStartPar
\DUrole{green-cell}{Beacon Color}
\\
\hline
\begin{DUlineblock}{0em}
\item[] Either the Pharos or Dira enclosures are set to administrative override.
\item[] This state is only possible if Vault\sphinxhyphen{}1 is armed.
\end{DUlineblock}
&
\sphinxAtStartPar
\DUrole{orange-cell}{Beacon Color}
\\
\hline
\begin{DUlineblock}{0em}
\item[] The Dira is armed. This state is possible with or without Vault\sphinxhyphen{}1 being
\item[] armed.
\end{DUlineblock}
&
\sphinxAtStartPar
\DUrole{white-cell}{Beacon Color}
\\
\hline
\begin{DUlineblock}{0em}
\item[] The Pharos is armed. This state is possible with or without Vault\sphinxhyphen{}1
\item[] being armed.
\end{DUlineblock}
&
\sphinxAtStartPar
\DUrole{blue-cell}{Beacon Color}
\\
\hline
\end{tabular}
\par
\sphinxattableend\end{savenotes}

\sphinxAtStartPar
For the enclosure specific beacon stacks, the \DUrole{orange}{orange} administrative override LED will only light if that specific enclosure is in override.
The general Vault\sphinxhyphen{}1 and Vault\sphinxhyphen{}1 Control beacon stacks will light the administrative override LED if either of the enclosures are in override.

\sphinxAtStartPar
The Pharos LASER ENCLOSURE INTERLOCK protocase is the only protocase without a white beacon.
That is because the state of the Dira does not affect the state inside of the Pharos enclosure.
However, the Dira LASER ENCLOSURE INTERLOCK protocase has a blue beacon because the Pharos exports a beam into the Dira enclosure, so the state of the Pharos affects the Dira enclosure.


\subsubsection{VIEWMARQ Display}
\label{\detokenize{user_documentation/Vault-1_laser:viewmarq-display}}
\sphinxAtStartPar
There is a VIEWMARQ display in Vault\sphinxhyphen{}1 Control that states the status of potential laser hazards in Vault\sphinxhyphen{}1.
This display will notify you if Vault\sphinxhyphen{}1, the Pharos, or the Dira is armed.
Also, it will notify you if the Pharos, the Dira, or both laser enclosures are in administrative override.


\begin{savenotes}\sphinxattablestart
\centering
\begin{tabulary}{\linewidth}[t]{|T|T|T|T|}
\hline

\noindent{\hspace*{\fill}\sphinxincludegraphics[scale=0.2]{{Vault-1_VIEWMARQ_safe}.jpg}\hspace*{\fill}}
&
\noindent{\hspace*{\fill}\sphinxincludegraphics[scale=0.2]{{Vault-1_VIEWMARQ_laser_hazard1}.jpg}\hspace*{\fill}}
&
\noindent{\hspace*{\fill}\sphinxincludegraphics[scale=0.2]{{Vault-1_VIEWMARQ_Pharos_armed}.jpg}\hspace*{\fill}}
&
\noindent{\hspace*{\fill}\sphinxincludegraphics[scale=0.2]{{Vault-1_VIEWMARQ_all_armed}.jpg}\hspace*{\fill}}
\\
\hline
\sphinxAtStartPar
Vault\sphinxhyphen{}1 Control VIEWMARQ display when Vault\sphinxhyphen{}1 is not armed as a laser lab. \DUrole{white-cell}{======================}
&
\sphinxAtStartPar
Vault\sphinxhyphen{}1 Control VIEWMARQ display when Vault\sphinxhyphen{}1 is armed as a laser lab. \DUrole{white-cell}{==========================}
&
\sphinxAtStartPar
Vault\sphinxhyphen{}1 Control VIEWMARQ display when the Pharos is armed. \DUrole{white-cell}{======================================}
&
\sphinxAtStartPar
Vault\sphinxhyphen{}1 Control VIEWMARQ display when all hazards are armed. \DUrole{white-cell}{====================================}
\\
\hline
\end{tabulary}
\par
\sphinxattableend\end{savenotes}

\begin{sphinxuseclass}{tight-table-caption-container}
\sphinxAtStartPar
\sphinxstylestrong{Table 4:} These are the Vault\sphinxhyphen{}1 laser interlock system VIEWMARQ display examples.

\end{sphinxuseclass}

\begin{savenotes}\sphinxattablestart
\centering
\begin{tabular}[t]{|*{2}{\X{1}{2}|}}
\hline
\sphinxstyletheadfamily 
\sphinxAtStartPar
VIEWMARQ Display Notes
&\sphinxstyletheadfamily 
\sphinxAtStartPar
VIEWMARQ Display Text
\\
\hline
\begin{DUlineblock}{0em}
\item[] This states if Vault\sphinxhyphen{}1 is armed as a laser lab or not.
\end{DUlineblock}
&
\begin{DUlineblock}{0em}
\item[] \DUrole{green}{LASER SAFE} / \DUrole{green}{DANGER LASER HAZARD}
\end{DUlineblock}
\\
\hline
\begin{DUlineblock}{0em}
\item[] This states which laser is armed.
\end{DUlineblock}
&
\begin{DUlineblock}{0em}
\item[] \DUrole{red}{PHAROS ARMED            DIRA ARMED}
\end{DUlineblock}
\\
\hline
\begin{DUlineblock}{0em}
\item[] This states if the Dira is in administrative override.
\end{DUlineblock}
&
\begin{DUlineblock}{0em}
\item[] \DUrole{red}{DIRA ADMIN OVERRIDE}
\end{DUlineblock}
\\
\hline
\begin{DUlineblock}{0em}
\item[] This states you if the Pharos is in administrative override.
\end{DUlineblock}
&
\begin{DUlineblock}{0em}
\item[] \DUrole{red}{PHAROS ADMIN OVERRIDE}
\end{DUlineblock}
\\
\hline
\end{tabular}
\par
\sphinxattableend\end{savenotes}

\sphinxAtStartPar
The top line always will either display \DUrole{green}{LASER SAFE} or \DUrole{red}{DANGER LASER HAZARD}, assuming no RF hazards are present.
All other possible states will only appear on the display when the hazard is presented.


\subsubsection{Laser Safety System Modules}
\label{\detokenize{user_documentation/Vault-1_laser:laser-safety-system-modules}}
\sphinxAtStartPar
The laser interlock system is interfaced through the laser safety systems modules. Below is an outline of the modules and what they do.

\begin{figure}[htbp]
\centering
\capstart

\noindent\sphinxincludegraphics{{warning_module}.gif}
\caption{\sphinxstylestrong{Figure 5:} Area Warming Module}\label{\detokenize{user_documentation/Vault-1_laser:id3}}\end{figure}


\begin{savenotes}\sphinxattablestart
\centering
\begin{tabular}[t]{|*{2}{\X{1}{2}|}}
\hline
\sphinxstyletheadfamily 
\sphinxAtStartPar
Module Location
&\sphinxstyletheadfamily 
\sphinxAtStartPar
Module Meaning
\\
\hline
\begin{DUlineblock}{0em}
\item[] \sphinxstylestrong{General Area Module}
\item[] Vault\sphinxhyphen{}1 Control
\item[] Vault\sphinxhyphen{}1 Entry
\end{DUlineblock}
&
\begin{DUlineblock}{0em}
\item[] These are warning modules tell you if Vault\sphinxhyphen{}1 is armed as a laser lab.
\item[] \DUrole{red}{DANGER LASER ON} = ARMED
\end{DUlineblock}
\\
\hline
\begin{DUlineblock}{0em}
\item[] \sphinxstylestrong{Enclosure Modules}
\item[] Pharos enclosure south wall
\item[] Pharos enclosure west wall
\end{DUlineblock}
&
\begin{DUlineblock}{0em}
\item[] These warning modules tell you if the enclosure is armed.
\item[] There is no indication on if the laser is armed.
\item[] \DUrole{red}{DANGER LASER ON} = ARMED
\end{DUlineblock}
\\
\hline
\begin{DUlineblock}{0em}
\item[] \sphinxstylestrong{Protocase Modules}
\item[] Pharos enclosure protocase
\item[] Dira enclosure protocase
\end{DUlineblock}
&
\begin{DUlineblock}{0em}
\item[] These warning modules tell you if the enclosure is forced to a safe state.
\item[] \DUrole{red}{DANGER LASER HAZARD} = SAFE STATE IS NOT FORCED
\end{DUlineblock}
\\
\hline
\end{tabular}
\par
\sphinxattableend\end{savenotes}

\begin{figure}[htbp]
\centering
\capstart

\noindent\sphinxincludegraphics{{control_module}.gif}
\caption{\sphinxstylestrong{Figure 6:} Control Module}\label{\detokenize{user_documentation/Vault-1_laser:id4}}
\begin{sphinxlegend}
\sphinxAtStartPar
This module is a control module for the local laser interlock, however, for the users it serves as another warning module.
This warning module tells you if the room interlock is armed or not.
\end{sphinxlegend}
\end{figure}

\begin{figure}[htbp]
\centering
\capstart

\noindent\sphinxincludegraphics{{room_arm}.png}
\caption{\sphinxstylestrong{Figure 7:} Room Arm Module}\label{\detokenize{user_documentation/Vault-1_laser:id5}}
\begin{sphinxlegend}
\sphinxAtStartPar
This module is used to arm systems in the laser interlock system.
For example, there are two in Vault\sphinxhyphen{}1, one to arm the vault and one to arm the Pharos enclosure.
\end{sphinxlegend}
\end{figure}


\begin{savenotes}\sphinxattablestart
\centering
\begin{tabular}[t]{|*{2}{\X{1}{2}|}}
\hline
\sphinxstyletheadfamily 
\sphinxAtStartPar
Module Message
&\sphinxstyletheadfamily 
\sphinxAtStartPar
Message Meaning
\\
\hline
\sphinxAtStartPar
\DUrole{orange}{ROOM ARMED}
&
\begin{DUlineblock}{0em}
\item[] If this LED is on, then the corresponding system is armed
\item[] and interlocked.
\end{DUlineblock}
\\
\hline
\sphinxAtStartPar
\DUrole{green}{ROOM DISARMED (READY TO ARM)}
&
\begin{DUlineblock}{0em}
\item[] If this LED is on, the the system is the correct state to armed
\item[] the module.
\end{DUlineblock}
\\
\hline
\sphinxAtStartPar
\DUrole{orange}{ROOM CRASHED (CANNOT ARM)}
&
\begin{DUlineblock}{0em}
\item[] If this LED is on, then there was a fault that tripped the
\item[] system, or a fault that will not allow the system to be armed.
\end{DUlineblock}
\\
\hline
\end{tabular}
\par
\sphinxattableend\end{savenotes}

\begin{figure}[htbp]
\centering
\capstart

\noindent\sphinxincludegraphics{{local_arm}.png}
\caption{\sphinxstylestrong{Figure 8:} Local Arm Module}\label{\detokenize{user_documentation/Vault-1_laser:id6}}
\begin{sphinxlegend}
\sphinxAtStartPar
This module is used to arm the local interlock modules that are sub\sphinxhyphen{}systems of the room arm modules.
For example, one the Pharos enclosure is armed, it enables the laser and the shutters to be armed by their local arming modules.
\end{sphinxlegend}
\end{figure}


\begin{savenotes}\sphinxattablestart
\centering
\begin{tabular}[t]{|*{2}{\X{1}{2}|}}
\hline
\sphinxstyletheadfamily 
\sphinxAtStartPar
Module Message
&\sphinxstyletheadfamily 
\sphinxAtStartPar
Message Meaning
\\
\hline
\sphinxAtStartPar
\DUrole{orange}{LOCAL CONTACTS ARMED}
&
\begin{DUlineblock}{0em}
\item[] If this LED is on, then the corresponding sub\sphinxhyphen{}system is armed.
\end{DUlineblock}
\\
\hline
\sphinxAtStartPar
\DUrole{green}{LOCAL CONTACTS DISARMED}
&
\begin{DUlineblock}{0em}
\item[] If this LED is on, then the connected room module is armed,
\item[] but this module is not.
\end{DUlineblock}
\\
\hline
\begin{DUlineblock}{0em}
\item[] \DUrole{green}{LOCAL CONTACTS DISARMED}
\item[] \DUrole{green}{ROOM NOT ARMED}
\item[] \DUrole{green}{LOCAL CONTACT CANNOT ARM}
\end{DUlineblock}
&
\begin{DUlineblock}{0em}
\item[] If this LED is on, then there was a fault that tripped, or the room
\item[] module is not armed.
\end{DUlineblock}
\\
\hline
\end{tabular}
\par
\sphinxattableend\end{savenotes}

\begin{figure}[htbp]
\centering
\capstart

\noindent\sphinxincludegraphics{{push_to_exit}.png}
\caption{\sphinxstylestrong{Figure 9:} Push to Exit Module}\label{\detokenize{user_documentation/Vault-1_laser:id7}}
\begin{sphinxlegend}
\sphinxAtStartPar
When the rooms are armed, the doors are magnetically locked.
This button will temporarily unlock the door to allow you to exit the room.
\end{sphinxlegend}
\end{figure}

\begin{figure}[htbp]
\centering
\capstart

\noindent\sphinxincludegraphics{{key_pad}.jpg}
\caption{\sphinxstylestrong{Figure 10:} Keypad}\label{\detokenize{user_documentation/Vault-1_laser:id8}}
\begin{sphinxlegend}
\sphinxAtStartPar
This is the key pad that is used to enter a room that is armed as a laser lab.
This keypad has a primary pin for permanent users, and a secondary pin for temporary users that is meant to be changed frequently.
\end{sphinxlegend}
\end{figure}

\begin{figure}[htbp]
\centering
\capstart

\noindent\sphinxincludegraphics{{door_monitor}.jpg}
\caption{\sphinxstylestrong{Figure 11:} Door Monitor Module}\label{\detokenize{user_documentation/Vault-1_laser:id9}}
\begin{sphinxlegend}
\sphinxAtStartPar
This module is used to monitor the state of a door or curtain.
It will display does not show \DUrole{green}{CLOSED}, then it is open.
If the system is put into an administrative override state, then the door monitor will always show \DUrole{green}{CLOSED}.
\end{sphinxlegend}
\end{figure}

\begin{figure}[htbp]
\centering
\capstart

\noindent\sphinxincludegraphics{{e_stop}.png}
\caption{\sphinxstylestrong{Figure 12:} Laser Emergency Stop Button.}\label{\detokenize{user_documentation/Vault-1_laser:id10}}
\begin{sphinxlegend}
\sphinxAtStartPar
All the laser enclosures are equipped with laser emergency stop buttons.
The e\sphinxhyphen{}stops on an enclosure can only crash that specific laser.
\end{sphinxlegend}
\end{figure}

\sphinxAtStartPar
Additionally, there are ionizing radiation emergency stop buttons in Vault\sphinxhyphen{}1 and Vault\sphinxhyphen{}1 Control.
Those only serve the purpose of crashing the transmitters and are not located on the laser enclosures.
When the e\sphinxhyphen{}stop is pressed, the LED in the center will turn on.
To reset the e\sphinxhyphen{}stop, twist the button clockwise.


\subsection{Arming Vault\sphinxhyphen{}1 Laser Systems}
\label{\detokenize{user_documentation/Vault-1_laser:arming-vault-1-laser-systems}}
\sphinxAtStartPar
In Vault\sphinxhyphen{}1, the laser systems that can be armed are Vault\sphinxhyphen{}1 (as a laser lab), the Pharos enclosure, and the Pharos.
The Dira is in Laser\sphinxhyphen{}1 and must be armed from Laser\sphinxhyphen{}1.
The arming of the Dira is discussed in User Documentation: Laser\sphinxhyphen{}1 Interlock System User Manual.


\subsubsection{Arming the Pharos Enclosure and Laser}
\label{\detokenize{user_documentation/Vault-1_laser:arming-the-pharos-enclosure-and-laser}}
\sphinxAtStartPar
Because the Pharos is contained in an enclosure, Vault\sphinxhyphen{}1 does not need to be armed to arm the Pharos enclosure and laser.
To arm the Pharos enclosure and laser, press arm on the room interlock module on the south wall of the Pharos enclosure.
The room interlock module arms the enclosure.
For the Pharos system only, the local interlock module arms the Pharos laser underneath the room interlock module auto\sphinxhyphen{}arms with the room interlock module.
Once the Pharos arming modules are armed, the Pharos main shutter controller must be reset to clear a fault that arming causes by pressing the RESET REMOTE INTERLOCK on the controller.

\sphinxAtStartPar
Once the Pharos system is armed the following will change in the interlock system:
\begin{itemize}
\item {} 
\sphinxAtStartPar
The Pharos is now producing a laser beam.

\item {} 
\sphinxAtStartPar
Beacon stack \DUrole{blue}{blue} LEDs will turn on, indicating the Pharos.

\item {} 
\sphinxAtStartPar
The VIEWMARQ display will show \DUrole{red}{PHAROS ARMED}.

\item {} 
\sphinxAtStartPar
The laser warning modules on the Pharos enclosure will show \DUrole{red}{DANGER LASER ON}.

\item {} 
\sphinxAtStartPar
Pharos enclosure e\sphinxhyphen{}stop buttons will turn on, shown by the LED in the center. The e\sphinxhyphen{}stops are now functional and will crash the Pharos is pressed.

\end{itemize}


\begin{savenotes}\sphinxattablestart
\centering
\begin{tabulary}{\linewidth}[t]{|T|T|}
\hline

\noindent{\hspace*{\fill}\sphinxincludegraphics[scale=0.2]{{Pharos_enclosure_unarmed1}.jpg}\hspace*{\fill}}
&
\noindent{\hspace*{\fill}\sphinxincludegraphics[scale=0.2]{{Pharos_enclosure_armed1}.jpg}\hspace*{\fill}}
\\
\hline
\sphinxAtStartPar
Pharos enclosure in an unarmed state. \DUrole{white-cell}{===================================================}
&
\sphinxAtStartPar
Pharos enclosure in an armed state. \DUrole{white-cell}{=====================================================}
\\
\hline
\end{tabulary}
\par
\sphinxattableend\end{savenotes}

\begin{sphinxuseclass}{tight-table-caption-container}
\sphinxAtStartPar
\sphinxstylestrong{Figure 13:} These are the laser safety system modules for arming the Pharos in armed and unarmed states.

\end{sphinxuseclass}
\begin{figure}[htbp]
\centering
\capstart

\noindent\sphinxincludegraphics[scale=0.2]{{Pharos_main_shutter}.jpg}
\caption{\sphinxstylestrong{Figure 14:} This is the main shutter controller for the Pharos.
At the bottom is the button that must be pressed to reset the controller after the Pharos is armed.}\label{\detokenize{user_documentation/Vault-1_laser:id11}}\end{figure}


\subsubsection{Arming Vault\sphinxhyphen{}1 as a Laser Lab}
\label{\detokenize{user_documentation/Vault-1_laser:arming-vault-1-as-a-laser-lab}}
\sphinxAtStartPar
If work with a live laser must be performed in the enclosure, then Vault\sphinxhyphen{}1 must be armed as a laser lab.
To arm Vault\sphinxhyphen{}1 as a laser lab, press arm on the room interlock module in the Vault\sphinxhyphen{}1 entry.
Unlike for the ionizing radiation interlock system, the laser system does not have search buttons, however it is the responsibility of the user to ensure that Vault\sphinxhyphen{}1 is cleared or proper PPE is distributed before arming Vault\sphinxhyphen{}1.
Once Vault\sphinxhyphen{}1 is armed, the laser curtain door must be closed.
The curtain door is not interlocked and it strictly the responsibility of the user.
When the laser curtain is open, there will be a chiming prompting the user to close the curtain and informing people outside that the curtain is open.


\begin{savenotes}\sphinxattablestart
\centering
\begin{tabulary}{\linewidth}[t]{|T|T|}
\hline

\noindent{\hspace*{\fill}\sphinxincludegraphics[scale=0.2]{{Vault-1_unarmed1}.jpg}\hspace*{\fill}}
&
\noindent{\hspace*{\fill}\sphinxincludegraphics[scale=0.2]{{Vault-1_armed1}.jpg}\hspace*{\fill}}
\\
\hline
\sphinxAtStartPar
Vault\sphinxhyphen{}1 laser safety system modules in an unarmed state. \DUrole{white-cell}{====================================}
&
\sphinxAtStartPar
Vault\sphinxhyphen{}1 laser safety system modules in an armed state. \DUrole{white-cell}{======================================}
\\
\hline
\end{tabulary}
\par
\sphinxattableend\end{savenotes}

\begin{sphinxuseclass}{tight-table-caption-container}
\sphinxAtStartPar
\sphinxstylestrong{Figure 15:} These are the laser safety system modules for arming Vault\sphinxhyphen{}1 in armed and unarmed states.

\end{sphinxuseclass}
\sphinxAtStartPar
Once Vault\sphinxhyphen{}1 is armed as a laser lab the following will change in the interlock system:
\begin{itemize}
\item {} 
\sphinxAtStartPar
Beacon stack \DUrole{green}{green} LEDs will turn off, indicating that Vault\sphinxhyphen{}1 is not longer laser safe.

\item {} 
\sphinxAtStartPar
The VIEWMARQ display will show \DUrole{red}{DANGER LASER HAZARD}.

\item {} 
\sphinxAtStartPar
The laser warning modules in Vault\sphinxhyphen{}1 Control and Vault\sphinxhyphen{}1 entry will show \DUrole{red}{DANGER LASER ON}.

\item {} 
\sphinxAtStartPar
The push to exit button will turn on, shown by the LED in the center.

\item {} 
\sphinxAtStartPar
The Vault\sphinxhyphen{}1 door will be magnetically locked.

\end{itemize}

\sphinxAtStartPar
Once Vault\sphinxhyphen{}1 is armed the door is magnetically locked.
To get into Vault\sphinxhyphen{}1, you must type the Vault\sphinxhyphen{}1 laser pin into the keypad in Vault\sphinxhyphen{}1 Control.
Once the pin is entered, the door will temporarily unlock.
To exit Vault\sphinxhyphen{}1, you must push the push to exit button, which will again temporarily unlock the door.
It is important to note that Vault\sphinxhyphen{}1 will disarm itself and shutter all exposed laser hazards if the Vault\sphinxhyphen{}1 door is open for longer than the timer.


\begin{savenotes}\sphinxattablestart
\centering
\begin{tabulary}{\linewidth}[t]{|T|T|}
\hline

\noindent{\hspace*{\fill}\sphinxincludegraphics[scale=0.2]{{Vault-1_entry_unarmed}.jpg}\hspace*{\fill}}
&
\noindent{\hspace*{\fill}\sphinxincludegraphics[scale=0.2]{{Vault-1_entry_armed1}.jpg}\hspace*{\fill}}
\\
\hline
\sphinxAtStartPar
Vault\sphinxhyphen{}1 Control laser safety system modules in an unarmed state. \DUrole{white-cell}{===============================}
&
\sphinxAtStartPar
Vault\sphinxhyphen{}1 Control laser safety system modules in an armed state. \DUrole{white-cell}{=================================}
\\
\hline
\end{tabulary}
\par
\sphinxattableend\end{savenotes}

\begin{sphinxuseclass}{tight-table-caption-container}
\sphinxAtStartPar
\sphinxstylestrong{Figure 16:} These are the laser safety system modules for arming Vault\sphinxhyphen{}1 in armed and unarmed states.

\end{sphinxuseclass}

\subsection{User Laser Enclosure Interlock Protocases for Overriding Interlocks and Manual Shutter Control}
\label{\detokenize{user_documentation/Vault-1_laser:user-laser-enclosure-interlock-protocases-for-overriding-interlocks-and-manual-shutter-control}}
\sphinxAtStartPar
The shutters in the laser enclosures can be armed for manual control by the protocase LOCAL INTERLOCK CONTRACT CONTROL local interlock modules.
However, when the laser enclosures are interlocked, regardless of the arming status of the enclosure and Vault\sphinxhyphen{}1, if someone attempts to open the rolling enclosure doors the shutters will disarm and close.

\sphinxAtStartPar
What you will see happen on the enclosure protocase if the rolling door is opened when interlocked is:
\begin{itemize}
\item {} 
\sphinxAtStartPar
Laser warning modules will show \DUrole{green}{LASER SAFE}.

\item {} 
\sphinxAtStartPar
Door monitor module will be blank, meaning open.

\item {} 
\sphinxAtStartPar
LOCAL INTERLOCK CONTACT CONTROL local interlock modules will be disarmed if armed, automatically closing the shutters.

\end{itemize}


\subsection{Interlock to Override}
\label{\detokenize{user_documentation/Vault-1_laser:interlock-to-override}}
\sphinxAtStartPar
The only way to work in the laser enclosures with light on the table is to change the enclosures interlocks to administrative override.
In administrative override the interlocks system sees the rolling doors and closed even if they are opened, bypassing the interlocks.

\sphinxAtStartPar
For a laser enclosure to be put into administrative override, both Vault\sphinxhyphen{}1 and the enclosure must be armed.
Specifically for the working with the Dira, both the Pharos and Dira enclosures must be set to administrative override.
This is because the Pharos exports a beam into the Dira enclosure, so both enclosure interlocks need to be bypassed.
The controls for the administrative overrides are on the enclosures LASER ENCLOSURE INTERLOCK protocase.
Turn the key on the protocase under INTERLOCK OVERRIDE from INTERLOCK TO OVERRIDE.

\sphinxAtStartPar
Once the enclosure is put into override the following will change in the interlock system:
\begin{itemize}
\item {} 
\sphinxAtStartPar
The administrative override \DUrole{orange}{orange} LED on the enclosure specific protocase will turn on.

\item {} 
\sphinxAtStartPar
The Vault\sphinxhyphen{}1 Control and Vault\sphinxhyphen{}1 entry eat wall administrative override \DUrole{orange}{orange} LEDs will turn on.

\item {} 
\sphinxAtStartPar
The enclosure specific protocase STATUS LED will turn \DUrole{red}{red}.

\item {} 
\sphinxAtStartPar
If you open the enclosure, the laser warning module will still show \DUrole{red}{LASER DANGER ON}, the door monitor module will show \DUrole{green}{CLOSED}, and the local interlock modules for arming shutter manual control will not disarm.

\end{itemize}

\sphinxAtStartPar
At this point, the LOCAL INTERLOCK CONTACT CONTROL interlock modules can be armed, and the shutters can be controlled manually without the interlocks disarming manual usage.


\begin{savenotes}\sphinxattablestart
\centering
\begin{tabulary}{\linewidth}[t]{|T|T|}
\hline

\noindent{\hspace*{\fill}\sphinxincludegraphics[scale=0.2]{{Pharos_protocase_override}.jpg}\hspace*{\fill}}
&
\noindent{\hspace*{\fill}\sphinxincludegraphics[scale=0.2]{{Dira_protocase_override}.jpg}\hspace*{\fill}}
\\
\hline
\sphinxAtStartPar
Pharos enclosure protocase in an override state. \DUrole{white-cell}{============================================}
&
\sphinxAtStartPar
Dira enclosure protocase in an override state. \DUrole{white-cell}{==============================================}
\\
\hline
\end{tabulary}
\par
\sphinxattableend\end{savenotes}

\begin{sphinxuseclass}{tight-table-caption-container}
\sphinxAtStartPar
\sphinxstylestrong{Figure 17:} These are the laser safety system modules for arming Vault\sphinxhyphen{}1 in armed and unarmed states.

\end{sphinxuseclass}

\subsection{Disarming the Laser Interlock System}
\label{\detokenize{user_documentation/Vault-1_laser:disarming-the-laser-interlock-system}}
\sphinxAtStartPar
To take either enclosures out of administrative override, simply change the INTERLOCK OVERRIDE key on the LASER ENCLOSURE INTERLOCK protocase back from OVERRIDE to INTERLOCK.
Also, all the arming laser modules have disarming buttons where you can either disarm specific modules you no longer need, or you can disarm the room modules to auto\sphinxhyphen{}disarm their local modules.

\sphinxstepscope


\section{Hutch\sphinxhyphen{}1 Ionizing Radiation Interlock System User Manual}
\label{\detokenize{user_documentation/Hutch-1_ionizing_radiation:hutch-1-ionizing-radiation-interlock-system-user-manual}}\label{\detokenize{user_documentation/Hutch-1_ionizing_radiation::doc}}
\sphinxAtStartPar
This document provides a user overview for the Hutch\sphinxhyphen{}1 Ionizing Radiation Interlock System.
It covers the hazard indicators, control protocase, and beacons in Hutch\sphinxhyphen{}1 Control.
The control protocase allows users to view and change the secure state of Vault\sphinxhyphen{}1, arm the accelerator and transmitters, and monitor area monitors.
The beacons indicate the enabled state of RF and ionizing radiation emergency stop buttons.


\subsection{Hutch\sphinxhyphen{}1 Ionizing Radiation Hazard Indicators}
\label{\detokenize{user_documentation/Hutch-1_ionizing_radiation:hutch-1-ionizing-radiation-hazard-indicators}}
\sphinxAtStartPar
This section will cover the ionizing radiation hazard indicators in Hutch\sphinxhyphen{}1 Control.
These hazards will correspond to hazards present in Hutch\sphinxhyphen{}1.

\sphinxAtStartPar
Unlike the Vault\sphinxhyphen{}1 ionizing radiation interlocks system, the VIEWMARQ display in Hutch\sphinxhyphen{}1 Control only shows the status of the Hutch\sphinxhyphen{}1 laser interlocks system.


\subsubsection{Hutch\sphinxhyphen{}1 Control Protocase}
\label{\detokenize{user_documentation/Hutch-1_ionizing_radiation:hutch-1-control-protocase}}
\sphinxAtStartPar
The Hutch\sphinxhyphen{}1 Control IONIZING RADIATION INTERLOCK protocase is an interface to view if Hutch\sphinxhyphen{}1 is secure, control the double tungsten shutters, and view the status of the area monitors.
This panel is located on the north wall in Hutch\sphinxhyphen{}1 Control next to the Hutch\sphinxhyphen{}1 door.

\sphinxAtStartPar
The SECURE PERIMETER section of the protocase shows the status of search buttons being engaged and if the shield door is open.
If all the lamps are \DUrole{green}{green}, then Hutch\sphinxhyphen{}1 is in a secure state. When Hutch\sphinxhyphen{}1 is secure the hutch cannot be entered until the accelerator is put into an unarmed state.
If the shield door is opened when one of the double tungsten shutters is open, the interlocks system will trip and close the shutter.
Once Hutch\sphinxhyphen{}1 is secured the allowed dose rate on the Hutch\sphinxhyphen{}1 rate meters changes from the non\sphinxhyphen{}radiation facility user set point of 50 µS/hr to the radiation worker set point of 500 µS/hr.

\sphinxAtStartPar
The AREA MONITOR section of the protocase shows if any rate meter in the CXLS suite is alarming.
If the RADIATION lamp is \DUrole{red}{red}, then there is either an alert, alarm, or fail status from one of the rate meters.
If any of the radiation meters alarm then the interlocks will trip, and the accelerator will be put into a safe state.

\sphinxAtStartPar
The BEAM STOP section of the protocase shows if the beam stop is open.
If the lamp is \DUrole{green}{green}, then the beam stop is open, and beam is being imported into Hutch\sphinxhyphen{}1.
Above that are the BEAM STATUS AND BEAM SELECT lamps, which show if the divergent or collimated beam is being imported into Hutch\sphinxhyphen{}1.
These lamps should show the same thing, BEAM SELECT shows what the user has selected and BEAM STATUS shows the actual positions of the shutters, \DUrole{green}{green} being open.

\begin{figure}[htbp]
\centering
\capstart

\noindent\sphinxincludegraphics[scale=0.2]{{Hutch-1_protocase}.jpg}
\caption{\sphinxstylestrong{Figure 1:} This is the Hutch\sphinxhyphen{}1 Control IONIZING RADIATION INTERLOCK protocase. In this state, Hutch\sphinxhyphen{}1 is not secured}\label{\detokenize{user_documentation/Hutch-1_ionizing_radiation:id1}}\end{figure}


\subsubsection{O2 Main and Remote Units}
\label{\detokenize{user_documentation/Hutch-1_ionizing_radiation:o2-main-and-remote-units}}
\sphinxAtStartPar
There are two O2 sensors in the Hutch\sphinxhyphen{}1 ionizing radiation interlock system.
One is located on the Hutch\sphinxhyphen{}1 west wall and the other is located in the Astrella enclosure.
These units will have an audible alarm and flash one of the AL\# LEDs depending on the alarm set point it passed.
Any \(O_{2}\) reading below 19\% will cause the sensors to alarm.
Each O2 sensor has a remote unit that only displays information from the main sensor unit.
The Hutch\sphinxhyphen{}1 west wall and Astrella enclosure O2 remote units are located in Hutch\sphinxhyphen{}1 Control next to the Hutch\sphinxhyphen{}1 Control IONIZING RADIATION INTERLOCK protocase.


\begin{savenotes}\sphinxattablestart
\centering
\begin{tabulary}{\linewidth}[t]{|T|T|}
\hline

\noindent{\hspace*{\fill}\sphinxincludegraphics[scale=0.2]{{Vault-1_O2_main}.jpg}\hspace*{\fill}}
&
\noindent{\hspace*{\fill}\sphinxincludegraphics[scale=0.2]{{Vault-1_O2_remote}.jpg}\hspace*{\fill}}
\\
\hline
\sphinxAtStartPar
O2 main unit. \DUrole{white-cell}{=====================================================================}
&
\sphinxAtStartPar
O2 remote unit. \DUrole{white-cell}{===================================================================}
\\
\hline
\end{tabulary}
\par
\sphinxattableend\end{savenotes}

\begin{sphinxuseclass}{tight-table-caption-container}
\sphinxAtStartPar
\sphinxstylestrong{Figure 2:} This is the O2 sensor pair.

\end{sphinxuseclass}

\subsection{Beacons}
\label{\detokenize{user_documentation/Hutch-1_ionizing_radiation:beacons}}
\sphinxAtStartPar
In Hutch\sphinxhyphen{}1 Control, there are two beacons that show ionizing radiation hazards.
There is one for the \(O_{2}\) levels and one for the ionizing radiation emergency stop button.


\begin{savenotes}\sphinxattablestart
\centering
\begin{tabulary}{\linewidth}[t]{|T|T|}
\hline

\noindent{\hspace*{\fill}\sphinxincludegraphics[scale=0.2]{{Hutch-1_protocase_beacon}.jpg}\hspace*{\fill}}
&
\noindent{\hspace*{\fill}\sphinxincludegraphics[scale=0.2]{{Hutch-1_O2_beacon}.jpg}\hspace*{\fill}}
\\
\hline
\sphinxAtStartPar
Protocase beacon \DUrole{white-cell}{=================================================================}
&
\sphinxAtStartPar
O2 beacon \DUrole{white-cell}{========================================================================}
\\
\hline
\end{tabulary}
\par
\sphinxattableend\end{savenotes}

\begin{sphinxuseclass}{tight-table-caption-container}
\sphinxAtStartPar
\sphinxstylestrong{Figure 3:}  These are the beacons in Hutch\sphinxhyphen{}1 Control.

\end{sphinxuseclass}

\begin{savenotes}\sphinxattablestart
\centering
\begin{tabulary}{\linewidth}[t]{|T|T|}
\hline
\sphinxstyletheadfamily 
\sphinxAtStartPar
Status
&\sphinxstyletheadfamily 
\sphinxAtStartPar
Beacon Color
\\
\hline
\sphinxAtStartPar
The \DUrole{red}{red} beacon indicated that an ionizing radiation emergency stop button had been

\sphinxAtStartPar
pressed. This beacon is also on the Vault\sphinxhyphen{}1 Control protocase.
&
\sphinxAtStartPar
\DUrole{red-cell}{Beacon Color}
\\
\hline
\sphinxAtStartPar
The \DUrole{orange}{orange} beacon indicates that one of the O2 meters is reading below 19\% oxygen

\sphinxAtStartPar
levels.
&
\sphinxAtStartPar
\DUrole{orange-cell}{Beacon Color}
\\
\hline
\end{tabulary}
\par
\sphinxattableend\end{savenotes}


\subsubsection{Ionizing Radiation Monitor}
\label{\detokenize{user_documentation/Hutch-1_ionizing_radiation:ionizing-radiation-monitor}}
\begin{sphinxadmonition}{note}{Note:}
\sphinxAtStartPar
The ionizing radiation monitor may go through changes in the near future.
This section will be updated when those changes are made.
\end{sphinxadmonition}


\subsection{Ionizing Radiation Emergency Stop Buttons}
\label{\detokenize{user_documentation/Hutch-1_ionizing_radiation:ionizing-radiation-emergency-stop-buttons}}
\sphinxAtStartPar
Throughout the CXLS suite there are ionizing radiation emergency stop buttons.
These e\sphinxhyphen{}stop buttons will cut power to the transmitters, putting the accelerator in a safe state.
Once the transmitters are crashed, there will not longer be a source of ionizing radiation.
When an ionizing radiation e\sphinxhyphen{}stop button is pressed, the LED on the unit will turn on, all red beacons will turn on, and the VIEWMARQ displays will show \DUrole{red}{IONIZING RADIATION E\sphinxhyphen{}STOP ACTIVATED}.
To disengage the e\sphinxhyphen{}stop, rotate the button clockwise.

\sphinxAtStartPar
It is important to note that only the ionizing radiation emergency stop buttons will put the accelerator into a safe state.
There is also laser emergency stop buttons that will only cut power to their specific laser if armed and do not affect the transmitters.


\begin{savenotes}\sphinxattablestart
\centering
\begin{tabulary}{\linewidth}[t]{|T|T|}
\hline

\noindent{\hspace*{\fill}\sphinxincludegraphics[scale=0.2]{{Vault-1_estop_off}.jpg}\hspace*{\fill}}
&
\noindent{\hspace*{\fill}\sphinxincludegraphics[scale=0.2]{{Vault-1_estop_on}.jpg}\hspace*{\fill}}
\\
\hline
\sphinxAtStartPar
Ionizing radiation emergency stop button off. \DUrole{white-cell}{==============================================}
&
\sphinxAtStartPar
Ionizing radiation emergency stop button on. \DUrole{white-cell}{===============================================}
\\
\hline
\end{tabulary}
\par
\sphinxattableend\end{savenotes}

\begin{sphinxuseclass}{tight-table-caption-container}
\sphinxAtStartPar
\sphinxstylestrong{Figure 4:} This is the ionizing radiation emergency stop button in both states.

\end{sphinxuseclass}

\subsection{Search Procedure for Securing Hutch\sphinxhyphen{}1}
\label{\detokenize{user_documentation/Hutch-1_ionizing_radiation:search-procedure-for-securing-hutch-1}}
\sphinxAtStartPar
To be able to operate the double tungsten shutters, Hutch\sphinxhyphen{}1 must be searched and secured.
Starting at the North\sphinxhyphen{}East end of Hutch\sphinxhyphen{}1 (upstream steam of the chambers), while verifying the hutch is empty, press the search button labeled 1.
As you continue to search and clear press 2 then 3 as you’re working your way towards the hutch entrance.
Once the 3rd search button is pressed, a chime will be audible and a timer will start and all the SECURE PERIMETER SEARCH lamps on the Hutch\sphinxhyphen{}1 Control IONIZING RADIATION INTERLOCK protocase will be \DUrole{green}{green}.
If the search buttons are pressed out of order, or the search takes too long, the search will need to be restarted.

\begin{figure}[htbp]
\centering
\capstart

\noindent\sphinxincludegraphics[scale=0.5]{{Hutch1_Search_Buttons}.png}
\caption{\sphinxstylestrong{Figure 5:} This is a diagram of the Hutch\sphinxhyphen{}1 search buttons. The numbers indicate the order in which they need to be pressed.}\label{\detokenize{user_documentation/Hutch-1_ionizing_radiation:id2}}\end{figure}


\begin{savenotes}\sphinxattablestart
\centering
\begin{tabulary}{\linewidth}[t]{|T|T|}
\hline

\noindent{\hspace*{\fill}\sphinxincludegraphics[scale=0.2]{{Vault-1_search_off}.jpg}\hspace*{\fill}}
&
\noindent{\hspace*{\fill}\sphinxincludegraphics[scale=0.2]{{Vault-1_search_on}.jpg}\hspace*{\fill}}
\\
\hline
\sphinxAtStartPar
Vault\sphinxhyphen{}1 search button off. \DUrole{white-cell}{============================================================}
&
\sphinxAtStartPar
Vault\sphinxhyphen{}1 search button on. \DUrole{white-cell}{=============================================================}
\\
\hline
\end{tabulary}
\par
\sphinxattableend\end{savenotes}

\begin{sphinxuseclass}{tight-table-caption-container}
\sphinxAtStartPar
\sphinxstylestrong{Figure 6:} This is the Vault\sphinxhyphen{}1 search button in both states.

\end{sphinxuseclass}
\begin{figure}[htbp]
\centering
\capstart

\noindent\sphinxincludegraphics[scale=0.2]{{Hutch-1_searched}.jpg}
\caption{\sphinxstylestrong{Figure 7:} This is the Hutch\sphinxhyphen{}1 Control IONIZING RADIATION INTERLOCK protocase when Hutch\sphinxhyphen{}1 is searched.}\label{\detokenize{user_documentation/Hutch-1_ionizing_radiation:id3}}\end{figure}

\sphinxAtStartPar
Once Hutch\sphinxhyphen{}1 is searched and all the search buttons have been pressed in the correct sequence, all the SECURE PERIMETER SEARCH lamps on the Hutch\sphinxhyphen{}1 Control IONIZING RADIATION INTERLOCK protocase will be \DUrole{green}{green}.
Unlike the Vault\sphinxhyphen{}1 door, this door is closed by pulling the door shut.
Once the door is fully closed and actuating the door switches the SECURE PERIMETER SHIELD DOOR lamp on the Hutch\sphinxhyphen{}1 Control IONIZING RADIATION INTERLOCK protocase will be \DUrole{green}{green}.

\begin{figure}[htbp]
\centering
\capstart

\noindent\sphinxincludegraphics[scale=0.2]{{Hutch-1_door}.jpg}
\caption{\sphinxstylestrong{Figure 8:} This is the Hutch\sphinxhyphen{}1 Control IONIZING RADIATION INTERLOCK protocase when Hutch\sphinxhyphen{}1 is secured.}\label{\detokenize{user_documentation/Hutch-1_ionizing_radiation:id4}}\end{figure}


\subsection{Controlling the Beam Status in Hutch\sphinxhyphen{}1}
\label{\detokenize{user_documentation/Hutch-1_ionizing_radiation:controlling-the-beam-status-in-hutch-1}}
\sphinxAtStartPar
Once Hutch\sphinxhyphen{}1 is searched and secured, the beam stop can be opened by turning the BEAM STOP OPEN key on the Hutch\sphinxhyphen{}1 Control IONIZING RADIATION INTERLOCK protocase, the lamp should turn \DUrole{green}{green} when the stop is open.
If the shield door is opened with the beam stop open, the beam stop will close, and Hutch\sphinxhyphen{}1 will no longer be secure.

\sphinxAtStartPar
At any point, the shutters can be closed again by hitting BEAM STOP RESET on the Hutch\sphinxhyphen{}1 IONIZING RADIATION INTERLOCK protocase.

\sphinxAtStartPar
When the beam stop is open, the shutters will only allow either the collimated or divergent beam into Hutch\sphinxhyphen{}1.
To select which beam is allowed into Hutch\sphinxhyphen{}1 use the BEAM SELECT key on the Hutch\sphinxhyphen{}1 Control IONIZING RADIATION INTERLOCK protocase.
The BEAM SELECT lamp shows what has been selected, and the BEAM STATUS lamp shows the status of the shutters.


\begin{savenotes}\sphinxattablestart
\centering
\begin{tabulary}{\linewidth}[t]{|T|T|}
\hline

\noindent{\hspace*{\fill}\sphinxincludegraphics[scale=0.2]{{Hutch-1_Divergent_open}.jpg}\hspace*{\fill}}
&
\noindent{\hspace*{\fill}\sphinxincludegraphics[scale=0.2]{{Hutch-1_Collimated_open}.jpg}\hspace*{\fill}}
\\
\hline
\sphinxAtStartPar
Divergent beam open. \DUrole{white-cell}{===============================================================}
&
\sphinxAtStartPar
Collimated beam open. \DUrole{white-cell}{==============================================================}
\\
\hline
\end{tabulary}
\par
\sphinxattableend\end{savenotes}

\begin{sphinxuseclass}{tight-table-caption-container}
\sphinxAtStartPar
\sphinxstylestrong{Figure 9:} This is the Hutch\sphinxhyphen{}1 Control IONIZING RADIATION INTERLOCK protocase when either shutter is open. When the beam stop is open a shutter will automatically open to whatever beam select is set to before hand.

\end{sphinxuseclass}

\subsection{Putting Hutch\sphinxhyphen{}1 into a Non\sphinxhyphen{}Secure State}
\label{\detokenize{user_documentation/Hutch-1_ionizing_radiation:putting-hutch-1-into-a-non-secure-state}}
\sphinxAtStartPar
Once work in Hutch\sphinxhyphen{}1 is completed and is no longer required to be in a secure state, press the BEAM STOP RESET button on the Hutch\sphinxhyphen{}1 IONIZING RADIATION INTERLOCK protocase and open the shield door.


\subsection{Hutch\sphinxhyphen{}1 Radiation Survey Procedure}
\label{\detokenize{user_documentation/Hutch-1_ionizing_radiation:hutch-1-radiation-survey-procedure}}
\begin{sphinxadmonition}{note}{Note:}
\sphinxAtStartPar
As of now, there is no ionizing radiation survey procedure for Hutch\sphinxhyphen{}1.
\end{sphinxadmonition}

\sphinxstepscope


\section{Hutch\sphinxhyphen{}1 Laser Interlock System User Manual}
\label{\detokenize{user_documentation/Hutch-1_laser:hutch-1-laser-interlock-system-user-manual}}\label{\detokenize{user_documentation/Hutch-1_laser::doc}}
\sphinxAtStartPar
This document provides a user overview for the Hutch\sphinxhyphen{}1 Laser Interlock System.
It covers the laser modules, hazard indicators, control protocases, VIEWMARQ display, and the beacons in Hutch\sphinxhyphen{}1 and Hutch\sphinxhyphen{}1 Control.
The laser modules are used for indicating the status of the laser interlock system and arming the laser systems.
The control protocase allows the users to view the override state of the laser enclosure and control the manual shutter control.
The VIEWMARQ display provides a quick overview of the laser hazard status in Hutch\sphinxhyphen{}1.
The beacons are used to indicate the state of laser hazards in Hutch\sphinxhyphen{}1 and the laser enclosure.


\subsection{Hutch\sphinxhyphen{}1 Laser Hazard Warning Indicators}
\label{\detokenize{user_documentation/Hutch-1_laser:hutch-1-laser-hazard-warning-indicators}}
\sphinxAtStartPar
This section will cover the laser hazard indicators in Hutch\sphinxhyphen{}1 Control and Hutch\sphinxhyphen{}1.


\subsubsection{Astrella Enclosure Protocase}
\label{\detokenize{user_documentation/Hutch-1_laser:astrella-enclosure-protocase}}
\sphinxAtStartPar
The Astrella LASER ENCLOSURE INTERLOCK protocase shows hazards in for the administrative override status of the Astrella enclosure.

\sphinxAtStartPar
The PERIMETER section of the Astrella LASER ENCLOSURE INTERLOCK protocase has a laser warning module and door monitor module.
The laser warning module displays if the Astrella is forced into a safe state. This means it will always show \DUrole{red}{DANGER LASER ON} unless the interlocks are tripped.
The door monitor shows if the enclosure doors are opened or closed. If the enclosure is put into an override state, then the monitor will always show closed because the interlocks are bypassed.

\sphinxAtStartPar
The LOCAL INTERLOCK CONTACT CONTROL section of the Astrella LASER ENCLOSURE INTERLOCK protocase are two local interlock modules.
These modules are for arming the UV and IV shutter controllers for manual operation.
These can only be armed when the enclosure is armed. However, if the enclosure is not set to override, then these will disarm when opening the enclosure.

\sphinxAtStartPar
The INTERLOCK OVERRIDE section of the Astrella LASER ENCLOSURE INTERLOCK protocase shows the status of the enclosure interlocks.
If the key is set to OVERRIDE and the STATUS LED is \DUrole{red}{red}, that means that the enclosure interlocks are bypass, and there could be a laser hazard if the enclosure is opened.

\begin{figure}[htbp]
\centering
\capstart

\noindent\sphinxincludegraphics[scale=0.2]{{Astrella_protocase}.jpg}
\caption{\sphinxstylestrong{Figure 1:} Hutch\sphinxhyphen{}1 Astrella enclosure protocase.
It is located on the south wall of the enclosure.}\label{\detokenize{user_documentation/Hutch-1_laser:id1}}\end{figure}


\subsubsection{Beacon Stacks}
\label{\detokenize{user_documentation/Hutch-1_laser:beacon-stacks}}
\sphinxAtStartPar
There are beacon stacks in Hutch\sphinxhyphen{}1 Control and on the Astrella LASER ENCLOSURE INTERLOCK protocase.
The beacons stacks can notify you of the arming status for Hutch\sphinxhyphen{}1 and the Astella, as well as the override status of the enclosure.


\begin{savenotes}\sphinxattablestart
\centering
\begin{tabulary}{\linewidth}[t]{|T|T|}
\hline
\sphinxstyletheadfamily 
\sphinxAtStartPar
Status
&\sphinxstyletheadfamily 
\sphinxAtStartPar
Beacon Color
\\
\hline
\sphinxAtStartPar
Hutch\sphinxhyphen{}1 is not armed as a laser lab.
&
\sphinxAtStartPar
\DUrole{green-cell}{Beacon Color}
\\
\hline
\sphinxAtStartPar
The Astrella is set to administrative override.

\sphinxAtStartPar
This state is only possible if Hutch\sphinxhyphen{}1 is armed.
&
\sphinxAtStartPar
\DUrole{orange-cell}{Beacon Color}
\\
\hline
\sphinxAtStartPar
The Astrella is armed.

\sphinxAtStartPar
This state is possible with or without Hutch\sphinxhyphen{}1 being armed.
&
\sphinxAtStartPar
\DUrole{white-cell}{Beacon Color}
\\
\hline
\end{tabulary}
\par
\sphinxattableend\end{savenotes}


\subsubsection{VIEWMARQ Display}
\label{\detokenize{user_documentation/Hutch-1_laser:viewmarq-display}}
\sphinxAtStartPar
There is a VIEWMARQ display in Hutch\sphinxhyphen{}1 Control that states the status of potential laser hazards in Hutch\sphinxhyphen{}1.
This display can notify you of the arming status for Hutch\sphinxhyphen{}1 and the Astella, as well as the interlock status of the enclosure.


\begin{savenotes}\sphinxattablestart
\centering
\begin{tabulary}{\linewidth}[t]{|T|T|T|T|T|}
\hline

\noindent{\hspace*{\fill}\sphinxincludegraphics[scale=0.2]{{Hutch-1_VIEWMARQ_laser_safe}.jpg}\hspace*{\fill}}
&
\noindent{\hspace*{\fill}\sphinxincludegraphics[scale=0.2]{{Hutch-1_VIEWMARQ_laser_safe_armed}.jpg}\hspace*{\fill}}
&
\noindent{\hspace*{\fill}\sphinxincludegraphics[scale=0.2]{{Hutch-1_VIEWMARQ_laser_hazard}.jpg}\hspace*{\fill}}
&
\noindent{\hspace*{\fill}\sphinxincludegraphics[scale=0.2]{{Hutch-1_VIEWMARQ_laser_hazard_armed}.jpg}\hspace*{\fill}}
&
\noindent{\hspace*{\fill}\sphinxincludegraphics[scale=0.2]{{Hutch-1_VIEWMARQ_override}.jpg}\hspace*{\fill}}
\\
\hline
\sphinxAtStartPar
Hutch\sphinxhyphen{}1 Control beacon stack and VIEWMARQ display when Hutch\sphinxhyphen{}1 is laser safe. \DUrole{white-cell}{=================}
&
\sphinxAtStartPar
Hutch\sphinxhyphen{}1 Control beacon stack and VIEWMARQ display when the Astrella is armed. \DUrole{white-cell}{=================}
&
\sphinxAtStartPar
Hutch\sphinxhyphen{}1 Control beacon stack and VIEWMARQ display when Hutch\sphinxhyphen{}1 is armed as as laser lab. \DUrole{white-cell}{======}
&
\sphinxAtStartPar
Hutch\sphinxhyphen{}1 Control beacon stack and VIEWMARQ display when Hutch\sphinxhyphen{}1 and the Astrella are armed. \DUrole{white-cell}{====}
&
\sphinxAtStartPar
Hutch\sphinxhyphen{}1 Control beacon stack and VIEWMARQ display when the Astrella enclosure is in administrative override.
\\
\hline
\end{tabulary}
\par
\sphinxattableend\end{savenotes}


\begin{savenotes}\sphinxattablestart
\centering
\begin{tabulary}{\linewidth}[t]{|T|T|}
\hline
\sphinxstyletheadfamily 
\sphinxAtStartPar
VIEWMARQ Display Notes
&\sphinxstyletheadfamily 
\sphinxAtStartPar
VIEWMARQ Display Text
\\
\hline
\sphinxAtStartPar
This states if Hutch\sphinxhyphen{}1 is armed as a laser lab or not.
&
\sphinxAtStartPar
\DUrole{green}{LASER SAFE} / \DUrole{green}{DANGER LASER HAZARD}
\\
\hline
\sphinxAtStartPar
This states if the Astrella is armed.
&
\sphinxAtStartPar
\DUrole{red}{Astrella ARMED}
\\
\hline
\sphinxAtStartPar
This states if the Astrella is in administrative override.
&
\sphinxAtStartPar
\DUrole{red}{Astrella ADMIN OVERRIDE}
\\
\hline
\end{tabulary}
\par
\sphinxattableend\end{savenotes}

\sphinxAtStartPar
The top line always will either display LASER SAFE or DANGER LASER HAZARD.
All other possible states will only appear on the display when the hazard is presented.


\subsubsection{Laser Safety System Modules}
\label{\detokenize{user_documentation/Hutch-1_laser:laser-safety-system-modules}}
\sphinxAtStartPar
The laser interlock system is interfaced through the laser safety systems modules. Below is an outline of the modules and what they do.

\begin{figure}[htbp]
\centering
\capstart

\noindent\sphinxincludegraphics{{warning_module}.gif}
\caption{\sphinxstylestrong{Figure 3:} Area Warming Module}\label{\detokenize{user_documentation/Hutch-1_laser:id2}}\end{figure}


\begin{savenotes}\sphinxattablestart
\centering
\begin{tabular}[t]{|*{2}{\X{1}{2}|}}
\hline
\sphinxstyletheadfamily 
\sphinxAtStartPar
Module Location
&\sphinxstyletheadfamily 
\sphinxAtStartPar
Module Meaning
\\
\hline
\begin{DUlineblock}{0em}
\item[] \sphinxstylestrong{General Area Module}
\item[] Vault\sphinxhyphen{}1 Control
\item[] Vault\sphinxhyphen{}1 Entry
\end{DUlineblock}
&
\begin{DUlineblock}{0em}
\item[] These are warning modules tell you if Vault\sphinxhyphen{}1 is armed as a laser lab.
\item[] \DUrole{red}{DANGER LASER ON} = ARMED
\end{DUlineblock}
\\
\hline
\begin{DUlineblock}{0em}
\item[] \sphinxstylestrong{Enclosure Modules}
\item[] Pharos enclosure south wall
\item[] Pharos enclosure west wall
\end{DUlineblock}
&
\begin{DUlineblock}{0em}
\item[] These warning modules tell you if the enclosure is armed.
\item[] There is no indication on if the laser is armed.
\item[] \DUrole{red}{DANGER LASER ON} = ARMED
\end{DUlineblock}
\\
\hline
\begin{DUlineblock}{0em}
\item[] \sphinxstylestrong{Protocase Modules}
\item[] Pharos enclosure protocase
\item[] Dira enclosure protocase
\end{DUlineblock}
&
\begin{DUlineblock}{0em}
\item[] These warning modules tell you if the enclosure is forced to a safe state.
\item[] \DUrole{red}{DANGER LASER HAZARD} = SAFE STATE IS NOT FORCED
\end{DUlineblock}
\\
\hline
\end{tabular}
\par
\sphinxattableend\end{savenotes}

\begin{figure}[htbp]
\centering
\capstart

\noindent\sphinxincludegraphics{{control_module}.gif}
\caption{\sphinxstylestrong{Figure 4:} Control Module}\label{\detokenize{user_documentation/Hutch-1_laser:id3}}
\begin{sphinxlegend}
\sphinxAtStartPar
This module is a control module for the local laser interlock, however, for the users it serves as another warning module.
This warning module tells you if the local interlock is armed or not.
\end{sphinxlegend}
\end{figure}

\begin{figure}[htbp]
\centering
\capstart

\noindent\sphinxincludegraphics{{room_arm}.png}
\caption{\sphinxstylestrong{Figure 5:} Room Arm Module}\label{\detokenize{user_documentation/Hutch-1_laser:id4}}
\begin{sphinxlegend}
\sphinxAtStartPar
This module is used to arming system systems in the laser interlock system.
For example, there are two in Vault\sphinxhyphen{}1, one to arm the vault and one to arm the Pharos enclosure.
\end{sphinxlegend}
\end{figure}


\begin{savenotes}\sphinxattablestart
\centering
\begin{tabular}[t]{|*{2}{\X{1}{2}|}}
\hline
\sphinxstyletheadfamily 
\sphinxAtStartPar
Module Message
&\sphinxstyletheadfamily 
\sphinxAtStartPar
Message Meaning
\\
\hline
\sphinxAtStartPar
\DUrole{orange}{ROOM ARMED}
&
\begin{DUlineblock}{0em}
\item[] If this LED is on, then the corresponding system is armed
\item[] and interlocked.
\end{DUlineblock}
\\
\hline
\sphinxAtStartPar
\DUrole{green}{ROOM DISARMED (READY TO ARM)}
&
\begin{DUlineblock}{0em}
\item[] If this LED is on, the the system is the correct state to armed
\item[] the module.
\end{DUlineblock}
\\
\hline
\sphinxAtStartPar
\DUrole{orange}{ROOM CRASHED (CANNOT ARM)}
&
\begin{DUlineblock}{0em}
\item[] If this LED is on, then there was a fault that tripped the
\item[] system, or a fault that will not allow the system to be armed.
\end{DUlineblock}
\\
\hline
\end{tabular}
\par
\sphinxattableend\end{savenotes}

\begin{figure}[htbp]
\centering
\capstart

\noindent\sphinxincludegraphics{{local_arm}.png}
\caption{\sphinxstylestrong{Figure 6:} Local Arm Module}\label{\detokenize{user_documentation/Hutch-1_laser:id5}}
\begin{sphinxlegend}
\sphinxAtStartPar
This module is used to arm the local interlock modules that are sub\sphinxhyphen{}systems of the room arm modules.
For example, one the Pharos enclosure is armed, it enables the laser and the shutters to be armed by their local arming modules.
\end{sphinxlegend}
\end{figure}


\begin{savenotes}\sphinxattablestart
\centering
\begin{tabular}[t]{|*{2}{\X{1}{2}|}}
\hline
\sphinxstyletheadfamily 
\sphinxAtStartPar
Module Message
&\sphinxstyletheadfamily 
\sphinxAtStartPar
Message Meaning
\\
\hline
\sphinxAtStartPar
\DUrole{orange}{LOCAL CONTACTS ARMED}
&
\sphinxAtStartPar
If this LED is on, then the corresponding sub\sphinxhyphen{}system is armed.
\\
\hline
\sphinxAtStartPar
\DUrole{green}{LOCAL CONTACTS DISARMED}
&
\begin{DUlineblock}{0em}
\item[] If this LED is on, then the connected room module is armed,
\item[] but this module is not.
\end{DUlineblock}
\\
\hline
\begin{DUlineblock}{0em}
\item[] \DUrole{green}{LOCAL CONTACTS DISARMED}
\item[] \DUrole{green}{ROOM NOT ARMED}
\item[] \DUrole{green}{LOCAL CONTACT CANNOT ARM}
\end{DUlineblock}
&
\begin{DUlineblock}{0em}
\item[] If this LED is on, then there was a fault that tripped, or the room
\item[] module is not armed.
\end{DUlineblock}
\\
\hline
\end{tabular}
\par
\sphinxattableend\end{savenotes}

\begin{figure}[htbp]
\centering
\capstart

\noindent\sphinxincludegraphics{{push_to_exit}.png}
\caption{\sphinxstylestrong{Figure 7:} Push to Exit Module}\label{\detokenize{user_documentation/Hutch-1_laser:id6}}
\begin{sphinxlegend}
\sphinxAtStartPar
This module is used to exit when a room is armed as a laser lab.
When the rooms are armed, the doors are magnetically locked.
This button will temporarily unlock the door to allow you to exit the room.
\end{sphinxlegend}
\end{figure}

\begin{figure}[htbp]
\centering
\capstart

\noindent\sphinxincludegraphics{{key_pad}.jpg}
\caption{\sphinxstylestrong{Figure 8:} Keypad}\label{\detokenize{user_documentation/Hutch-1_laser:id7}}
\begin{sphinxlegend}
\sphinxAtStartPar
This is the key pad that is used to enter a room that is armed as a laser lab.
This keypad has a primary pin for permanent users, and a secondary pin for temporary users that is meant to be changed frequently.
\end{sphinxlegend}
\end{figure}

\begin{figure}[htbp]
\centering
\capstart

\noindent\sphinxincludegraphics{{door_monitor}.jpg}
\caption{\sphinxstylestrong{Figure 9:} Door Monitor Module}\label{\detokenize{user_documentation/Hutch-1_laser:id8}}
\begin{sphinxlegend}
\sphinxAtStartPar
This module is used to monitor the state of a door or curtain.
It will display does not show \DUrole{green}{CLOSED}, then it is open.
If the system is put into an administrative override state, then the door monitor will always show \DUrole{green}{CLOSED}.
\end{sphinxlegend}
\end{figure}

\begin{figure}[htbp]
\centering
\capstart

\noindent\sphinxincludegraphics{{e_stop}.png}
\caption{\sphinxstylestrong{Figure 10:} Laser Emergency Stop Button.}\label{\detokenize{user_documentation/Hutch-1_laser:id9}}
\begin{sphinxlegend}
\sphinxAtStartPar
All the laser enclosures are equipped with laser emergency stop buttons.
The e\sphinxhyphen{}stops on an enclosure can only crash that specific laser.

\sphinxAtStartPar
Additionally, there are ionizing radiation emergency stop buttons in Vault\sphinxhyphen{}1 and Vault\sphinxhyphen{}1 Control.
Those only serve the purpose of crashing the transmitters and are not located on the laser enclosures.

\sphinxAtStartPar
When the e\sphinxhyphen{}stop is pressed, the LED in the center will turn on.
To reset the e\sphinxhyphen{}stop, twist the button clockwise.
\end{sphinxlegend}
\end{figure}


\subsection{Arming Hutch\sphinxhyphen{}1 Laser Systems}
\label{\detokenize{user_documentation/Hutch-1_laser:arming-hutch-1-laser-systems}}
\sphinxAtStartPar
The laser systems that can be armed are Hutch\sphinxhyphen{}1, the Astrella enclosure, and the Astrella laser.


\subsubsection{Arming the Astrella Enclosure and Laser}
\label{\detokenize{user_documentation/Hutch-1_laser:arming-the-astrella-enclosure-and-laser}}
\sphinxAtStartPar
Because the Astrella is contained in an enclosure, Hutch\sphinxhyphen{}1 does not need to be armed to arm the Astrella enclosure and laser.
To arm the Astrella enclosure, press arm on the room interlock module on the south wall of the enclosure.
The room interlock module arms the enclosure.

\sphinxAtStartPar
Once the Astrella enclosure is armed, the Astrella laser can be armed with the local interlock module to the right of the room interlock module.

\sphinxAtStartPar
Once the Astrella system is armed the following will change in the interlock system:
\begin{itemize}
\item {} 
\sphinxAtStartPar
Beacon stack white LEDs will turn on, indicating that the Astrella is armed.

\item {} 
\sphinxAtStartPar
The VIEWMARQ display will show \DUrole{red}{ASTRELLA ARMED}.

\item {} 
\sphinxAtStartPar
The laser warning module on the Astrella enclosure protocase will show \DUrole{red}{DANGER LASER ON}.

\item {} 
\sphinxAtStartPar
Astrella enclosure e\sphinxhyphen{}stop buttons will turn on, shown by the LED in the center.
The e\sphinxhyphen{}stops are now functional and will crash the Astrella laser if pressed.

\end{itemize}


\begin{savenotes}\sphinxattablestart
\centering
\begin{tabulary}{\linewidth}[t]{|T|T|}
\hline

\noindent{\hspace*{\fill}\sphinxincludegraphics[scale=0.2]{{Astrella_enclosure_unarmed}.jpg}\hspace*{\fill}}
&
\noindent{\hspace*{\fill}\sphinxincludegraphics[scale=0.2]{{Astrella_enclosure_armed}.jpg}\hspace*{\fill}}
\\
\hline
\sphinxAtStartPar
Astrella enclosure in an unarmed state. \DUrole{white-cell}{===================================================}
&
\sphinxAtStartPar
Astrella enclosure in an armed state. \DUrole{white-cell}{=====================================================}
\\
\hline
\end{tabulary}
\par
\sphinxattableend\end{savenotes}

\begin{sphinxuseclass}{tight-table-caption-container}
\sphinxAtStartPar
\sphinxstylestrong{Figure 11:} Astrella enclosure armed and unarmed.

\end{sphinxuseclass}

\subsubsection{Arming Hutch\sphinxhyphen{}1}
\label{\detokenize{user_documentation/Hutch-1_laser:arming-hutch-1}}
\sphinxAtStartPar
If work with a live laser must be performed in the enclosure, then Hutch\sphinxhyphen{}1 must be armed as a laser lab.
To arm Hutch\sphinxhyphen{}1 as a laser lab, press arm on the room interlock module in the Hutch\sphinxhyphen{}1 entry.
Unlike for the ionizing radiation interlock system, the laser system does not have search buttons, however it is the responsibility of the user to ensure that Vault\sphinxhyphen{}1 is cleared or proper PPE is distributed before arming Hutch\sphinxhyphen{}1.
Once Hutch\sphinxhyphen{}1 is armed, the laser curtain door must be closed.
The curtain door is not interlocked and it strictly the responsibility of the user.
When the laser curtain is open, there will be a chiming prompting the user to close the curtain and informing people outside that the curtain is open.


\begin{savenotes}\sphinxattablestart
\centering
\begin{tabulary}{\linewidth}[t]{|T|T|}
\hline

\noindent{\hspace*{\fill}\sphinxincludegraphics[scale=0.2]{{Hutch-1_unarmed}.jpg}\hspace*{\fill}}
&
\noindent{\hspace*{\fill}\sphinxincludegraphics[scale=0.2]{{Hutch-1_armed}.jpg}\hspace*{\fill}}
\\
\hline
\sphinxAtStartPar
Hutch\sphinxhyphen{}1 in an unarmed state. \DUrole{white-cell}{=========================================================}
&
\sphinxAtStartPar
Hutch\sphinxhyphen{}1 in an armed state. \DUrole{white-cell}{===========================================================}
\\
\hline
\end{tabulary}
\par
\sphinxattableend\end{savenotes}

\begin{sphinxuseclass}{tight-table-caption-container}
\sphinxAtStartPar
\sphinxstylestrong{Figure 12:} Hutch\sphinxhyphen{}1 armed and unarmed.

\end{sphinxuseclass}
\sphinxAtStartPar
These are the laser safety modules in the Hutch\sphinxhyphen{}1 entry.
On the left are the modules in an unarmed state, and on the right are the modules in an armed state.
In these images, from the top down are the laser control module (serving as a warning module), the push to exit button, and the room interlock module (arms Hutch\sphinxhyphen{}1 as a laser lab).

\sphinxAtStartPar
Once Hutch\sphinxhyphen{}1 is armed as a laser lab the following will change in the interlock system:
\begin{itemize}
\item {} 
\sphinxAtStartPar
Beacon stack \DUrole{green}{green} LEDs will turn off, indicating that Hutch\sphinxhyphen{}1 is armed.

\item {} 
\sphinxAtStartPar
The VIEWMARQ display will show \DUrole{red}{DANGER LASER HAZARD} in place of LASER SAFE.

\item {} 
\sphinxAtStartPar
The laser warning modules in Hutch\sphinxhyphen{}1 Control and Hutch\sphinxhyphen{}1 Entry will show \DUrole{red}{DANGER LASER ON}.

\item {} 
\sphinxAtStartPar
The push to exit button will be on, shown by the LED in the button.

\item {} 
\sphinxAtStartPar
The Hutch\sphinxhyphen{}1 curtain door will be magnetically locked.

\end{itemize}

\sphinxAtStartPar
Once Hutch\sphinxhyphen{}1 is armed the door is magnetically locked.
To get into Hutch\sphinxhyphen{}1, you must type the Vault\sphinxhyphen{}1 laser pin into the keypad in Hutch\sphinxhyphen{}1 Control.
Once the pin is entered, the door will temporarily unlock.
To exit Hutch\sphinxhyphen{}1, you must push the push to exit button, which will again temporarily unlock the door.
It is important to note that Vault\sphinxhyphen{}1 will disarm itself and shutter all exposed laser hazards if the Hutch\sphinxhyphen{}1 door is open for longer than the timer.


\begin{savenotes}\sphinxattablestart
\centering
\begin{tabulary}{\linewidth}[t]{|T|T|}
\hline

\noindent{\hspace*{\fill}\sphinxincludegraphics[scale=0.2]{{Hutch-1_entry_disarmed}.jpg}\hspace*{\fill}}
&
\noindent{\hspace*{\fill}\sphinxincludegraphics[scale=0.2]{{Hutch-1_entry_armed}.jpg}\hspace*{\fill}}
\\
\hline
\sphinxAtStartPar
Hutch\sphinxhyphen{}1 entry in an unarmed state. \DUrole{white-cell}{======================================================}
&
\sphinxAtStartPar
Hutch\sphinxhyphen{}1 entry in an armed state. \DUrole{white-cell}{========================================================}
\\
\hline
\end{tabulary}
\par
\sphinxattableend\end{savenotes}

\begin{sphinxuseclass}{tight-table-caption-container}
\sphinxAtStartPar
\sphinxstylestrong{Figure 13:} Hutch\sphinxhyphen{}1 entry armed and unarmed.

\end{sphinxuseclass}

\subsection{Using Laser Enclosure Interlock Protocase for Overriding Interlocks and Manual Shutter Control}
\label{\detokenize{user_documentation/Hutch-1_laser:using-laser-enclosure-interlock-protocase-for-overriding-interlocks-and-manual-shutter-control}}
\sphinxAtStartPar
The shutters in the laser enclosures can be armed for manual control by the protocase LOCAL INTERLOCK CONTRACT CONTROL local interlock modules.
However, when the laser enclosures are interlocked, regardless of the arming status of the enclosure and Hutch\sphinxhyphen{}1, if someone attempts to open the enclosure doors the shutters will close.

\sphinxAtStartPar
What you will see happen on the enclosure protocase if the door is opened when interlocked is:
\begin{itemize}
\item {} 
\sphinxAtStartPar
Laser warning module will show \DUrole{green}{LASER SAFE}.

\item {} 
\sphinxAtStartPar
Door monitor module will be blank, meaning open.

\item {} 
\sphinxAtStartPar
LOCAL INTERLOCK CONTACT CONTROL local interlock modules will disarm if armed, automatically closing the shutters.

\end{itemize}


\subsubsection{Interlock to Override}
\label{\detokenize{user_documentation/Hutch-1_laser:interlock-to-override}}
\sphinxAtStartPar
The only way to work in the laser enclosures with light on the table is to change the enclosures interlocks to administrative override.
In administrative override the interlocks system sees the rolling doors and closed even if they are opened, bypassing the interlocks.

\sphinxAtStartPar
For a laser enclosure to be put into administrative override, both Hutch\sphinxhyphen{}1 and the enclosure must be armed.
The controls for the administrative overrides are on the enclosures LASER ENCLOSURE INTERLOCK protocase.
Turn the key on the protocase under INTERLOCK OVERRIDE from INTERLOCK TO OVERRIDE.

\sphinxAtStartPar
Once the enclosure is put into override the following will change in the interlock system:
\begin{itemize}
\item {} 
\sphinxAtStartPar
The VIEWMARQ display will show \DUrole{red}{ASTRELLA ADMIN OVERRIDE}.

\item {} 
\sphinxAtStartPar
The Hutch\sphinxhyphen{}1 Control and Hutch\sphinxhyphen{}1 protocase beacon stack \DUrole{orange}{orange} administrative override LED will turn on.

\item {} 
\sphinxAtStartPar
If you open the enclosure,  the laser warning module will still show \DUrole{red}{DANGER LASER ON}, the door monitor module will show \DUrole{green}{CLOSED}, and the local interlock modules for arming shutters manual control will not disarm.

\end{itemize}

\sphinxAtStartPar
At this point, the LOCAL INTERLOCK CONTACT CONTROL local interlock modules can be armed, and the shutters can be controlled manually without the interlocks disarming manual usage.

\begin{figure}[htbp]
\centering
\capstart

\noindent\sphinxincludegraphics[scale=0.2]{{Astrella_override}.jpg}
\caption{\sphinxstylestrong{Figure 14:} Astrella enclosure override.}\label{\detokenize{user_documentation/Hutch-1_laser:id10}}\end{figure}


\subsection{Disarming the Laser Interlock System}
\label{\detokenize{user_documentation/Hutch-1_laser:disarming-the-laser-interlock-system}}
\sphinxAtStartPar
To take the Astrella enclosure out of administrative override, simply change the INTERLOCK OVERRIDE key on the Astrella LASER ENCLOSURE INTERLOCK protocase back from OVERRIDE to INTERLOCK.
Also, all the arming laser modules have disarming buttons where you can either disarm specific modules you no longer need, or you can disarm the room modules to auto\sphinxhyphen{}disarm their local modules.

\sphinxstepscope


\section{Laser\sphinxhyphen{}1 Interlock System User Manual}
\label{\detokenize{user_documentation/Laser-1:laser-1-interlock-system-user-manual}}\label{\detokenize{user_documentation/Laser-1::doc}}
\sphinxAtStartPar
This document provides a user overview for the Laser\sphinxhyphen{}1 Laser Interlock System.
It covers the laser modules, hazard indicators, control protocases, VIEWMARQ display, and the beacons in Vault\sphinxhyphen{}1 and Vault\sphinxhyphen{}1 Control.
The laser modules are used for indicating the status of the laser interlock system and arming the laser systems.
The control protocase allows the users to view the override state of the laser enclosure and control the manual shutter control.
The VIEWMARQ display provides a quick overview of the laser hazard status in Vault\sphinxhyphen{}1.
The beacons are used to indicate the state of laser hazards in Vault\sphinxhyphen{}1 and the laser enclosures.


\subsection{Laser\sphinxhyphen{}1 Laser Hazard Warning Indicators}
\label{\detokenize{user_documentation/Laser-1:laser-1-laser-hazard-warning-indicators}}
\sphinxAtStartPar
Unlike the Vault\sphinxhyphen{}1 and Hutch\sphinxhyphen{}1 laser interlocks systems, there are no beacon stacks for displaying the state of the laser interlock system.
Additionally, because the Dira enclosure is in Vault\sphinxhyphen{}1 there is no protocase corresponding to Laser\sphinxhyphen{}1.


\subsubsection{VIEWMARQ Display}
\label{\detokenize{user_documentation/Laser-1:viewmarq-display}}
\sphinxAtStartPar
There is a VIEWMARQ display outside of the Laser\sphinxhyphen{}1 entrance and inside of the Laser\sphinxhyphen{}1 airlock.


\begin{savenotes}\sphinxattablestart
\centering
\begin{tabulary}{\linewidth}[t]{|T|T|}
\hline
\sphinxstyletheadfamily 
\sphinxAtStartPar
VIEWMARQ Display Notes
&\sphinxstyletheadfamily 
\sphinxAtStartPar
VIEWMARQ Display Text
\\
\hline
\sphinxAtStartPar
This states if Laser\sphinxhyphen{}1 is armed as a laser lab or not.
&
\sphinxAtStartPar
\DUrole{green}{LASER SAFE} / \DUrole{red}{DANGER LASER HAZARD}
\\
\hline
\sphinxAtStartPar
This states which hazard is armed.
&
\sphinxAtStartPar
\DUrole{red}{IR HAZARD}        \DUrole{red}{AUX 3 HAZARD}
\\
\hline
\sphinxAtStartPar
This states which hazard is armed.
&
\sphinxAtStartPar
\DUrole{red}{UV HAZARD}     \DUrole{red}{AUX 4 HAZARD}
\\
\hline
\sphinxAtStartPar
This states which hazard is armed.
&
\sphinxAtStartPar
\DUrole{red}{AUX 2 HAZARD}
\\
\hline
\end{tabulary}
\par
\sphinxattableend\end{savenotes}

\sphinxAtStartPar
The top line of the Laser\sphinxhyphen{}1 entry VIEWMARQ will always either display \DUrole{green}{LASER SAFE} or \DUrole{red}{DANGER LASER HAZARD}.
All other possible states will only appear on the display when the hazard is presented.
Additionally, the VIEWMARQ display in the Laser\sphinxhyphen{}1 airlock will display the same message as the VIEWMARQ outside of Laser\sphinxhyphen{}1.
The airlock VIEWMARQ can only support one\sphinxhyphen{}line messages, so the whole message is truncated to one line and moves across the display.


\begin{savenotes}\sphinxattablestart
\centering
\begin{tabulary}{\linewidth}[t]{|T|T|T|}
\hline

\noindent{\hspace*{\fill}\sphinxincludegraphics[scale=0.2]{{Laser-1_VIEWMARQ_entry_safe}.jpg}\hspace*{\fill}}
&
\noindent{\hspace*{\fill}\sphinxincludegraphics[scale=0.2]{{Laser-1_VIEWMARQ_entry_armed}.jpg}\hspace*{\fill}}
&
\noindent{\hspace*{\fill}\sphinxincludegraphics[scale=0.2]{{Laser-1_VIEWMARQ_entry_IR}.jpg}\hspace*{\fill}}
\\
\hline
\sphinxAtStartPar
This figure shows the Laser\sphinxhyphen{}1 entry VIEWMARQ in a safe condition. \DUrole{white-cell}{============================}
&
\sphinxAtStartPar
This figure shows the Laser\sphinxhyphen{}1 entry VIEWMARQ when Laser\sphinxhyphen{}1 is armed. \DUrole{white-cell}{==========================}
&
\sphinxAtStartPar
This figure shows the Laser\sphinxhyphen{}1 entry VIEWMARQ when the Dira is armed. \DUrole{white-cell}{=========================}
\\
\hline
\end{tabulary}
\par
\sphinxattableend\end{savenotes}

\begin{sphinxuseclass}{tight-table-caption-container}
\sphinxAtStartPar
\sphinxstylestrong{Figure 1:} This is the Laser\sphinxhyphen{}1 entry VIEWMARQ in different states.

\end{sphinxuseclass}

\begin{savenotes}\sphinxattablestart
\centering
\begin{tabulary}{\linewidth}[t]{|T|T|T|}
\hline

\noindent{\hspace*{\fill}\sphinxincludegraphics[scale=0.2]{{Laser-1_VIEWMARQ_airlock_safe}.jpg}\hspace*{\fill}}
&
\noindent{\hspace*{\fill}\sphinxincludegraphics[scale=0.2]{{Laser-1_VIEWMARQ_airlock_armed}.jpg}\hspace*{\fill}}
&
\noindent{\hspace*{\fill}\sphinxincludegraphics[scale=0.56]{{Laser-1_VIEWMARQ_airlock_IR}.gif}\hspace*{\fill}}
\\
\hline
\sphinxAtStartPar
This figure shows the Laser\sphinxhyphen{}1 airlock VIEWMARQ in a safe condition. \DUrole{white-cell}{============================}
&
\sphinxAtStartPar
This figure shows the Laser\sphinxhyphen{}1 airlock VIEWMARQ when Laser\sphinxhyphen{}1 is armed. \DUrole{white-cell}{==========================}
&
\sphinxAtStartPar
This figure shows the Laser\sphinxhyphen{}1 airlock VIEWMARQ when the Dira is armed. \DUrole{white-cell}{=========================}
\\
\hline
\end{tabulary}
\par
\sphinxattableend\end{savenotes}

\begin{sphinxuseclass}{tight-table-caption-container}
\sphinxAtStartPar
\sphinxstylestrong{Figure 2:} This is the Laser\sphinxhyphen{}1 airlock VIEWMARQ in different states.

\end{sphinxuseclass}

\subsubsection{Laser Safety System Modules}
\label{\detokenize{user_documentation/Laser-1:laser-safety-system-modules}}
\sphinxAtStartPar
The laser interlock system is interfaced through the laser safety systems modules. Below is an outline of the modules and what they do.

\begin{figure}[htbp]
\centering
\capstart

\noindent\sphinxincludegraphics{{warning_module}.gif}
\caption{\sphinxstylestrong{Figure 3:} Area Warming Module}\label{\detokenize{user_documentation/Laser-1:id1}}\end{figure}


\begin{savenotes}\sphinxattablestart
\centering
\begin{tabular}[t]{|*{2}{\X{1}{2}|}}
\hline
\sphinxstyletheadfamily 
\sphinxAtStartPar
Module Location
&\sphinxstyletheadfamily 
\sphinxAtStartPar
Module Meaning
\\
\hline
\begin{DUlineblock}{0em}
\item[] \sphinxstylestrong{General Area Module}
\item[] Vault\sphinxhyphen{}1 Control
\item[] Vault\sphinxhyphen{}1 Entry
\end{DUlineblock}
&
\begin{DUlineblock}{0em}
\item[] These are warning modules tell you if Vault\sphinxhyphen{}1 is armed as a laser lab.
\item[] \DUrole{red}{DANGER LASER ON} = ARMED
\end{DUlineblock}
\\
\hline
\begin{DUlineblock}{0em}
\item[] \sphinxstylestrong{Enclosure Modules}
\item[] Pharos enclosure south wall
\item[] Pharos enclosure west wall
\end{DUlineblock}
&
\begin{DUlineblock}{0em}
\item[] These warning modules tell you if the enclosure is armed.
\item[] There is no indication on if the laser is armed.
\item[] \DUrole{red}{DANGER LASER ON} = ARMED
\end{DUlineblock}
\\
\hline
\begin{DUlineblock}{0em}
\item[] \sphinxstylestrong{Protocase Modules}
\item[] Pharos enclosure protocase
\item[] Dira enclosure protocase
\end{DUlineblock}
&
\begin{DUlineblock}{0em}
\item[] These warning modules tell you if the enclosure is forced to a safe state.
\item[] \DUrole{red}{DANGER LASER HAZARD} = SAFE STATE IS NOT FORCED
\end{DUlineblock}
\\
\hline
\end{tabular}
\par
\sphinxattableend\end{savenotes}

\begin{figure}[htbp]
\centering
\capstart

\noindent\sphinxincludegraphics{{control_module}.gif}
\caption{\sphinxstylestrong{Figure 4:} Control Module}\label{\detokenize{user_documentation/Laser-1:id2}}
\begin{sphinxlegend}
\sphinxAtStartPar
This module is a control module for the local laser interlock, however, for the users it serves as another warning module.
This warning module tells you if the room interlock is armed or not.
\end{sphinxlegend}
\end{figure}

\begin{figure}[htbp]
\centering
\capstart

\noindent\sphinxincludegraphics{{room_arm}.png}
\caption{\sphinxstylestrong{Figure 5:} Room Arm Module}\label{\detokenize{user_documentation/Laser-1:id3}}
\begin{sphinxlegend}
\sphinxAtStartPar
This module is used to arm systems in the laser interlock system.
For example, there are two in Vault\sphinxhyphen{}1, one to arm the vault and one to arm the Pharos enclosure.
\end{sphinxlegend}
\end{figure}


\begin{savenotes}\sphinxattablestart
\centering
\begin{tabular}[t]{|*{2}{\X{1}{2}|}}
\hline
\sphinxstyletheadfamily 
\sphinxAtStartPar
Module Message
&\sphinxstyletheadfamily 
\sphinxAtStartPar
Message Meaning
\\
\hline
\sphinxAtStartPar
\DUrole{orange}{ROOM ARMED}
&
\begin{DUlineblock}{0em}
\item[] If this LED is on, then the corresponding system is armed
\item[] and interlocked.
\end{DUlineblock}
\\
\hline
\sphinxAtStartPar
\DUrole{green}{ROOM DISARMED (READY TO ARM)}
&
\begin{DUlineblock}{0em}
\item[] If this LED is on, the the system is the correct state to armed
\item[] the module.
\end{DUlineblock}
\\
\hline
\sphinxAtStartPar
\DUrole{orange}{ROOM CRASHED (CANNOT ARM)}
&
\begin{DUlineblock}{0em}
\item[] If this LED is on, then there was a fault that tripped the
\item[] system, or a fault that will not allow the system to be armed.
\end{DUlineblock}
\\
\hline
\end{tabular}
\par
\sphinxattableend\end{savenotes}

\begin{figure}[htbp]
\centering
\capstart

\noindent\sphinxincludegraphics{{local_arm}.png}
\caption{\sphinxstylestrong{Figure 6:} Local Arm Module}\label{\detokenize{user_documentation/Laser-1:id4}}
\begin{sphinxlegend}
\sphinxAtStartPar
This module is used to arm the local interlock modules that are sub\sphinxhyphen{}systems of the room arm modules.
For example, one the Pharos enclosure is armed, it enables the laser and the shutters to be armed by their local arming modules.
\end{sphinxlegend}
\end{figure}


\begin{savenotes}\sphinxattablestart
\centering
\begin{tabular}[t]{|*{2}{\X{1}{2}|}}
\hline
\sphinxstyletheadfamily 
\sphinxAtStartPar
Module Message
&\sphinxstyletheadfamily 
\sphinxAtStartPar
Message Meaning
\\
\hline
\sphinxAtStartPar
\DUrole{orange}{LOCAL CONTACTS ARMED}
&
\begin{DUlineblock}{0em}
\item[] If this LED is on, then the corresponding sub\sphinxhyphen{}system is armed.
\end{DUlineblock}
\\
\hline
\sphinxAtStartPar
\DUrole{green}{LOCAL CONTACTS DISARMED}
&
\begin{DUlineblock}{0em}
\item[] If this LED is on, then the connected room module is armed,
\item[] but this module is not.
\end{DUlineblock}
\\
\hline
\begin{DUlineblock}{0em}
\item[] \DUrole{green}{LOCAL CONTACTS DISARMED}
\item[] \DUrole{green}{ROOM NOT ARMED}
\item[] \DUrole{green}{LOCAL CONTACT CANNOT ARM}
\end{DUlineblock}
&
\begin{DUlineblock}{0em}
\item[] If this LED is on, then there was a fault that tripped, or the room
\item[] module is not armed.
\end{DUlineblock}
\\
\hline
\end{tabular}
\par
\sphinxattableend\end{savenotes}

\begin{figure}[htbp]
\centering
\capstart

\noindent\sphinxincludegraphics{{push_to_exit}.png}
\caption{\sphinxstylestrong{Figure 7:} Push to Exit Module}\label{\detokenize{user_documentation/Laser-1:id5}}
\begin{sphinxlegend}
\sphinxAtStartPar
When the rooms are armed, the doors are magnetically locked.
This button will temporarily unlock the door to allow you to exit the room.
\end{sphinxlegend}
\end{figure}

\begin{figure}[htbp]
\centering
\capstart

\noindent\sphinxincludegraphics{{key_pad}.jpg}
\caption{\sphinxstylestrong{Figure 8:} Keypad}\label{\detokenize{user_documentation/Laser-1:id6}}
\begin{sphinxlegend}
\sphinxAtStartPar
This is the key pad that is used to enter a room that is armed as a laser lab.
This keypad has a primary pin for permanent users, and a secondary pin for temporary users that is meant to be changed frequently.
\end{sphinxlegend}
\end{figure}

\begin{figure}[htbp]
\centering
\capstart

\noindent\sphinxincludegraphics{{door_monitor}.jpg}
\caption{\sphinxstylestrong{Figure 9:} Door Monitor Module}\label{\detokenize{user_documentation/Laser-1:id7}}
\begin{sphinxlegend}
\sphinxAtStartPar
This module is used to monitor the state of a door or curtain.
It will display does not show \DUrole{green}{CLOSED}, then it is open.
If the system is put into an administrative override state, then the door monitor will always show \DUrole{green}{CLOSED}.
\end{sphinxlegend}
\end{figure}

\begin{figure}[htbp]
\centering
\capstart

\noindent\sphinxincludegraphics{{e_stop}.png}
\caption{\sphinxstylestrong{Figure 10:} Laser Emergency Stop Button.}\label{\detokenize{user_documentation/Laser-1:id8}}
\begin{sphinxlegend}
\sphinxAtStartPar
All the laser enclosures are equipped with laser emergency stop buttons.
The e\sphinxhyphen{}stops on an enclosure can only crash that specific laser.
\end{sphinxlegend}
\end{figure}

\sphinxAtStartPar
Additionally, there are ionizing radiation emergency stop buttons in Vault\sphinxhyphen{}1 and Vault\sphinxhyphen{}1 Control.
Those only serve the purpose of crashing the transmitters and are not located on the laser enclosures.
When the e\sphinxhyphen{}stop is pressed, the LED in the center will turn on.
To reset the e\sphinxhyphen{}stop, twist the button clockwise.


\subsection{Arming Laser\sphinxhyphen{}1 Laser Systems}
\label{\detokenize{user_documentation/Laser-1:arming-laser-1-laser-systems}}
\sphinxAtStartPar
In Laser\sphinxhyphen{}1, there are arming modules for Laser\sphinxhyphen{}1, the Dira, and 4 auxiliary hazard. All arming modules are located inside on the arming panel that faces the west wall.

\begin{figure}[htbp]
\centering
\capstart

\noindent\sphinxincludegraphics[scale=0.2]{{Laser-1_arming_panel}.jpg}
\caption{\sphinxstylestrong{Figure 11:} This is the arming panel for Laser\sphinxhyphen{}1.}\label{\detokenize{user_documentation/Laser-1:id9}}\end{figure}


\subsubsection{Arming Laser\sphinxhyphen{}1 and the Dira Enclosure}
\label{\detokenize{user_documentation/Laser-1:arming-laser-1-and-the-dira-enclosure}}
\sphinxAtStartPar
Laser\sphinxhyphen{}1 must be armed to perform work with the Dira. To arm Laser\sphinxhyphen{}1, press arm on the room interlock module labeled Laser\sphinxhyphen{}1.
The Laser\sphinxhyphen{}1 arming module also serves as the arming module for the Dira enclosure.

\sphinxAtStartPar
When Laser\sphinxhyphen{}1 is armed, the following will happen to the interlock system:
\begin{itemize}
\item {} 
\sphinxAtStartPar
The Laser\sphinxhyphen{}1 VIEWMARQ displays will show \DUrole{red}{DANGER LASER HAZARD}.

\item {} 
\sphinxAtStartPar
Laser emergency stop buttons in Laser\sphinxhyphen{}1 and around the Vault\sphinxhyphen{}1 Dira enclosure will turn on.

\item {} 
\sphinxAtStartPar
Laser warning modules outside of Laser\sphinxhyphen{}1, inside of the Laser\sphinxhyphen{}1 airlock, and on the Dira LASER ENCLOSURE INTERLOCK protocase will display \DUrole{red}{DANGER LASER ON}.

\end{itemize}

\sphinxAtStartPar
Once Laser\sphinxhyphen{}1 is armed the door is magnetically locked.
To get into Vault\sphinxhyphen{}1, you must type the Laser\sphinxhyphen{}1 laser pin into the keypad in Vault\sphinxhyphen{}1 Control.
Once it is entered the door will be temporarily unlocked.
To exit Laser\sphinxhyphen{}1, you must push the push to exit button. Once pressed the door will be temporarily unlocked.
It is important to note that Laser\sphinxhyphen{}1 will disarm itself and shutter all laser hazards if the Laser\sphinxhyphen{}1 door is open for  \#.


\begin{savenotes}\sphinxattablestart
\centering
\begin{tabulary}{\linewidth}[t]{|T|T|}
\hline

\noindent{\hspace*{\fill}\sphinxincludegraphics[scale=0.2]{{Laser-1_control_module_safe}.jpg}\hspace*{\fill}}
&
\noindent{\hspace*{\fill}\sphinxincludegraphics[scale=0.2]{{Laser-1_control_module_armed}.jpg}\hspace*{\fill}}
\\
\hline
\sphinxAtStartPar
This figure shows the Laser\sphinxhyphen{}1 control module in the safe state. \DUrole{white-cell}{================================}
&
\sphinxAtStartPar
This figure shows the Laser\sphinxhyphen{}1 control module in the armed state. \DUrole{white-cell}{===============================}
\\
\hline
\end{tabulary}
\par
\sphinxattableend\end{savenotes}

\begin{sphinxuseclass}{tight-table-caption-container}
\sphinxAtStartPar
\sphinxstylestrong{Figure 12:} This is the Laser\sphinxhyphen{}1 arming module in different states.

\end{sphinxuseclass}

\begin{savenotes}\sphinxattablestart
\centering
\begin{tabulary}{\linewidth}[t]{|T|T|}
\hline

\noindent{\hspace*{\fill}\sphinxincludegraphics[scale=0.2]{{Laser-1_push_to_exit_safe}.jpg}\hspace*{\fill}}
&
\noindent{\hspace*{\fill}\sphinxincludegraphics[scale=0.2]{{Laser-1_push_to_exit}.jpg}\hspace*{\fill}}
\\
\hline
\sphinxAtStartPar
This figure shows the Laser\sphinxhyphen{}1 push to exit button in the safe state. \DUrole{white-cell}{==============================}
&
\sphinxAtStartPar
This figure shows the Laser\sphinxhyphen{}1 push to exit button in the armed state. \DUrole{white-cell}{=============================}
\\
\hline
\end{tabulary}
\par
\sphinxattableend\end{savenotes}

\begin{sphinxuseclass}{tight-table-caption-container}
\sphinxAtStartPar
\sphinxstylestrong{Figure 13:} This is the Laser\sphinxhyphen{}1 push to exit button in different states.

\end{sphinxuseclass}

\begin{savenotes}\sphinxattablestart
\centering
\begin{tabulary}{\linewidth}[t]{|T|T|}
\hline

\noindent{\hspace*{\fill}\sphinxincludegraphics[scale=0.2]{{Laser-1_entry_safe}.jpg}\hspace*{\fill}}
&
\noindent{\hspace*{\fill}\sphinxincludegraphics[scale=0.2]{{Laser-1_entry_armed}.jpg}\hspace*{\fill}}
\\
\hline
\sphinxAtStartPar
This figure shows the Laser\sphinxhyphen{}1 entryway with the modules showing LASER SAFE. \DUrole{white-cell}{===================}
&
\sphinxAtStartPar
This figure shows the Laser\sphinxhyphen{}1 entryway with the modules showing DANGER LASER HAZARD. \DUrole{white-cell}{==========}
\\
\hline
\end{tabulary}
\par
\sphinxattableend\end{savenotes}

\begin{sphinxuseclass}{tight-table-caption-container}
\sphinxAtStartPar
\sphinxstylestrong{Figure 14:} This is the Laser\sphinxhyphen{}1 entryway with the modules showing different states.

\end{sphinxuseclass}

\subsubsection{Arming the Dira}
\label{\detokenize{user_documentation/Laser-1:arming-the-dira}}
\sphinxAtStartPar
To arm the Dira, press arm on the local interlock module labeled Dira.
Because the Dira exports a laser hazard into Vault\sphinxhyphen{}1, Vault\sphinxhyphen{}1 laser hazard indicators will update.

\sphinxAtStartPar
When the Dira is armed, the following will happen to the interlock system:
\begin{itemize}
\item {} 
\sphinxAtStartPar
The Laser\sphinxhyphen{}1 entry VIEWMARQ display will show \DUrole{red}{IR HAZARD}.

\item {} 
\sphinxAtStartPar
The Laser\sphinxhyphen{}1 airlock VIEWMARQ display wil show \DUrole{red}{DANGER LASER ON \textendash{} IR EYE PROTECTION REQUIRED}.

\item {} 
\sphinxAtStartPar
The Vault\sphinxhyphen{}1 Control VIEWMARQ display will show \DUrole{red}{DIRA ARMED}.

\item {} 
\sphinxAtStartPar
The beacon stacks in the Vault\sphinxhyphen{}1 laser interlock system will turn on the white Dira armed LED.

\end{itemize}

\sphinxAtStartPar
If the Dira is disarmed from either the arming panel or from a laser e\sphinxhyphen{}stop, the Dira will not rearm normally.
You must first toggle one of the e\sphinxhyphen{}stops that are on the Dira laser.
After this, the Dira can be properly armed from the arming panel.


\begin{savenotes}\sphinxattablestart
\centering
\begin{tabulary}{\linewidth}[t]{|T|T|T|}
\hline

\noindent{\hspace*{\fill}\sphinxincludegraphics[scale=0.2]{{Dira_disarmed}.jpg}\hspace*{\fill}}
&
\noindent{\hspace*{\fill}\sphinxincludegraphics[scale=0.2]{{Dira_incorrect_arm}.jpg}\hspace*{\fill}}
&
\noindent{\hspace*{\fill}\sphinxincludegraphics[scale=0.2]{{Dira_armed}.jpg}\hspace*{\fill}}
\\
\hline
\sphinxAtStartPar
This shows the Dira control panel when the Dira is disarmed.  \DUrole{white-cell}{===============================}
&
\sphinxAtStartPar
This shows the Dira control panel after attempting to rearm without toggling a Dira e\sphinxhyphen{}stop.
&
\sphinxAtStartPar
This shows the Dira control panel when the Dira is armed. \DUrole{white-cell}{===================================}
\\
\hline
\end{tabulary}
\par
\sphinxattableend\end{savenotes}

\begin{sphinxuseclass}{tight-table-caption-container}
\sphinxAtStartPar
\sphinxstylestrong{Figure 15:} This shows the Dira control panel software under different arming states. Here you can see where the issue arises, causing a need for the Dira to be reset before it can be armed again.

\end{sphinxuseclass}

\subsection{Arming Auxiliary hazards}
\label{\detokenize{user_documentation/Laser-1:arming-auxiliary-hazards}}
\begin{sphinxadmonition}{note}{Note:}
\sphinxAtStartPar
The auxiliary hazards are not currently in use.
\end{sphinxadmonition}


\subsection{Disarming the Laser Interlock System}
\label{\detokenize{user_documentation/Laser-1:disarming-the-laser-interlock-system}}
\sphinxAtStartPar
All the arming laser modules have disarming buttons.
You can either disarm specific modules you no longer need, or you can disarm the room modules to auto\sphinxhyphen{}disarm their local modules.

\sphinxstepscope


\section{Ionizing Radiation Surveys and Results}
\label{\detokenize{user_documentation/radiation_data:ionizing-radiation-surveys-and-results}}\label{\detokenize{user_documentation/radiation_data::doc}}
\sphinxAtStartPar
This section provides a summary of the radiation data that has been collected in the CXLS suite.


\subsection{Routine Radiation Survey of Vault\sphinxhyphen{}1 and Beam Line}
\label{\detokenize{user_documentation/radiation_data:routine-radiation-survey-of-vault-1-and-beam-line}}
\sphinxAtStartPar
Radiation surveys of the CXLS beamline and Vault\sphinxhyphen{}1 are taken routinely every week and after the accelerator is run.
Below are maps of the locations where radiation measurements are taken using the Ludlum 9DP.

\begin{figure}[htbp]
\centering
\capstart

\noindent\sphinxincludegraphics{{beamline_map}.JPG}
\caption{\sphinxstylestrong{Figure 1:} Survey points along the beamline.}\label{\detokenize{user_documentation/radiation_data:id1}}\end{figure}

\begin{figure}[htbp]
\centering
\capstart

\noindent\sphinxincludegraphics{{Vault-1_map}.JPG}
\caption{\sphinxstylestrong{Figure 2:} Survey points throughout Vault\sphinxhyphen{}1.}\label{\detokenize{user_documentation/radiation_data:id2}}\end{figure}

\begin{figure}[htbp]
\centering
\capstart

\noindent\sphinxincludegraphics{{Accelerator_Lab_map}.JPG}
\caption{\sphinxstylestrong{Figure 3:} Survey point in the Accelerator Lab for reference.}\label{\detokenize{user_documentation/radiation_data:id3}}\end{figure}

\sphinxAtStartPar
Below is a summary of the data collected from the radiation surveys.

\begin{figure}[htbp]
\centering
\capstart

\noindent\sphinxincludegraphics{{contact_distance_average}.png}
\caption{\sphinxstylestrong{Figure 4:} Average radiation survey readings at contact and 30 cm from beamline.}\label{\detokenize{user_documentation/radiation_data:id4}}\end{figure}

\begin{figure}[htbp]
\centering
\capstart

\noindent\sphinxincludegraphics{{area_average}.png}
\caption{\sphinxstylestrong{Figure 5:} Average radiation survey readings in the general area of Vault\sphinxhyphen{}1.}\label{\detokenize{user_documentation/radiation_data:id5}}\end{figure}

\begin{figure}[htbp]
\centering
\capstart

\noindent\sphinxincludegraphics{{dose_rate_distribution}.png}
\caption{\sphinxstylestrong{Figure 6:} Dose rate distribution in Vault\sphinxhyphen{}1.
Note that there is one point that was taken after a long beam time near 1200, however the vast majority of the points are below 20 uSv/hr.}\label{\detokenize{user_documentation/radiation_data:id6}}\end{figure}


\subsection{Radiation Characterization Survey of Vault\sphinxhyphen{}1}
\label{\detokenize{user_documentation/radiation_data:radiation-characterization-survey-of-vault-1}}
\sphinxAtStartPar
After the Apantec area monitors were installed, and the software was developed to communicate and collect data from these devices, a survey was taken of Vault\sphinxhyphen{}1.
The Apantec area monitor in Vault\sphinxhyphen{}1 was made mobile and placed around various locations in Vault\sphinxhyphen{}1.
Data from the surrounding area detector were also collected.
Below are the map of the detector locations and the summary of the data.

\begin{figure}[htbp]
\centering
\capstart

\noindent\sphinxincludegraphics{{Vault-1_cart_overlay}.png}
\caption{\sphinxstylestrong{Figure 7:} This is the map of the locations where Vault\sphinxhyphen{}1 radiation meter was placed.}\label{\detokenize{user_documentation/radiation_data:id7}}\end{figure}


\begin{savenotes}\sphinxattablestart
\centering
\begin{tabulary}{\linewidth}[t]{|T|T|T|}
\hline

\noindent{\hspace*{\fill}\sphinxincludegraphics{{Vault-1_XELERA_Overlay}.png}\hspace*{\fill}}
&
\noindent{\hspace*{\fill}\sphinxincludegraphics{{XELERA_20pC_compare}.png}\hspace*{\fill}}
&
\noindent{\hspace*{\fill}\sphinxincludegraphics{{XELERA_200pC_compare}.png}\hspace*{\fill}}
\\
\hline
\sphinxAtStartPar
Overlay of the XELERA simulations with the cart positions used in the survey.  \DUrole{white-cell}{==============}
&
\sphinxAtStartPar
Comparison of the XELERA simulations with the actual measurements at 20 pC.  \DUrole{white-cell}{================}
&
\sphinxAtStartPar
Comparison of the XELERA simulations with the actual measurements at 200 pC.  \DUrole{white-cell}{===============}
\\
\hline
\end{tabulary}
\par
\sphinxattableend\end{savenotes}


\begin{savenotes}\sphinxattablestart
\centering
\begin{tabulary}{\linewidth}[t]{|T|T|T|T|T|T|}
\hline

\noindent{\hspace*{\fill}\sphinxincludegraphics{{Vault-1_readings_1}.png}\hspace*{\fill}}
&
\noindent{\hspace*{\fill}\sphinxincludegraphics{{Vault-1_readings_2}.png}\hspace*{\fill}}
&
\noindent{\hspace*{\fill}\sphinxincludegraphics{{Vault-1_readings_3}.png}\hspace*{\fill}}
&
\noindent{\hspace*{\fill}\sphinxincludegraphics{{Vault-1_readings_4}.png}\hspace*{\fill}}
&
\noindent{\hspace*{\fill}\sphinxincludegraphics{{Vault-1_readings_5}.png}\hspace*{\fill}}
&
\noindent{\hspace*{\fill}\sphinxincludegraphics{{Vault-1_readings_6}.png}\hspace*{\fill}}
\\
\hline
\sphinxAtStartPar
Radiation survey readings at Vault\sphinxhyphen{}1. \DUrole{white-cell}{=======================================================}
&
\sphinxAtStartPar
Radiation survey readings at Vault\sphinxhyphen{}1. \DUrole{white-cell}{=======================================================}
&
\sphinxAtStartPar
Radiation survey readings at Vault\sphinxhyphen{}1. \DUrole{white-cell}{=======================================================}
&
\sphinxAtStartPar
Radiation survey readings at Vault\sphinxhyphen{}1. \DUrole{white-cell}{=======================================================}
&
\sphinxAtStartPar
Radiation survey readings at Vault\sphinxhyphen{}1. \DUrole{white-cell}{=======================================================}
&
\sphinxAtStartPar
Radiation survey readings at Vault\sphinxhyphen{}1. \DUrole{white-cell}{=======================================================}
\\
\hline
\end{tabulary}
\par
\sphinxattableend\end{savenotes}


\subsection{Radiation Characterization Survey of Hutch\sphinxhyphen{}1}
\label{\detokenize{user_documentation/radiation_data:radiation-characterization-survey-of-hutch-1}}
\sphinxAtStartPar
Once holes were put into the Hutch\sphinxhyphen{}1 / Vault\sphinxhyphen{}1 wall for the divergent and collimated beamline, a survey was taken of Hutch\sphinxhyphen{}1.
Inside of Hutch\sphinxhyphen{}1, radiation measurements were taken near the penetration, inside of the experimental chamber, and behind the detector chamber.
Below are the maps of the detectors, the beam charge and energy, and the summary of the data that was collected and measured above background.

\begin{figure}[htbp]
\centering
\capstart

\noindent\sphinxincludegraphics{{Hutch-1_sensor_positions_1}.png}
\caption{\sphinxstylestrong{Figure 8:} This is the map of the locations where Hutch\sphinxhyphen{}1 radiation meter was placed.}\label{\detokenize{user_documentation/radiation_data:id8}}\end{figure}

\begin{figure}[htbp]
\centering
\capstart

\noindent\sphinxincludegraphics{{Hutch-1_sensor_positions_2}.png}
\caption{\sphinxstylestrong{Figure 9:} This is the map of the locations where Hutch\sphinxhyphen{}1 radiation meter was placed.}\label{\detokenize{user_documentation/radiation_data:id9}}\end{figure}

\begin{figure}[htbp]
\centering
\capstart

\noindent\sphinxincludegraphics{{Hutch-1_beam_levels}.png}
\caption{\sphinxstylestrong{Figure 10:} This is the map of the beam levels in Hutch\sphinxhyphen{}1.}\label{\detokenize{user_documentation/radiation_data:id10}}\end{figure}


\begin{savenotes}\sphinxattablestart
\centering
\begin{tabulary}{\linewidth}[t]{|T|T|T|T|}
\hline

\noindent{\hspace*{\fill}\sphinxincludegraphics[scale=0.9]{{Hutch-1_readings_1}.png}\hspace*{\fill}}
&
\noindent{\hspace*{\fill}\sphinxincludegraphics[scale=0.9]{{Hutch-1_readings_2}.png}\hspace*{\fill}}
&
\noindent{\hspace*{\fill}\sphinxincludegraphics[scale=0.9]{{Hutch-1_readings_3}.png}\hspace*{\fill}}
&
\noindent{\hspace*{\fill}\sphinxincludegraphics[scale=0.9]{{Hutch-1_readings_4}.png}\hspace*{\fill}}
\\
\hline
\sphinxAtStartPar
Radiation in Hutch\sphinxhyphen{}1 near the penetration points. Spikes when downstream pop screens are in and a shutter is open.
&
\sphinxAtStartPar
Radiation in Hutch\sphinxhyphen{}1 behind the detector chamber. Spikes when POP10 is in and collimated shutter open.
&
\sphinxAtStartPar
Radiation in Hutch\sphinxhyphen{}1 near the penetration points. Spikes when downstream pop screens are in and a shutter is open.
&
\sphinxAtStartPar
Radiation in Hutch\sphinxhyphen{}1 inside the experiment chamber. Spikes when downstream pop screens are in and collimated shutter is open.
\\
\hline
\end{tabulary}
\par
\sphinxattableend\end{savenotes}

\sphinxstepscope


\section{Vault\sphinxhyphen{}1 Ionizing Radiation Interlock Testing Protocol}
\label{\detokenize{testing_documentation/Vault-1_ionizing_radiation:vault-1-ionizing-radiation-interlock-testing-protocol}}\label{\detokenize{testing_documentation/Vault-1_ionizing_radiation::doc}}
\sphinxAtStartPar
The objective of this testing procedure is to verify the functionality of the Vault\sphinxhyphen{}1 ionizing radiation interlock system.
The following falls under the scope of this testing procedure:
\begin{itemize}
\item {} 
\sphinxAtStartPar
Vault\sphinxhyphen{}1 search and securing system.

\item {} 
\sphinxAtStartPar
RF and accelerator arming.

\item {} 
\sphinxAtStartPar
Transmitter override.

\item {} 
\sphinxAtStartPar
Ionizing radiation monitoring interlocks.

\end{itemize}


\subsection{Starting Conditions}
\label{\detokenize{testing_documentation/Vault-1_ionizing_radiation:starting-conditions}}
\sphinxAtStartPar
The default state for testing the Vault\sphinxhyphen{}1 ionizing radiation interlock system is in a completely disarmed state.
\begin{enumerate}
\sphinxsetlistlabels{\arabic}{enumi}{enumii}{}{.}%
\item {} 
\sphinxAtStartPar
The IONIZING RADIATION INTERLOCK protocase in Vault\sphinxhyphen{}1 Control shows:
\begin{itemize}
\item {} 
\sphinxAtStartPar
All SECURE PERIMETER lamps are \DUrole{red}{red}.

\item {} 
\sphinxAtStartPar
All AREA MONITOR lamps are \DUrole{green}{green}.

\item {} 
\sphinxAtStartPar
TRANSMITTERS and ACCELERATOR lamps are \DUrole{red}{red}.

\end{itemize}

\item {} 
\sphinxAtStartPar
VIEWMARQ in Vault\sphinxhyphen{}1 Control displays \DUrole{green}{LASER SAFE}.

\item {} 
\sphinxAtStartPar
VIEWMARQ in Accelerator Lab displays \DUrole{green}{RF SAFE}.

\item {} 
\sphinxAtStartPar
All individual \DUrole{red}{red}, \DUrole{blue}{blue}, and \DUrole{orange}{orange} beacons are off.

\item {} 
\sphinxAtStartPar
Search buttons LEDs are off in Vault\sphinxhyphen{}1.
\begin{itemize}
\item {} 
\sphinxAtStartPar
Button 1

\item {} 
\sphinxAtStartPar
Button 2

\item {} 
\sphinxAtStartPar
Button 3

\end{itemize}

\end{enumerate}

\begin{figure}[htbp]
\centering
\capstart

\noindent\sphinxincludegraphics[scale=0.2]{{Vault-1_protocase}.jpg}
\caption{\sphinxstylestrong{Figure 1:} This is the Vault\sphinxhyphen{}1 Control IONIZING RADIATION INTERLOCK protocase in a disarmed, unsecured, safe state.}\label{\detokenize{testing_documentation/Vault-1_ionizing_radiation:id1}}\end{figure}


\begin{savenotes}\sphinxattablestart
\centering
\begin{tabulary}{\linewidth}[t]{|T|T|}
\hline

\noindent{\hspace*{\fill}\sphinxincludegraphics[scale=0.28]{{Vault-1_Control_VIEWMARQ_safe}.jpg}\hspace*{\fill}}
&
\noindent{\hspace*{\fill}\sphinxincludegraphics[scale=0.2]{{Accelerator_lab_VIEWMARQ_safe}.jpg}\hspace*{\fill}}
\\
\hline
\sphinxAtStartPar
Vault\sphinxhyphen{}1 Control VIEWMARQ display when the system is safe. \DUrole{white-cell}{=================================}
&
\sphinxAtStartPar
Accelerator Lab VIEWMARQ display when the system is safe. \DUrole{white-cell}{=================================}
\\
\hline
\end{tabulary}
\par
\sphinxattableend\end{savenotes}

\begin{sphinxuseclass}{tight-table-caption-container}
\sphinxAtStartPar
\sphinxstylestrong{Figure 2:} These are the VIEWMARQ displays in Vault\sphinxhyphen{}1 Control and Accelerator Lab when the system is safe.

\end{sphinxuseclass}

\begin{savenotes}\sphinxattablestart
\centering
\begin{tabulary}{\linewidth}[t]{|T|T|}
\hline

\noindent{\hspace*{\fill}\sphinxincludegraphics[scale=0.2]{{Vault-1_search_off}.jpg}\hspace*{\fill}}
&
\noindent{\hspace*{\fill}\sphinxincludegraphics[scale=0.2]{{Vault-1_search_on}.jpg}\hspace*{\fill}}
\\
\hline
\sphinxAtStartPar
Vault\sphinxhyphen{}1 search button off. \DUrole{white-cell}{============================================================}
&
\sphinxAtStartPar
Vault\sphinxhyphen{}1 search button on. \DUrole{white-cell}{=============================================================}
\\
\hline
\end{tabulary}
\par
\sphinxattableend\end{savenotes}

\begin{sphinxuseclass}{tight-table-caption-container}
\sphinxAtStartPar
\sphinxstylestrong{Figure 3:} This is an example of the Vault\sphinxhyphen{}1 search buttons in both states.

\end{sphinxuseclass}

\subsection{Testing Unsecure Vault\sphinxhyphen{}1 Conditions}
\label{\detokenize{testing_documentation/Vault-1_ionizing_radiation:testing-unsecure-vault-1-conditions}}
\sphinxAtStartPar
When Vault\sphinxhyphen{}1 is in a non\sphinxhyphen{}secure state, neither the accelerator nor the transmitters should be able to be armed.
\begin{enumerate}
\sphinxsetlistlabels{\arabic}{enumi}{enumii}{}{.}%
\item {} 
\sphinxAtStartPar
Switch the ENABLE key for the accelerator.
The accelerator lamp remains \DUrole{red}{red}.

\item {} 
\sphinxAtStartPar
Switch the ENABLE keys for transmitters 1 and 2.
Both transmitter lamps remain \DUrole{red}{red}.

\end{enumerate}


\subsection{Searching Procedure}
\label{\detokenize{testing_documentation/Vault-1_ionizing_radiation:searching-procedure}}\begin{enumerate}
\sphinxsetlistlabels{\arabic}{enumi}{enumii}{}{.}%
\item {} 
\sphinxAtStartPar
Push search buttons three and two and verify they will not activate without going in the correct sequence.
\begin{itemize}
\item {} 
\sphinxAtStartPar
Button 3

\item {} 
\sphinxAtStartPar
Button 2

\end{itemize}

\item {} 
\sphinxAtStartPar
Go through the vault and in sequence, click the three search buttons.
The LED on the search button should glow and the corresponding lamp on the Vault\sphinxhyphen{}1 Control IONIZING RADIATION INTERLOCK protocase should turn \DUrole{green}{green}.
\begin{itemize}
\item {} 
\sphinxAtStartPar
Button 1

\item {} 
\sphinxAtStartPar
Button 2

\item {} 
\sphinxAtStartPar
Button 3

\end{itemize}

\item {} 
\sphinxAtStartPar
When the third search button is hit, a chime will start, and the LED on the chime will flash.

\item {} 
\sphinxAtStartPar
Once all three search buttons have been hit in order, close the shield door.
The shield door lamp on the Vault\sphinxhyphen{}1 Control IONIZING RADIATION INTERLOCK protocase will turn green once the mechanical door switches are fully actuated.

\end{enumerate}

\begin{figure}[htbp]
\centering
\capstart

\noindent\sphinxincludegraphics[scale=0.2]{{Vault-1_searched}.jpg}
\caption{\sphinxstylestrong{Figure 4:} Vault\sphinxhyphen{}1 IONIZING RADIATION INTERLOCK protocase when Vault\sphinxhyphen{}1 is searched.}\label{\detokenize{testing_documentation/Vault-1_ionizing_radiation:id2}}\end{figure}

\begin{figure}[htbp]
\centering
\capstart

\noindent\sphinxincludegraphics[scale=0.2]{{Vault-1_door}.jpg}
\caption{\sphinxstylestrong{Figure 5:} Vault\sphinxhyphen{}1 IONIZING RADIATION INTERLOCK protocase when the shield door is closed and Vault\sphinxhyphen{}1 is secured.
Under this state the accelerator can now be armed.}\label{\detokenize{testing_documentation/Vault-1_ionizing_radiation:id3}}\end{figure}


\subsection{Arming the Accelerator and transmitters}
\label{\detokenize{testing_documentation/Vault-1_ionizing_radiation:arming-the-accelerator-and-transmitters}}\begin{enumerate}
\sphinxsetlistlabels{\arabic}{enumi}{enumii}{}{.}%
\item {} 
\sphinxAtStartPar
With the shield door still closed, enable the accelerator on the Vault\sphinxhyphen{}1 Control IONIZING RADIATION INTERLOCK protocase using the ENABLE key.
The accelerator status lamp should turn \DUrole{green}{green}.

\item {} 
\sphinxAtStartPar
Turn the ENABLE key for transmitter 1 on the Vault\sphinxhyphen{}1 Control IONIZING RADIATION INTERLOCK protocase.
The transmitter 1 status lamp should turn \DUrole{green}{green}.

\item {} 
\sphinxAtStartPar
Once the transmitter is enabled, the VIEWMARQ displays will show \DUrole{red}{VAULT SECURE \sphinxhyphen{} RF ARMED}.
\begin{itemize}
\item {} 
\sphinxAtStartPar
Vault\sphinxhyphen{}1 Control

\item {} 
\sphinxAtStartPar
Accelerator Lab

\end{itemize}

\item {} 
\sphinxAtStartPar
The \DUrole{blue}{blue} beacons next to each VIEWMARQ are on.
\begin{itemize}
\item {} 
\sphinxAtStartPar
Vault\sphinxhyphen{}1 Control

\item {} 
\sphinxAtStartPar
Accelerator Lab

\end{itemize}

\item {} 
\sphinxAtStartPar
Hit the reset button and repeat 2\sphinxhyphen{}4 with transmitter 2.

\end{enumerate}

\begin{figure}[htbp]
\centering
\capstart

\noindent\sphinxincludegraphics[scale=0.2]{{Vault-1_protocase_accelerator_armed}.jpg}
\caption{\sphinxstylestrong{Figure 6:} Vault\sphinxhyphen{}1 Control IONIZING RADIATION INTERLOCK protocase when the accelerator is armed.}\label{\detokenize{testing_documentation/Vault-1_ionizing_radiation:id4}}\end{figure}


\begin{savenotes}\sphinxattablestart
\centering
\begin{tabulary}{\linewidth}[t]{|T|T|}
\hline

\noindent{\hspace*{\fill}\sphinxincludegraphics[scale=0.2]{{Vault-1_protocase_transmitter_armed_1}.jpg}\hspace*{\fill}}
&
\noindent{\hspace*{\fill}\sphinxincludegraphics[scale=0.2]{{Vault-1_protocase_transmitter_armed_2}.jpg}\hspace*{\fill}}
\\
\hline
\sphinxAtStartPar
Vault\sphinxhyphen{}1 Control IONIZING RADIATION INTERLOCK protocase when transmitter 1 is armed. \DUrole{white-cell}{============}
&
\sphinxAtStartPar
Vault\sphinxhyphen{}1 Control IONIZING RADIATION INTERLOCK protocase when transmitter 2 is armed. \DUrole{white-cell}{============}
\\
\hline
\end{tabulary}
\par
\sphinxattableend\end{savenotes}

\begin{sphinxuseclass}{tight-table-caption-container}
\sphinxAtStartPar
\sphinxstylestrong{Figure 7:} This is the Vault\sphinxhyphen{}1 Control IONIZING RADIATION INTERLOCK protocase when the transmitters are armed.

\end{sphinxuseclass}

\begin{savenotes}\sphinxattablestart
\centering
\begin{tabulary}{\linewidth}[t]{|T|T|}
\hline

\noindent{\hspace*{\fill}\sphinxincludegraphics[scale=0.28]{{Vault-1_Control_VIEWMARQ_armed}.jpg}\hspace*{\fill}}
&
\noindent{\hspace*{\fill}\sphinxincludegraphics[scale=0.2]{{Accelerator_lab_VIEWMARQ_armed}.jpg}\hspace*{\fill}}
\\
\hline
\sphinxAtStartPar
Vault\sphinxhyphen{}1 Control VIEWMARQ display when the RF is armed. \DUrole{white-cell}{=================================}
&
\sphinxAtStartPar
Accelerator Lab VIEWMARQ display when the RF is armed. \DUrole{white-cell}{=================================}
\\
\hline
\end{tabulary}
\par
\sphinxattableend\end{savenotes}

\begin{sphinxuseclass}{tight-table-caption-container}
\sphinxAtStartPar
\sphinxstylestrong{Figure 8:} These are the VIEWMARQ displays in Vault\sphinxhyphen{}1 Control and Accelerator Lab when the RF is armed.

\end{sphinxuseclass}

\subsection{Overriding and Resetting Transmitters and Accelerator}
\label{\detokenize{testing_documentation/Vault-1_ionizing_radiation:overriding-and-resetting-transmitters-and-accelerator}}\begin{enumerate}
\sphinxsetlistlabels{\arabic}{enumi}{enumii}{}{.}%
\item {} 
\sphinxAtStartPar
With the accelerator and transmitters armed, switch the OVERRIDE keys on the Vault\sphinxhyphen{}1 Control IONIZING RADIATION INTERLOCK protocase.
The TRANSMITTERS lamps will turn \DUrole{orange}{orange}.

\item {} 
\sphinxAtStartPar
Switch the OVERRIDE keys back to interlock.
The TRANSMITTERS lamps will turn \DUrole{green}{green}.

\item {} 
\sphinxAtStartPar
Hit the reset button on the Vault\sphinxhyphen{}1 Control IONIZING RADIATION INTERLOCK protocase.
The ACCELERATOR and TRANSMITTERS lamps will turn \DUrole{red}{red}.

\end{enumerate}


\begin{savenotes}\sphinxattablestart
\centering
\begin{tabulary}{\linewidth}[t]{|T|T|}
\hline

\noindent{\hspace*{\fill}\sphinxincludegraphics[scale=0.2]{{Vault-1_protocase_transmitter_override_2}.jpg}\hspace*{\fill}}
&
\noindent{\hspace*{\fill}\sphinxincludegraphics[scale=0.2]{{Vault-1_protocase_transmitter_override_both}.jpg}\hspace*{\fill}}
\\
\hline
\sphinxAtStartPar
Vault\sphinxhyphen{}1 Control IONIZING RADIATION INTERLOCK protocase when transmitter 2 is in override. \DUrole{white-cell}{=======}
&
\sphinxAtStartPar
Vault\sphinxhyphen{}1 Control IONIZING RADIATION INTERLOCK protocase when both transmitters are in override. \DUrole{white-cell}{==}
\\
\hline
\end{tabulary}
\par
\sphinxattableend\end{savenotes}

\begin{sphinxuseclass}{tight-table-caption-container}
\sphinxAtStartPar
\sphinxstylestrong{Figure 9:} This is the Vault\sphinxhyphen{}1 Control IONIZING RADIATION INTERLOCK protocase when the transmitters are in override.

\end{sphinxuseclass}

\subsection{Timing out the Shield Door and Search sequence}
\label{\detokenize{testing_documentation/Vault-1_ionizing_radiation:timing-out-the-shield-door-and-search-sequence}}\begin{enumerate}
\sphinxsetlistlabels{\arabic}{enumi}{enumii}{}{.}%
\item {} 
\sphinxAtStartPar
Rearm the accelerator and transmitters and open the shield door.
The accelerator and transmitter lamps should turn \DUrole{red}{red}.

\item {} 
\sphinxAtStartPar
Push only the first search button.
After \DUrole{red}{x} seconds, the button LED should turn off.

\item {} 
\sphinxAtStartPar
Go through the search procedure again except do not close the shield door and allow the system to trip.
After \DUrole{red}{x}, the search lamps should turn \DUrole{red}{red}.

\end{enumerate}


\subsection{Return to Starting Conditions}
\label{\detokenize{testing_documentation/Vault-1_ionizing_radiation:return-to-starting-conditions}}\begin{enumerate}
\sphinxsetlistlabels{\arabic}{enumi}{enumii}{}{.}%
\item {} 
\sphinxAtStartPar
Return Vault\sphinxhyphen{}1 ionizing radiation interlock system to the default state.

\end{enumerate}

\sphinxstepscope


\section{Hutch\sphinxhyphen{}1 Ionizing Radiation Interlock System Testing Protocol}
\label{\detokenize{testing_documentation/Hutch-1_ionizing_radiation:hutch-1-ionizing-radiation-interlock-system-testing-protocol}}\label{\detokenize{testing_documentation/Hutch-1_ionizing_radiation::doc}}
\sphinxAtStartPar
The objective of this testing procedure is to verify the functionality of the Hutch\sphinxhyphen{}1 ionizing radiation interlock system.
The following fall under the scope of this testing procedure:
\begin{itemize}
\item {} 
\sphinxAtStartPar
Hutch\sphinxhyphen{}1 search and securing system.

\item {} 
\sphinxAtStartPar
Trouble tungsten shutter control.

\item {} 
\sphinxAtStartPar
Ionizing radiation monitoring interlocks.

\end{itemize}


\subsection{Starting Conditions}
\label{\detokenize{testing_documentation/Hutch-1_ionizing_radiation:starting-conditions}}
\sphinxAtStartPar
The default state for testing of the Hutch\sphinxhyphen{}1 ionizing radiation interlock system is in a completely disarmed state.
\begin{enumerate}
\sphinxsetlistlabels{\arabic}{enumi}{enumii}{}{.}%
\item {} 
\sphinxAtStartPar
Search button LEDs are off in Hutch\sphinxhyphen{}1.
\begin{itemize}
\item {} 
\sphinxAtStartPar
Button 1

\item {} 
\sphinxAtStartPar
Button 2

\item {} 
\sphinxAtStartPar
Button 3

\end{itemize}

\item {} 
\sphinxAtStartPar
The following Hutch\sphinxhyphen{}1 Control Ionizing Radiation Interlock protocase lamps are \DUrole{red}{red}.
\begin{itemize}
\item {} 
\sphinxAtStartPar
SECURE PERIMETER

\item {} 
\sphinxAtStartPar
BEAM SELECT

\item {} 
\sphinxAtStartPar
BEAM STOP

\end{itemize}

\item {} 
\sphinxAtStartPar
THe following Hutch\sphinxhyphen{}1 Control Ionizing Radiation Interlock protocase lamps are \DUrole{green}{green}.
\begin{itemize}
\item {} 
\sphinxAtStartPar
AREA MONITOR

\item {} 
\sphinxAtStartPar
BEAM status

\end{itemize}

\item {} 
\sphinxAtStartPar
Hutch\sphinxhyphen{}1 Control Ionizing Radiation Interlock protocase BEAM SELECT key is set to DIVERGENT.

\item {} 
\sphinxAtStartPar
Hutch\sphinxhyphen{}1 Control VIEWMARQ displays \DUrole{green}{LASER SAFE}.

\item {} 
\sphinxAtStartPar
There is no 24VDC across the \DUrole{blue}{blue} and \DUrole{yellow}{yellow} contact blocks in Hutch\sphinxhyphen{}1 panel for the double tungsten shutters.
\begin{itemize}
\item {} 
\sphinxAtStartPar
Set 1

\item {} 
\sphinxAtStartPar
Set 2

\end{itemize}

\item {} 
\sphinxAtStartPar
The double tungsten shutters are in the closed position.

\end{enumerate}

\begin{figure}[htbp]
\centering
\capstart

\noindent\sphinxincludegraphics[scale=0.2]{{Hutch-1_protocase}.jpg}
\caption{\sphinxstylestrong{Figure 1:} Hutch\sphinxhyphen{}1 Control Ionizing Radiation Interlock protocase.}\label{\detokenize{testing_documentation/Hutch-1_ionizing_radiation:id1}}\end{figure}


\begin{savenotes}\sphinxattablestart
\centering
\begin{tabulary}{\linewidth}[t]{|T|T|}
\hline

\noindent{\hspace*{\fill}\sphinxincludegraphics[scale=0.2]{{Vault-1_search_off}.jpg}\hspace*{\fill}}
&
\noindent{\hspace*{\fill}\sphinxincludegraphics[scale=0.2]{{Vault-1_search_on}.jpg}\hspace*{\fill}}
\\
\hline
\sphinxAtStartPar
Hutch\sphinxhyphen{}1 search button off. \DUrole{white-cell}{============================================================}
&
\sphinxAtStartPar
Hutch\sphinxhyphen{}1 search button on. \DUrole{white-cell}{=============================================================}
\\
\hline
\end{tabulary}
\par
\sphinxattableend\end{savenotes}

\begin{sphinxuseclass}{tight-table-caption-container}
\sphinxAtStartPar
\sphinxstylestrong{Figure 2:} This is an example of the Hutch\sphinxhyphen{}1 search buttons in both states.

\end{sphinxuseclass}

\begin{savenotes}\sphinxattablestart
\centering
\begin{tabulary}{\linewidth}[t]{|T|T|}
\hline

\noindent{\hspace*{\fill}\sphinxincludegraphics[scale=0.92]{{shutter_contacts_1}.jpg}\hspace*{\fill}}
&
\noindent{\hspace*{\fill}\sphinxincludegraphics[scale=0.92]{{shutter_contacts_2}.jpg}\hspace*{\fill}}
\\
\hline
\sphinxAtStartPar
Hutch\sphinxhyphen{}1 double tungsten shutter contact set 1. \DUrole{white-cell}{============================================}
&
\sphinxAtStartPar
Hutch\sphinxhyphen{}1 double tungsten shutter contact set 2. \DUrole{white-cell}{============================================}
\\
\hline
\end{tabulary}
\par
\sphinxattableend\end{savenotes}

\begin{sphinxuseclass}{tight-table-caption-container}
\sphinxAtStartPar
\sphinxstylestrong{Figure 3:} These are the contract blocks for the double tungsten shutters in the Hutch\sphinxhyphen{}1 aggregator panel. The specific sets being referenced here are the blue and yellow contact blocks next to the red IDEM relays. If there is 24VDC across these contact blocks, the shutters are open.

\end{sphinxuseclass}

\begin{savenotes}\sphinxattablestart
\centering
\begin{tabulary}{\linewidth}[t]{|T|T|}
\hline

\noindent{\hspace*{\fill}\sphinxincludegraphics[scale=0.2]{{double_tungsten_shutter}.jpg}\hspace*{\fill}}
&
\noindent{\hspace*{\fill}\sphinxincludegraphics[scale=0.2]{{double_tungsten_shutter_zoom}.jpg}\hspace*{\fill}}
\\
\hline
\sphinxAtStartPar
This is the upright double tungsten shutter. \DUrole{white-cell}{===============================================}
&
\sphinxAtStartPar
This is a zoomed in image of the upright double tungsten shutter.
It can be seen here that the contact plate is on the bottom sensors. \DUrole{white-cell}{====}
\\
\hline
\end{tabulary}
\par
\sphinxattableend\end{savenotes}

\begin{sphinxuseclass}{tight-table-caption-container}
\sphinxAtStartPar
\sphinxstylestrong{Figure 4:} This is the upright double tungsten shutter, and an example of this shutter being in the closed position. If the plate where to be on the top sensors, the shutters would be open. Additionally, the other double tungsten shutter is rotated 180 degrees from this one, which can be seen in the first image as the closer shutter only shows its top. Because it is rotated, the contact on the upper sensors is the closed position for the other shutter.

\end{sphinxuseclass}

\subsection{Search Procedure}
\label{\detokenize{testing_documentation/Hutch-1_ionizing_radiation:search-procedure}}\begin{enumerate}
\sphinxsetlistlabels{\arabic}{enumi}{enumii}{}{.}%
\item {} 
\sphinxAtStartPar
Push search buttons three and two in Hutch\sphinxhyphen{}1 and verify they will not activate without going in the correct sequence.
\begin{itemize}
\item {} 
\sphinxAtStartPar
Button 3

\item {} 
\sphinxAtStartPar
Button 2

\end{itemize}

\item {} 
\sphinxAtStartPar
Go through Hutch\sphinxhyphen{}1 and in the correct sequence, click the three search buttons. The LED on the search button should turn on and the corresponding lamp on the Hutch\sphinxhyphen{}1 Ionizing Radiation interlock protocase should turn \DUrole{green}{green}.
\begin{itemize}
\item {} 
\sphinxAtStartPar
Button 1

\item {} 
\sphinxAtStartPar
Button 2

\item {} 
\sphinxAtStartPar
Button 3

\end{itemize}

\item {} 
\sphinxAtStartPar
When the third search button is hit, there will be an audible chime and a flashing light.

\item {} 
\sphinxAtStartPar
Once all three search buttons have been hit in order, close the shield door. The Hutch\sphinxhyphen{}1 Control Ionizing Radiation Interlock protocase shield door lamp turns \DUrole{green}{green}.

\item {} 
\sphinxAtStartPar
The Hutch\sphinxhyphen{}1 Control Ionizing Radiation Interlock protocase Beam Select Divergent lamp turns \DUrole{green}{green}.

\end{enumerate}

\begin{figure}[htbp]
\centering
\capstart

\noindent\sphinxincludegraphics[scale=0.2]{{Hutch-1_searched}.jpg}
\caption{\sphinxstylestrong{Figure 5:} This is the Hutch\sphinxhyphen{}1 Control IONIZING RADIATION INTERLOCK protocase when Hutch\sphinxhyphen{}1 is searched.}\label{\detokenize{testing_documentation/Hutch-1_ionizing_radiation:id2}}\end{figure}

\begin{figure}[htbp]
\centering
\capstart

\noindent\sphinxincludegraphics[scale=0.2]{{Hutch-1_door}.jpg}
\caption{\sphinxstylestrong{Figure 6:} This is the Hutch\sphinxhyphen{}1 Control IONIZING RADIATION INTERLOCK protocase when Hutch\sphinxhyphen{}1 is secured.}\label{\detokenize{testing_documentation/Hutch-1_ionizing_radiation:id3}}\end{figure}


\subsection{Changing Beam status}
\label{\detokenize{testing_documentation/Hutch-1_ionizing_radiation:changing-beam-status}}\begin{enumerate}
\sphinxsetlistlabels{\arabic}{enumi}{enumii}{}{.}%
\item {} 
\sphinxAtStartPar
Turn the Hutch\sphinxhyphen{}1 Control IONIZING RADIATION INTERLOCK protocase BEAM STOP key to Open.
The Beam Stop lamp turn \DUrole{green}{green}.

\item {} 
\sphinxAtStartPar
The Hutch\sphinxhyphen{}1 Control IONIZING RADIATION INTERLOCK protocase BEAM STATUS DIVERGENT lamp turn \DUrole{green}{green}.

\item {} 
\sphinxAtStartPar
Change the Hutch\sphinxhyphen{}1 Ionizing Radiation Interlock protocase Beam Select key to Collimated.
\begin{itemize}
\item {} 
\sphinxAtStartPar
The Beam Select Divergent lamp turn \DUrole{red}{red}.

\item {} 
\sphinxAtStartPar
The Beam Select Collimated lamp turn \DUrole{green}{green}.

\item {} 
\sphinxAtStartPar
The Status Divergent lamp turns \DUrole{red}{red}.

\item {} 
\sphinxAtStartPar
The Beam Status Collimated lamp turn \DUrole{orange}{orange} while the shutter is moving, and then turns \DUrole{green}{green} when the shutter is closed.

\item {} 
\sphinxAtStartPar
Verify that the shutters are physically closed.

\end{itemize}

\item {} 
\sphinxAtStartPar
Change the Beam Select key back to Divergent. The inverse of step three should occur.
\begin{itemize}
\item {} 
\sphinxAtStartPar
The Beam Select Collimated lamp turns \DUrole{red}{red}.

\item {} 
\sphinxAtStartPar
The Beam Select Divergent lamp turns \DUrole{green}{green}.

\item {} 
\sphinxAtStartPar
The Beam Status Collimated lamp turns \DUrole{red}{red}.

\item {} 
\sphinxAtStartPar
The Beam Status Divergent lamp turns \DUrole{orange}{orange} while the shutter is moving, and then turns \DUrole{green}{green} when the shutter is closed.

\item {} 
\sphinxAtStartPar
Verify that the shutters are physically closed.

\end{itemize}

\item {} 
\sphinxAtStartPar
Press the Reset button.
All Hutch\sphinxhyphen{}1 Control Ionizing Radiation Interlock protocase Beam Status lamps are \DUrole{green}{green}.
Verify the shutters are physically in the correct position.

\end{enumerate}


\begin{savenotes}\sphinxattablestart
\centering
\begin{tabulary}{\linewidth}[t]{|T|T|}
\hline

\noindent{\hspace*{\fill}\sphinxincludegraphics[scale=0.2]{{Hutch-1_Divergent_open}.jpg}\hspace*{\fill}}
&
\noindent{\hspace*{\fill}\sphinxincludegraphics[scale=0.2]{{Hutch-1_Collimated_open}.jpg}\hspace*{\fill}}
\\
\hline
\sphinxAtStartPar
Divergent beam open. \DUrole{white-cell}{===============================================================}
&
\sphinxAtStartPar
Collimated beam open. \DUrole{white-cell}{==============================================================}
\\
\hline
\end{tabulary}
\par
\sphinxattableend\end{savenotes}

\begin{sphinxuseclass}{tight-table-caption-container}
\sphinxAtStartPar
\sphinxstylestrong{Figure 7:} This is the Hutch\sphinxhyphen{}1 Control IONIZING RADIATION INTERLOCK protocase when either shutter is open. When the beam stop is open a shutter will automatically open to whatever beam select is set to before hand.

\end{sphinxuseclass}

\subsection{Returning to Starting Conditions}
\label{\detokenize{testing_documentation/Hutch-1_ionizing_radiation:returning-to-starting-conditions}}\begin{enumerate}
\sphinxsetlistlabels{\arabic}{enumi}{enumii}{}{.}%
\item {} 
\sphinxAtStartPar
Return the Hutch\sphinxhyphen{}1 ionizing radiation interlock system back to starting conditions.

\end{enumerate}

\sphinxstepscope


\section{Ionizing Radiation Emergency Stop Testing Protocol}
\label{\detokenize{testing_documentation/e-stop_testing:ionizing-radiation-emergency-stop-testing-protocol}}\label{\detokenize{testing_documentation/e-stop_testing::doc}}
\sphinxAtStartPar
The objective of this testing procedure is to verify the functionality of the ionizing radiation emergency stop button system.
This system is used to cut power to the transmitters from a high\sphinxhyphen{}power state during an emergency.


\subsection{Starting Conditions}
\label{\detokenize{testing_documentation/e-stop_testing:starting-conditions}}\begin{enumerate}
\sphinxsetlistlabels{\arabic}{enumi}{enumii}{}{.}%
\item {} 
\sphinxAtStartPar
VIEWMARQ display in Accelerator Lab shows \DUrole{green}{RF SAFE}.

\item {} 
\sphinxAtStartPar
VIEWMARQ display in Vault\sphinxhyphen{}1 Control shows \DUrole{green}{LASER SAFE}.

\item {} 
\sphinxAtStartPar
Check that relay 1\sphinxhyphen{}4, and 7 in Vault\sphinxhyphen{}1 Control west aggregator panel shows all diagnostic LEDs on.
\begin{itemize}
\item {} 
\sphinxAtStartPar
Relay 1

\item {} 
\sphinxAtStartPar
Relay 2

\item {} 
\sphinxAtStartPar
Relay 3

\item {} 
\sphinxAtStartPar
Relay 4

\item {} 
\sphinxAtStartPar
Relay 7

\end{itemize}

\item {} 
\sphinxAtStartPar
Check that the relays in RF\sphinxhyphen{}1 aggregator panel show all diagnostic LEDs on.

\item {} 
\sphinxAtStartPar
Verify that the Accelerator Lab and Vault\sphinxhyphen{}1 Control E\sphinxhyphen{}Stop beacons are off.
\begin{itemize}
\item {} 
\sphinxAtStartPar
Vault\sphinxhyphen{}1 Control Ionizing Radiation Interlock protocase beacon.

\item {} 
\sphinxAtStartPar
Vault\sphinxhyphen{}1 Control \DUrole{red}{red} beacon module.

\item {} 
\sphinxAtStartPar
Accelerator Lab \DUrole{red}{red} beacon module.

\item {} 
\sphinxAtStartPar
Hutch\sphinxhyphen{}1 Control Ionizing Radiation Interlock protocase beacon.

\end{itemize}

\item {} 
\sphinxAtStartPar
Verify that all CXLS ionizing radiation emergency stop buttons are not engaged.
\begin{itemize}
\item {} 
\sphinxAtStartPar
Hutch\sphinxhyphen{}1 A

\item {} 
\sphinxAtStartPar
Hutch\sphinxhyphen{}1 B

\item {} 
\sphinxAtStartPar
Hutch\sphinxhyphen{}1 C

\item {} 
\sphinxAtStartPar
Hutch\sphinxhyphen{}1 Control A

\item {} 
\sphinxAtStartPar
Hutch\sphinxhyphen{}1 Control B

\item {} 
\sphinxAtStartPar
RF\sphinxhyphen{}1 A

\item {} 
\sphinxAtStartPar
RF\sphinxhyphen{}1 B

\item {} 
\sphinxAtStartPar
Vault\sphinxhyphen{}1 Control A

\item {} 
\sphinxAtStartPar
Vault\sphinxhyphen{}1 A

\item {} 
\sphinxAtStartPar
Vault\sphinxhyphen{}1 B

\item {} 
\sphinxAtStartPar
Vault\sphinxhyphen{}1 C

\item {} 
\sphinxAtStartPar
Vault\sphinxhyphen{}1 D

\item {} 
\sphinxAtStartPar
Vault\sphinxhyphen{}1 E

\item {} 
\sphinxAtStartPar
Vault\sphinxhyphen{}1 F

\item {} 
\sphinxAtStartPar
Vault\sphinxhyphen{}1 G

\end{itemize}

\end{enumerate}


\begin{savenotes}\sphinxattablestart
\centering
\begin{tabulary}{\linewidth}[t]{|T|T|}
\hline

\noindent{\hspace*{\fill}\sphinxincludegraphics[scale=0.2]{{e-stop_on1}.jpg}\hspace*{\fill}}
&
\noindent{\hspace*{\fill}\sphinxincludegraphics[scale=0.2]{{e-stop_off1}.jpg}\hspace*{\fill}}
\\
\hline
\sphinxAtStartPar
E\sphinxhyphen{}stop button engaged. \DUrole{white-cell}{===============================================================}
&
\sphinxAtStartPar
E\sphinxhyphen{}stop button disengaged. \DUrole{white-cell}{============================================================}
\\
\hline
\end{tabulary}
\par
\sphinxattableend\end{savenotes}

\begin{sphinxuseclass}{tight-table-caption-container}
\sphinxAtStartPar
\sphinxstylestrong{Figure 1:} These are examples of the ionizing radiation emergency stop buttons in the facility.

\end{sphinxuseclass}

\begin{savenotes}\sphinxattablestart
\centering
\begin{tabulary}{\linewidth}[t]{|T|T|}
\hline

\noindent{\hspace*{\fill}\sphinxincludegraphics[scale=0.28]{{Vault-1_Control_VIEWMARQ_safe}.jpg}\hspace*{\fill}}
&
\noindent{\hspace*{\fill}\sphinxincludegraphics[scale=0.2]{{Accelerator_lab_VIEWMARQ_safe}.jpg}\hspace*{\fill}}
\\
\hline
\sphinxAtStartPar
Vault\sphinxhyphen{}1 Control VIEWMARQ display when the system is safe. \DUrole{white-cell}{=================================}
&
\sphinxAtStartPar
Accelerator Lab VIEWMARQ display when the system is safe. \DUrole{white-cell}{=================================}
\\
\hline
\end{tabulary}
\par
\sphinxattableend\end{savenotes}

\begin{sphinxuseclass}{tight-table-caption-container}
\sphinxAtStartPar
\sphinxstylestrong{Figure 2:} These are the VIEWMARQ displays in Vault\sphinxhyphen{}1 Control and Accelerator Lab when the system is safe.

\end{sphinxuseclass}

\begin{savenotes}\sphinxattablestart
\centering
\begin{tabulary}{\linewidth}[t]{|T|T|}
\hline

\noindent{\hspace*{\fill}\sphinxincludegraphics[scale=0.2]{{relay_on}.jpg}\hspace*{\fill}}
&
\noindent{\hspace*{\fill}\sphinxincludegraphics[scale=0.2]{{relay_off}.jpg}\hspace*{\fill}}
\\
\hline
\sphinxAtStartPar
Relay on example. \DUrole{white-cell}{=================================================================}
&
\sphinxAtStartPar
Relay off example. \DUrole{white-cell}{================================================================}
\\
\hline
\end{tabulary}
\par
\sphinxattableend\end{savenotes}

\begin{sphinxuseclass}{tight-table-caption-container}
\sphinxAtStartPar
\sphinxstylestrong{Figure 3:} These are examples of the relays, in on and off states. If there are only some diagnostic LEDs on, the relay is in a fault state, and must be troubleshooted.

\end{sphinxuseclass}

\subsection{Testing}
\label{\detokenize{testing_documentation/e-stop_testing:testing}}\begin{enumerate}
\sphinxsetlistlabels{\arabic}{enumi}{enumii}{}{.}%
\item {} 
\sphinxAtStartPar
Push the Vault\sphinxhyphen{}1 Control E\sphinxhyphen{}stop. In response:
\begin{itemize}
\item {} 
\sphinxAtStartPar
E\sphinxhyphen{}stop LED turns on.

\item {} 
\sphinxAtStartPar
Ionizing Radiation Interlock protocase beacon in Hutch\sphinxhyphen{}1 Control and Vault\sphinxhyphen{}1 Control turns on.

\item {} 
\sphinxAtStartPar
VIEWMARQ display in Accelerator Lab and Vault\sphinxhyphen{}1 Control shows \DUrole{red}{IONIZING RADIATION E\sphinxhyphen{}STOP ACTIVATED}.

\item {} 
\sphinxAtStartPar
Individual \DUrole{red}{red} beacon modules in Accelerator Lab and Vault\sphinxhyphen{}1 Control turn on.

\item {} 
\sphinxAtStartPar
Black and white contacts blocks lose 24VDC signal.

\item {} 
\sphinxAtStartPar
Relay 15 diagnostic LEDs are off in the Hutch\sphinxhyphen{}1 panel.

\end{itemize}

\item {} 
\sphinxAtStartPar
For all other E\sphinxhyphen{}stops, only verify that hte E\sphinxhyphen{}stop light turns on and that the black and white contact block loses 24VDC signal.
\begin{itemize}
\item {} 
\sphinxAtStartPar
Hutch\sphinxhyphen{}1 A

\item {} 
\sphinxAtStartPar
Hutch\sphinxhyphen{}1 B

\item {} 
\sphinxAtStartPar
Hutch\sphinxhyphen{}1 C

\item {} 
\sphinxAtStartPar
Hutch\sphinxhyphen{}1 Control A

\item {} 
\sphinxAtStartPar
Hutch\sphinxhyphen{}1 Control B

\item {} 
\sphinxAtStartPar
RF\sphinxhyphen{}1 A

\item {} 
\sphinxAtStartPar
RF\sphinxhyphen{}1 B

\item {} 
\sphinxAtStartPar
Vault\sphinxhyphen{}1 Control A

\item {} 
\sphinxAtStartPar
Vault\sphinxhyphen{}1 A

\item {} 
\sphinxAtStartPar
Vault\sphinxhyphen{}1 B

\item {} 
\sphinxAtStartPar
Vault\sphinxhyphen{}1 C

\item {} 
\sphinxAtStartPar
Vault\sphinxhyphen{}1 D

\item {} 
\sphinxAtStartPar
Vault\sphinxhyphen{}1 E

\item {} 
\sphinxAtStartPar
Vault\sphinxhyphen{}1 F

\item {} 
\sphinxAtStartPar
Vault\sphinxhyphen{}1 G

\end{itemize}

\end{enumerate}


\begin{savenotes}\sphinxattablestart
\centering
\begin{tabulary}{\linewidth}[t]{|T|T|}
\hline

\noindent{\hspace*{\fill}\sphinxincludegraphics[scale=0.28]{{Vault-1_Control_VIEWMARQ_e-stop}.jpg}\hspace*{\fill}}
&
\noindent{\hspace*{\fill}\sphinxincludegraphics[scale=0.2]{{Accelerator_lab_VIEWMARQ_e-stop}.jpg}\hspace*{\fill}}
\\
\hline
\sphinxAtStartPar
Vault\sphinxhyphen{}1 Control VIEWMARQ display when the system is in an e\sphinxhyphen{}stop state. \DUrole{white-cell}{=======================}
&
\sphinxAtStartPar
Accelerator Lab VIEWMARQ display when the system is in an e\sphinxhyphen{}stop state. \DUrole{white-cell}{=======================}
\\
\hline
\end{tabulary}
\par
\sphinxattableend\end{savenotes}

\begin{sphinxuseclass}{tight-table-caption-container}
\sphinxAtStartPar
\sphinxstylestrong{Figure 4:} These are the VIEWMARQ displays in Vault\sphinxhyphen{}1 Control and Accelerator Lab when an ionizing radiation e\sphinxhyphen{}stop is pressed.

\end{sphinxuseclass}

\subsection{Emergency Tungsten Shutter Crash}
\label{\detokenize{testing_documentation/e-stop_testing:emergency-tungsten-shutter-crash}}\begin{enumerate}
\sphinxsetlistlabels{\arabic}{enumi}{enumii}{}{.}%
\item {} 
\sphinxAtStartPar
Secure Hutch\sphinxhyphen{}1.

\item {} 
\sphinxAtStartPar
Set the BEAM SELECT to DIVERGENT.

\item {} 
\sphinxAtStartPar
Chose any ionizing radiation e\sphinxhyphen{}stop in the facility and press it. In response:
\begin{itemize}
\item {} 
\sphinxAtStartPar
Hutch\sphinxhyphen{}1 Control Ionizing Radiation Interlock protocase lamps for Beam Status turn \DUrole{red}{red}.

\item {} 
\sphinxAtStartPar
The DIVERGENT shutter (closest, upside down shutter) closed.

\end{itemize}

\end{enumerate}


\subsection{High Power Transmitter Crash}
\label{\detokenize{testing_documentation/e-stop_testing:high-power-transmitter-crash}}\begin{enumerate}
\sphinxsetlistlabels{\arabic}{enumi}{enumii}{}{.}%
\item {} 
\sphinxAtStartPar
Every 6 months, the ionizing radiation emergency stop buttons are tested for successfully crashing the transmitters from a high\sphinxhyphen{}power state. Verify the last date for the e\sphinxhyphen{}stop crash test.

\item {} 
\sphinxAtStartPar
If 6 months have passed, put both transmitters into TRIG and verify it loses power when an e\sphinxhyphen{}stop is pressed.

\end{enumerate}

\sphinxstepscope


\section{Apantec Ionizing Radiation Sensors Testing Protocol}
\label{\detokenize{testing_documentation/apantec_testing:apantec-ionizing-radiation-sensors-testing-protocol}}\label{\detokenize{testing_documentation/apantec_testing::doc}}
\sphinxAtStartPar
The objective of this testing procedure is to test the functionality of the Apantec ionizing radiation detection equipment and interlocks.


\subsection{Starting Conditions}
\label{\detokenize{testing_documentation/apantec_testing:starting-conditions}}\begin{enumerate}
\sphinxsetlistlabels{\arabic}{enumi}{enumii}{}{.}%
\item {} 
\sphinxAtStartPar
All rate meters are in normal operating condition.
\begin{itemize}
\item {} 
\sphinxAtStartPar
Only the \DUrole{green}{NORMAL} condition LED is on. \DUrole{red}{ALARM}, \DUrole{yellow}{ALERT}, and FAIL LEDs are off.

\item {} 
\sphinxAtStartPar
There is no audible alarm from the rate meter.

\item {} 
\sphinxAtStartPar
The displays on the rate meters do not display any error messages.

\item {} 
\sphinxAtStartPar
All probes are calibrated.
\begin{itemize}
\item {} 
\sphinxAtStartPar
Hutch\sphinxhyphen{}1 Front

\item {} 
\sphinxAtStartPar
Hutch\sphinxhyphen{}1 Black

\item {} 
\sphinxAtStartPar
Laser\sphinxhyphen{}1

\item {} 
\sphinxAtStartPar
RF\sphinxhyphen{}1

\item {} 
\sphinxAtStartPar
Vault\sphinxhyphen{}1 Control

\end{itemize}

\end{itemize}

\end{enumerate}

\begin{sphinxadmonition}{note}{Note:}
\sphinxAtStartPar
Check calibration sheet in Accelerator Lab.
\end{sphinxadmonition}

\begin{sphinxadmonition}{warning}{Warning:}
\sphinxAtStartPar
Make calibration sheet and find a place to post it.
\end{sphinxadmonition}
\begin{enumerate}
\sphinxsetlistlabels{\arabic}{enumi}{enumii}{}{.}%
\item {} 
\sphinxAtStartPar
The Ionizing Radiation Interlock protocase AREA MONITOR RADIATION lamps are \DUrole{green}{green}.
\begin{itemize}
\item {} 
\sphinxAtStartPar
Vault\sphinxhyphen{}1 Control

\item {} 
\sphinxAtStartPar
Hutch\sphinxhyphen{}1 Control

\end{itemize}

\item {} 
\sphinxAtStartPar
The \DUrole{yellow}{yellow} and white contact blocks in the Vault\sphinxhyphen{}1 Control west aggregator panel for the radiation chain should have 24VDC across them.

\item {} 
\sphinxAtStartPar
Hutch\sphinxhyphen{}1 is in a non\sphinxhyphen{}secure state.

\item {} 
\sphinxAtStartPar
Relays in the Vault\sphinxhyphen{}1 Control west aggregator panel shows all diagnostic LEDs on.
\begin{itemize}
\item {} 
\sphinxAtStartPar
R4

\item {} 
\sphinxAtStartPar
R6

\item {} 
\sphinxAtStartPar
R7

\item {} 
\sphinxAtStartPar
8

\end{itemize}

\item {} 
\sphinxAtStartPar
Relays in the Vault\sphinxhyphen{}1 Control west aggregator panel show no power.
\begin{itemize}
\item {} 
\sphinxAtStartPar
R3

\item {} 
\sphinxAtStartPar
R5

\end{itemize}

\end{enumerate}

\begin{figure}[htbp]
\centering
\capstart

\noindent\sphinxincludegraphics[scale=0.2]{{Apantec_normal}.jpg}
\caption{\sphinxstylestrong{Figure 1:} This is an example of the RMW1 rate meter control in a normal operating state.}\label{\detokenize{testing_documentation/apantec_testing:id1}}\end{figure}

\begin{figure}[htbp]
\centering
\capstart

\noindent\sphinxincludegraphics[scale=0.2]{{yellow_white_contacts}.jpg}
\caption{\sphinxstylestrong{Figure 2:} These are the yellow and white contact blocks in the Vault\sphinxhyphen{}1 Control west panel.}\label{\detokenize{testing_documentation/apantec_testing:id2}}\end{figure}


\subsection{Testing Alert alarm}
\label{\detokenize{testing_documentation/apantec_testing:testing-alert-alarm}}\begin{enumerate}
\sphinxsetlistlabels{\arabic}{enumi}{enumii}{}{.}%
\item {} 
\sphinxAtStartPar
Using either the radiation monitoring program of using the Apantec physical interface, change the alarm set points to 0.
In response:
\begin{itemize}
\item {} 
\sphinxAtStartPar
The rate meter that controls the probe has the \DUrole{orange}{ALERT} LED on.

\item {} 
\sphinxAtStartPar
The Ionizing Radiation Interlock protocase AREA MONITOR RADIATION lamp turns \DUrole{red}{red}.

\item {} 
\sphinxAtStartPar
The yellow and white contact blocks lose 24VDC across them.

\item {} 
\sphinxAtStartPar
Manually change the alert set point back to 50.

\end{itemize}

\end{enumerate}


\begin{savenotes}\sphinxattablestart
\centering
\begin{tabulary}{\linewidth}[t]{|T|T|}
\hline
\sphinxstyletheadfamily 
\sphinxAtStartPar
\sphinxstylestrong{Gamma:}
&\sphinxstyletheadfamily 
\sphinxAtStartPar
\sphinxstylestrong{Neutron:}
\\
\hline
\sphinxAtStartPar
Hutch\sphinxhyphen{}1 Front
&
\sphinxAtStartPar
Hutch\sphinxhyphen{}1 Front
\\
\hline
\sphinxAtStartPar
Hutch\sphinxhyphen{}1 Black
&
\sphinxAtStartPar
Hutch\sphinxhyphen{}1 Black
\\
\hline
\sphinxAtStartPar
Vault\sphinxhyphen{}1 Control
&
\sphinxAtStartPar
Vault\sphinxhyphen{}1 Control
\\
\hline
\sphinxAtStartPar
Laser\sphinxhyphen{}1
&
\sphinxAtStartPar
N/A
\\
\hline
\sphinxAtStartPar
RF\sphinxhyphen{}1
&
\sphinxAtStartPar
N/A
\\
\hline
\end{tabulary}
\par
\sphinxattableend\end{savenotes}

\begin{figure}[htbp]
\centering
\capstart

\noindent\sphinxincludegraphics[scale=0.2]{{Apantec_alert}.jpg}
\caption{\sphinxstylestrong{Figure 3:} This is an example of the RMW1 rate meter control in an alert state.}\label{\detokenize{testing_documentation/apantec_testing:id3}}\end{figure}

\begin{figure}[htbp]
\centering
\capstart

\noindent\sphinxincludegraphics[scale=0.2]{{Hutch-1_Control_protocase_radiation_fail}.jpg}
\caption{\sphinxstylestrong{Figure 4:} This is an example of the Hutch\sphinxhyphen{}1 Control IONIZING RADIATION INTERLOCK protocase AREA MONITOR RADIATION lamp in a fail state.
This should occur on the Vault\sphinxhyphen{}1 Control IONIZING RADIATION INTERLOCK protocase AREA MONITOR RADIATION lamp as well.}\label{\detokenize{testing_documentation/apantec_testing:id4}}\end{figure}


\subsection{Testing High Alarm}
\label{\detokenize{testing_documentation/apantec_testing:testing-high-alarm}}\begin{enumerate}
\sphinxsetlistlabels{\arabic}{enumi}{enumii}{}{.}%
\item {} 
\sphinxAtStartPar
Secure Hutch\sphinxhyphen{}1.
In Vault\sphinxhyphen{}1 Control west panel, relays R3 and R5 should have power, and relay R4 should have no power.

\item {} 
\sphinxAtStartPar
Change the alert alarm settings on any of the Hutch1 probes to zero.
Nothing should happen, change the value back.

\item {} 
\sphinxAtStartPar
Using the same methods as with the alert alarm setting, change the high alarm setting to zero on each probe one at a time.
In response:
\begin{itemize}
\item {} 
\sphinxAtStartPar
The rate meter that controls the probe has the \DUrole{red}{HIGH ALARM} LED on.

\item {} 
\sphinxAtStartPar
The rate meter that control the probe has an audible alarm.

\item {} 
\sphinxAtStartPar
The IONIZING RADIATION INTERLOCK protocase AREA MONITOR RADIATION lamps turn \DUrole{red}{red}.

\item {} 
\sphinxAtStartPar
The yellow and white contact blocks lose 24VDC across them.

\item {} 
\sphinxAtStartPar
Manually change the set point back to 500.

\item {} 
\sphinxAtStartPar
Relay 16 in Hutch\sphinxhyphen{}1 panel loses power.

\end{itemize}

\end{enumerate}


\begin{savenotes}\sphinxattablestart
\centering
\begin{tabulary}{\linewidth}[t]{|T|T|}
\hline
\sphinxstyletheadfamily 
\sphinxAtStartPar
\sphinxstylestrong{Gamma:}
&\sphinxstyletheadfamily 
\sphinxAtStartPar
\sphinxstylestrong{Neutron:}
\\
\hline
\sphinxAtStartPar
Hutch\sphinxhyphen{}1 Front
&
\sphinxAtStartPar
Hutch\sphinxhyphen{}1 Front
\\
\hline
\sphinxAtStartPar
Hutch\sphinxhyphen{}1 Black
&
\sphinxAtStartPar
Hutch\sphinxhyphen{}1 Black
\\
\hline
\end{tabulary}
\par
\sphinxattableend\end{savenotes}

\begin{figure}[htbp]
\centering
\capstart

\noindent\sphinxincludegraphics[scale=0.2]{{Apantec_alarm}.jpg}
\caption{\sphinxstylestrong{Figure 5:} This is an example of the RMW1 rate meter control in an alarm state.}\label{\detokenize{testing_documentation/apantec_testing:id5}}\end{figure}


\subsection{Emergency Tungsten Shutter Crash}
\label{\detokenize{testing_documentation/apantec_testing:emergency-tungsten-shutter-crash}}\begin{enumerate}
\sphinxsetlistlabels{\arabic}{enumi}{enumii}{}{.}%
\item {} 
\sphinxAtStartPar
Secure Hutch\sphinxhyphen{}1.

\item {} 
\sphinxAtStartPar
Set the BEAM SELECT to COLLIMATED.

\item {} 
\sphinxAtStartPar
Change the Apantec gamma probe high alarm set point to 0 in Hutch\sphinxhyphen{}1.
In response:
\begin{itemize}
\item {} 
\sphinxAtStartPar
Hutch\sphinxhyphen{}1 Control Ionizing Radiation Interlock protocase lamps for Beam Status turn \DUrole{red}{red}.

\item {} 
\sphinxAtStartPar
The COLLIMATED shutter (furthest, right side up shutter) closed.

\end{itemize}

\end{enumerate}

\sphinxstepscope


\section{Narda Smarts II Microwave Sensor}
\label{\detokenize{testing_documentation/narda_testing:narda-smarts-ii-microwave-sensor}}\label{\detokenize{testing_documentation/narda_testing::doc}}
\sphinxAtStartPar
The purpose of this procedure is to validate the functionality of the Vault\sphinxhyphen{}1 and Hutch\sphinxhyphen{}1 Narda Smarts II Microwave Sensor.


\subsection{Starting Conditions}
\label{\detokenize{testing_documentation/narda_testing:starting-conditions}}\begin{enumerate}
\sphinxsetlistlabels{\arabic}{enumi}{enumii}{}{.}%
\item {} 
\sphinxAtStartPar
Narda meters are not alarming.

\item {} 
\sphinxAtStartPar
All meters are calibrated.
\begin{itemize}
\item {} 
\sphinxAtStartPar
Vault\sphinxhyphen{}1

\item {} 
\sphinxAtStartPar
RF\sphinxhyphen{}1

\end{itemize}

\item {} 
\sphinxAtStartPar
Vault\sphinxhyphen{}1 Control IONIZING RADIATION INTERLOCK protocase MICROWAVE AREA MONITOR lamp is \DUrole{green}{green}.

\item {} 
\sphinxAtStartPar
Narda relay contact \#10 in Vault\sphinxhyphen{}1 Control West aggregator panel LEDs are all on.

\item {} 
\sphinxAtStartPar
\DUrole{yellow}{Yellow} and \DUrole{blue}{blue} contact blocks in Vault\sphinxhyphen{}1 Control west aggregator panel have continuity to each other.

\end{enumerate}

\begin{figure}[htbp]
\centering
\capstart

\noindent\sphinxincludegraphics[scale=0.2]{{Narda}.jpg}
\caption{\sphinxstylestrong{Figure 1:} This is the Narda Smarts II Microwave Sensors.
One is located on the Vault\sphinxhyphen{}1 north wall and the other is located on the RF\sphinxhyphen{}1 east wall.}\label{\detokenize{testing_documentation/narda_testing:id1}}\end{figure}

\begin{figure}[htbp]
\centering
\capstart

\noindent\sphinxincludegraphics[scale=0.2]{{yellow_blue_contacts}.jpg}
\caption{\sphinxstylestrong{Figure 2:} These are the yellow and blue contact blocks in the Vault\sphinxhyphen{}1 Control west panel that correspond to the Narda meters.}\label{\detokenize{testing_documentation/narda_testing:id2}}\end{figure}


\subsection{Testing}
\label{\detokenize{testing_documentation/narda_testing:testing}}\begin{enumerate}
\sphinxsetlistlabels{\arabic}{enumi}{enumii}{}{.}%
\item {} 
\sphinxAtStartPar
Go up to the Narda meter in RF\sphinxhyphen{}1and press the red test button.
In response:
\begin{itemize}
\item {} 
\sphinxAtStartPar
An audible alarm from the meter.

\item {} 
\sphinxAtStartPar
IONIZING RADIATION INTERLOCK MICROWAVE AREA MONITOR lamp is Vault\sphinxhyphen{}1 Control turns \DUrole{red}{red}.

\item {} 
\sphinxAtStartPar
Relay 10 in the Vault\sphinxhyphen{}1 Control west panel shows all diagnostic LEDs on.

\item {} 
\sphinxAtStartPar
\DUrole{yellow}{Yellow} and \DUrole{blue}{blue} contact blocks will lose continuity with each other.

\end{itemize}

\item {} 
\sphinxAtStartPar
Repeat for Narda meter in Vault\sphinxhyphen{}1.

\end{enumerate}

\begin{figure}[htbp]
\centering
\capstart

\noindent\sphinxincludegraphics[scale=0.2]{{Vault-1_Control_protocase_microwave_fail}.jpg}
\caption{\sphinxstylestrong{Figure 3:} This is the Vault\sphinxhyphen{}1 Control IONIZING RADIATION INTERLOCK protocase MICROWAVE AREA MONITOR lamp when the Narda meter is in alarm.
There is no lamp that corresponds to the Hutch\sphinxhyphen{}1 Control IONIZING RADIATION INTERLOCK protocase.}\label{\detokenize{testing_documentation/narda_testing:id3}}\end{figure}

\sphinxstepscope


\section{O2 Sensor Testing Protocol}
\label{\detokenize{testing_documentation/O2_testing:o2-sensor-testing-protocol}}\label{\detokenize{testing_documentation/O2_testing::doc}}
\sphinxAtStartPar
The purpose of this document is to test the functionality of the O2 sensors in Vault\sphinxhyphen{}1, Hutch\sphinxhyphen{}1, and RF\sphinxhyphen{}1.


\subsection{Staring Conditions}
\label{\detokenize{testing_documentation/O2_testing:staring-conditions}}\begin{enumerate}
\sphinxsetlistlabels{\arabic}{enumi}{enumii}{}{.}%
\item {} 
\sphinxAtStartPar
O2 main and remote units only show LED for POWER.
\begin{itemize}
\item {} 
\sphinxAtStartPar
Vault\sphinxhyphen{}1 main unit

\item {} 
\sphinxAtStartPar
Vault\sphinxhyphen{}1 Control remote unit

\item {} 
\sphinxAtStartPar
RF\sphinxhyphen{}1 main unit

\item {} 
\sphinxAtStartPar
Accelerator Lab remote unit

\item {} 
\sphinxAtStartPar
Hutch\sphinxhyphen{}1 main unit

\item {} 
\sphinxAtStartPar
Hutch\sphinxhyphen{}1 Control remote unit (top)

\item {} 
\sphinxAtStartPar
Astrella enclosure main unit

\item {} 
\sphinxAtStartPar
Hutch\sphinxhyphen{}1 Control remote unit (bottom)

\end{itemize}

\item {} 
\sphinxAtStartPar
Vault\sphinxhyphen{}1 Control IONIZING RADIATION INTERLOCK protocase AREA MONITOR OXYGEN lamp is \DUrole{green}{green}.

\item {} 
\sphinxAtStartPar
Relay contact \#9 in Vault\sphinxhyphen{}1 Control west aggregator panel has LEDs on.

\item {} 
\sphinxAtStartPar
\DUrole{yellow}{Yellow} 4 contact blocks in Vault\sphinxhyphen{}1 Control west aggregator panel all have continuity with each other.

\end{enumerate}


\begin{savenotes}\sphinxattablestart
\centering
\begin{tabulary}{\linewidth}[t]{|T|T|}
\hline

\noindent{\hspace*{\fill}\sphinxincludegraphics[scale=0.2]{{Vault-1_O2_main}.jpg}\hspace*{\fill}}
&
\noindent{\hspace*{\fill}\sphinxincludegraphics[scale=0.2]{{Vault-1_O2_remote}.jpg}\hspace*{\fill}}
\\
\hline
\sphinxAtStartPar
O2 main unit. \DUrole{white-cell}{=====================================================================}
&
\sphinxAtStartPar
O2 remote unit. \DUrole{white-cell}{===================================================================}
\\
\hline
\end{tabulary}
\par
\sphinxattableend\end{savenotes}

\begin{sphinxuseclass}{tight-table-caption-container}
\sphinxAtStartPar
\sphinxstylestrong{Figure 1:} This is the O2 sensor pair. The O2 main unit is the sensor, which sends data for display to the O2 remote unit. Both display the same information.

\end{sphinxuseclass}
\begin{figure}[htbp]
\centering
\capstart

\noindent\sphinxincludegraphics[scale=0.2]{{yellow_4_contacts}.jpg}
\caption{\sphinxstylestrong{Figure 2:} These are the yellow 4 contact blocks in the Vault\sphinxhyphen{}1 Control west aggregator panel.
If any of the O2 sensors alarm, these blocks will loose continuity with each other.}\label{\detokenize{testing_documentation/O2_testing:id1}}\end{figure}


\subsection{Testing Connectivity Between Main and Remote Units}
\label{\detokenize{testing_documentation/O2_testing:testing-connectivity-between-main-and-remote-units}}\begin{enumerate}
\sphinxsetlistlabels{\arabic}{enumi}{enumii}{}{.}%
\item {} 
\sphinxAtStartPar
On the face of the main unit and remote unit, press mode until DIAG flashes on the display. Press enter, in return:
\begin{itemize}
\item {} 
\sphinxAtStartPar
All LEDs on the units will light.

\item {} 
\sphinxAtStartPar
The units will alarm.
\begin{itemize}
\item {} 
\sphinxAtStartPar
Vault\sphinxhyphen{}1 / Vault\sphinxhyphen{}1 Control

\item {} 
\sphinxAtStartPar
RF\sphinxhyphen{}1 / Accelerator Lab

\item {} 
\sphinxAtStartPar
Hutch\sphinxhyphen{}1 / Hutch\sphinxhyphen{}1 Control (top)

\item {} 
\sphinxAtStartPar
Astrella enclosure / Hutch\sphinxhyphen{}1 Control (bottom)

\end{itemize}

\end{itemize}

\end{enumerate}


\subsection{Testing Alarming}
\label{\detokenize{testing_documentation/O2_testing:testing-alarming}}\begin{enumerate}
\sphinxsetlistlabels{\arabic}{enumi}{enumii}{}{.}%
\item {} 
\sphinxAtStartPar
Using compressed gas (electronics duster will work fine), spray the gas into the main controller sensor until the unit sees an \(O_2\) concentration of less than 19\%.
\begin{itemize}
\item {} 
\sphinxAtStartPar
The main and remote units will audibly alarm.

\item {} 
\sphinxAtStartPar
The ELDs for AL1 to flash. More AL\# LEDs may flash depending on if you break the 17\% and or 15\% thresholds.
However, the AL1 relay is what is critical.

\item {} 
\sphinxAtStartPar
The Vault\sphinxhyphen{}1 Control IONIZING RADIATION INTERLOCK protocase AREA MONITOR OXYGEN lamp will turn \DUrole{red}{red}.

\item {} 
\sphinxAtStartPar
Relay contract \#9 in the Vault\sphinxhyphen{}1 Control west aggregator panel LEDs will turn off.

\item {} 
\sphinxAtStartPar
The correct \DUrole{yellow}{yellow 4} contact blocks will lose continuity with each other.

\item {} 
\sphinxAtStartPar
Corresponding \DUrole{orange}{orange} beacon to the unit being tested will flash.

\item {} 
\sphinxAtStartPar
These units have latching alarms.
Press the bottom left recessed button to rest the alarm.
\begin{itemize}
\item {} 
\sphinxAtStartPar
Vault\sphinxhyphen{}1 / Vault\sphinxhyphen{}1 Control

\item {} 
\sphinxAtStartPar
RF\sphinxhyphen{}1 / Accelerator Lab

\item {} 
\sphinxAtStartPar
Hutch\sphinxhyphen{}1 / Hutch\sphinxhyphen{}1 Control (top)

\item {} 
\sphinxAtStartPar
Astrella enclosure / Hutch\sphinxhyphen{}1 Control (bottom)

\end{itemize}

\end{itemize}

\end{enumerate}


\subsection{Emergency Tungsten Shutter Crash}
\label{\detokenize{testing_documentation/O2_testing:emergency-tungsten-shutter-crash}}\begin{enumerate}
\sphinxsetlistlabels{\arabic}{enumi}{enumii}{}{.}%
\item {} 
\sphinxAtStartPar
Secure Hutch\sphinxhyphen{}1.

\item {} 
\sphinxAtStartPar
Set the BEAM SELECT to DIVERGENT.

\item {} 
\sphinxAtStartPar
Spray compressed gas into a O2 unit. In response:
\begin{itemize}
\item {} 
\sphinxAtStartPar
Hutch\sphinxhyphen{}1 Control Ionizing Radiation Interlock protocase lamps for Beam Status turn \DUrole{red}{red}.

\item {} 
\sphinxAtStartPar
The DIVERGENT shutter (closest, upside down shutter) closed.

\end{itemize}

\end{enumerate}

\sphinxstepscope


\section{Laser\sphinxhyphen{}1 Interlock System Testing Protocol}
\label{\detokenize{testing_documentation/Laser-1:laser-1-interlock-system-testing-protocol}}\label{\detokenize{testing_documentation/Laser-1::doc}}
\sphinxAtStartPar
The purpose of this protocol is to test the Laser\sphinxhyphen{}1 interlocks system.
This includes:
\begin{itemize}
\item {} 
\sphinxAtStartPar
Arming Laser\sphinxhyphen{}1.

\item {} 
\sphinxAtStartPar
Arming the Dira.

\end{itemize}


\subsection{Starting Conditions}
\label{\detokenize{testing_documentation/Laser-1:starting-conditions}}\begin{enumerate}
\sphinxsetlistlabels{\arabic}{enumi}{enumii}{}{.}%
\item {} 
\sphinxAtStartPar
The Vault\sphinxhyphen{}1 Control VIEWMARQ only displays \DUrole{green}{LASER SAFE}.

\item {} 
\sphinxAtStartPar
The Vault\sphinxhyphen{}1 beacon stacks show only green LEDs on.
\begin{itemize}
\item {} 
\sphinxAtStartPar
Vault\sphinxhyphen{}1 Control

\item {} 
\sphinxAtStartPar
Vault\sphinxhyphen{}1 east wall

\item {} 
\sphinxAtStartPar
Pharos LASER ENCLOSURE INTERLOCK protocase

\item {} 
\sphinxAtStartPar
Dira LASER ENCLOSURE INTERLOCK protocase

\end{itemize}

\item {} 
\sphinxAtStartPar
Dira enclosure e\sphinxhyphen{}stops are not engaged, and the LEDs are off.
\begin{itemize}
\item {} 
\sphinxAtStartPar
East wall

\item {} 
\sphinxAtStartPar
West wall

\end{itemize}

\item {} 
\sphinxAtStartPar
The Laser Lab VIEWMARQs display \DUrole{green}{LASER SAFE}.
\begin{itemize}
\item {} 
\sphinxAtStartPar
Laser\sphinxhyphen{}1 entrance.

\item {} 
\sphinxAtStartPar
Laser\sphinxhyphen{}1 airlock.

\end{itemize}

\item {} 
\sphinxAtStartPar
Laser warning module in Laser\sphinxhyphen{}1 entrance shows \DUrole{green}{LASER SAFE}.

\item {} 
\sphinxAtStartPar
Entry keypad has \DUrole{green}{RELEASE} LED on.

\item {} 
\sphinxAtStartPar
Control module in Laser\sphinxhyphen{}1 airlock shows \DUrole{green}{LASER SAFE}.

\item {} 
\sphinxAtStartPar
Push to exit module is not on.

\item {} 
\sphinxAtStartPar
Laser\sphinxhyphen{}1 \sphinxhyphen{}estops are not engaged and LEDs are off.
\begin{itemize}
\item {} 
\sphinxAtStartPar
Laser\sphinxhyphen{}1 south\sphinxhyphen{}West

\item {} 
\sphinxAtStartPar
Laser\sphinxhyphen{}1 north\sphinxhyphen{}West

\item {} 
\sphinxAtStartPar
Laser\sphinxhyphen{}1 north

\item {} 
\sphinxAtStartPar
Laser\sphinxhyphen{}1 east

\end{itemize}

\item {} 
\sphinxAtStartPar
Laser\sphinxhyphen{}1 room interlock module has LED on for \DUrole{green}{ROOM DISARMED (READY TO ARM)}.

\item {} 
\sphinxAtStartPar
All local interlock modules have LEDs on for \DUrole{green}{LOCAL CONTACTS DISARMED} and \DUrole{green}{ROOM NOT ARMED \sphinxhyphen{} LOCAL CONTACT CANNOT ARM}.

\item {} 
\sphinxAtStartPar
All local interlock modules have LEDs on for \DUrole{green}{LOCAL CONTACT DISARMED} and \DUrole{green}{ROOM NOT ARMED \sphinxhyphen{} LOCAL CONTACT CANNOT ARM}.
\begin{itemize}
\item {} 
\sphinxAtStartPar
Dira

\item {} 
\sphinxAtStartPar
AUX 1

\item {} 
\sphinxAtStartPar
AUX 2

\item {} 
\sphinxAtStartPar
AUX 3

\item {} 
\sphinxAtStartPar
AUX 4

\end{itemize}

\item {} 
\sphinxAtStartPar
Door closure module next to RF\sphinxhyphen{}1 / Laser\sphinxhyphen{}1 door displays \DUrole{green}{CLOSED}.

\end{enumerate}


\begin{savenotes}\sphinxattablestart
\centering
\begin{tabulary}{\linewidth}[t]{|T|T|T|}
\hline

\noindent{\hspace*{\fill}\sphinxincludegraphics[scale=0.28]{{Vault-1_Control_VIEWMARQ_safe}.jpg}\hspace*{\fill}}
&
\noindent{\hspace*{\fill}\sphinxincludegraphics[scale=0.2]{{Laser-1_VIEWMARQ_entry_safe}.jpg}\hspace*{\fill}}
&
\noindent{\hspace*{\fill}\sphinxincludegraphics[scale=0.2]{{Laser-1_VIEWMARQ_airlock_safe}.jpg}\hspace*{\fill}}
\\
\hline
\sphinxAtStartPar
This is the Vault\sphinxhyphen{}1 Control VIEWMARQ in a safe condition. \DUrole{white-cell}{===================================}
&
\sphinxAtStartPar
This is the Laser\sphinxhyphen{}1 entrance VIEWMARQ in a safe condition. \DUrole{white-cell}{==================================}
&
\sphinxAtStartPar
This is the Laser\sphinxhyphen{}1 airlock VIEWMARQ in a safe condition. \DUrole{white-cell}{===================================}
\\
\hline
\end{tabulary}
\par
\sphinxattableend\end{savenotes}

\begin{sphinxuseclass}{tight-table-caption-container}
\sphinxAtStartPar
\sphinxstylestrong{Figure 1:} These are the VIEWMARQ displays that show the state of the Dira.

\end{sphinxuseclass}

\begin{savenotes}\sphinxattablestart
\centering
\begin{tabulary}{\linewidth}[t]{|T|T|T|T|}
\hline

\noindent{\hspace*{\fill}\sphinxincludegraphics[scale=0.76]{{Vault-1_Control_beacons}.jpg}\hspace*{\fill}}
&
\noindent{\hspace*{\fill}\sphinxincludegraphics[scale=0.2]{{Vault-1_beacons}.jpg}\hspace*{\fill}}
&
\noindent{\hspace*{\fill}\sphinxincludegraphics[scale=0.43]{{Pharos_beacons}.jpg}\hspace*{\fill}}
&
\noindent{\hspace*{\fill}\sphinxincludegraphics[scale=0.53]{{Dira_beacons}.jpg}\hspace*{\fill}}
\\
\hline
\sphinxAtStartPar
Vault\sphinxhyphen{}1 Control beacon stack. \DUrole{white-cell}{=========================================================}
&
\sphinxAtStartPar
Vault\sphinxhyphen{}1 beacon stack. \DUrole{white-cell}{=================================================================}
&
\sphinxAtStartPar
Pharos LASER ENCLOSURE INTERLOCK protocase beacon stack. \DUrole{white-cell}{==============================}
&
\sphinxAtStartPar
Dira LASER ENCLOSURE INTERLOCK protocase beacon stack. \DUrole{white-cell}{================================}
\\
\hline
\end{tabulary}
\par
\sphinxattableend\end{savenotes}

\begin{sphinxuseclass}{tight-table-caption-container}
\sphinxAtStartPar
\sphinxstylestrong{Figure 2:} These are the Vault\sphinxhyphen{}1 laser interlock system beacon stacks. These all show the state of the Dira.

\end{sphinxuseclass}
\begin{figure}[htbp]
\centering
\capstart

\noindent\sphinxincludegraphics[scale=0.2]{{Laser-1_control_module_safe}.jpg}
\caption{\sphinxstylestrong{Figure 3:} This is the control module in Laser\sphinxhyphen{}1 airlock in a safe condition}\label{\detokenize{testing_documentation/Laser-1:id1}}\end{figure}

\begin{figure}[htbp]
\centering
\capstart

\noindent\sphinxincludegraphics[scale=0.2]{{Laser-1_push_to_exit_safe}.jpg}
\caption{\sphinxstylestrong{Figure 4:} THis is the push\sphinxhyphen{}to\sphinxhyphen{}exit module in Laser\sphinxhyphen{}1 airlock in a safe condition.}\label{\detokenize{testing_documentation/Laser-1:id2}}\end{figure}

\begin{figure}[htbp]
\centering
\capstart

\noindent\sphinxincludegraphics[scale=0.2]{{Laser-1_entry_safe}.jpg}
\caption{\sphinxstylestrong{Figure 5:} This is the entry keypad in Laser\sphinxhyphen{}1 in a safe condition.}\label{\detokenize{testing_documentation/Laser-1:id3}}\end{figure}

\begin{figure}[htbp]
\centering
\capstart

\noindent\sphinxincludegraphics[scale=0.2]{{e-stop_off}.jpg}
\caption{\sphinxstylestrong{Figure 6:} This is the e\sphinxhyphen{}stop module in Laser\sphinxhyphen{}1 in a safe condition.}\label{\detokenize{testing_documentation/Laser-1:id4}}\end{figure}

\begin{figure}[htbp]
\centering
\capstart

\noindent\sphinxincludegraphics[scale=0.2]{{RF_door}.jpg}
\caption{\sphinxstylestrong{Figure 7:} This is the Laser\sphinxhyphen{}1 to RF\sphinxhyphen{}1 door monitor in a closed condition.}\label{\detokenize{testing_documentation/Laser-1:id5}}\end{figure}

\begin{figure}[htbp]
\centering
\capstart

\noindent\sphinxincludegraphics[scale=0.2]{{Laser-1_arming_panel}.jpg}
\caption{\sphinxstylestrong{Figure 8:} This is the arming panel for Laser\sphinxhyphen{}1.}\label{\detokenize{testing_documentation/Laser-1:id6}}\end{figure}


\subsection{Arming Laser\sphinxhyphen{}1}
\label{\detokenize{testing_documentation/Laser-1:arming-laser-1}}\begin{enumerate}
\sphinxsetlistlabels{\arabic}{enumi}{enumii}{}{.}%
\item {} 
\sphinxAtStartPar
When entering Laser\sphinxhyphen{}1, there should be an audible chime.

\item {} 
\sphinxAtStartPar
Attempt to arm all the local interlock modules before arming Laser\sphinxhyphen{}1.
The Dira will not arm.

\item {} 
\sphinxAtStartPar
Arm the Laser\sphinxhyphen{}1 room control interlock module.
Laser\sphinxhyphen{}1 room interlock module has LED on for \DUrole{orange}{ROOM ARMED}.

\item {} 
\sphinxAtStartPar
All local interlock modules have LEDs on for \DUrole{green}{LOCAL CONTACTS DISARMED}.
\begin{itemize}
\item {} 
\sphinxAtStartPar
Dira

\item {} 
\sphinxAtStartPar
AUX 1

\item {} 
\sphinxAtStartPar
AUX 2

\item {} 
\sphinxAtStartPar
AUX 3

\item {} 
\sphinxAtStartPar
AUX 4

\end{itemize}

\item {} 
\sphinxAtStartPar
VIEWMARQ displays show \DUrole{red}{DANGER LASER ON}.
\begin{itemize}
\item {} 
\sphinxAtStartPar
Laser\sphinxhyphen{}1 entrance.

\item {} 
\sphinxAtStartPar
Laser\sphinxhyphen{}1 airlock.

\end{itemize}

\item {} 
\sphinxAtStartPar
Dira LASER ENCLOSURE INTERLOCK protocase laser warning module shows \DUrole{red}{DANGER LASER ON}.

\item {} 
\sphinxAtStartPar
Dira LASER ENCLOSURE INTERLOCK protocase CONTROL CONTACTS AUX \#1 and \#2 should be auto\sphinxhyphen{}armed from arming Laser\sphinxhyphen{}1.
They cannot be disarmed.

\item {} 
\sphinxAtStartPar
Laser warning module displays \DUrole{red}{DANGER LASER ON}.

\item {} 
\sphinxAtStartPar
Entry keypad has LED on for \DUrole{green}{INTERLOCKED}.

\item {} 
\sphinxAtStartPar
Laser control module displays \DUrole{red}{DANGER LASER ON}.

\item {} 
\sphinxAtStartPar
The push to exit button is on.

\item {} 
\sphinxAtStartPar
The airlock / corridor door is magnetically locked.

\item {} 
\sphinxAtStartPar
Use the push to exit button to leave Laser\sphinxhyphen{}1.

\item {} 
\sphinxAtStartPar
Use a random pin on the entry keypad module and scan your badge.
The door should remain locked.

\item {} 
\sphinxAtStartPar
Use the correct pin and scan your badge.
Hold the door open for \DUrole{red}{x seconds} and allow the interlock system to trip.
The system should return to a completely disarmed state.

\end{enumerate}


\begin{savenotes}\sphinxattablestart
\centering
\begin{tabulary}{\linewidth}[t]{|T|T|}
\hline

\noindent{\hspace*{\fill}\sphinxincludegraphics[scale=0.2]{{Laser-1_VIEWMARQ_entry_armed}.jpg}\hspace*{\fill}}
&
\noindent{\hspace*{\fill}\sphinxincludegraphics[scale=0.2]{{Laser-1_VIEWMARQ_airlock_armed}.jpg}\hspace*{\fill}}
\\
\hline
\sphinxAtStartPar
This is the Laser\sphinxhyphen{}1 entrance VIEWMARQ in an armed condition. \DUrole{white-cell}{================================}
&
\sphinxAtStartPar
This is the Laser\sphinxhyphen{}1 airlock VIEWMARQ in an armed condition. \DUrole{white-cell}{=================================}
\\
\hline
\end{tabulary}
\par
\sphinxattableend\end{savenotes}

\begin{sphinxuseclass}{tight-table-caption-container}
\sphinxAtStartPar
\sphinxstylestrong{Figure 9:} These are the Laser\sphinxhyphen{}1 VIEWMARQ displays when Laser\sphinxhyphen{}1 is armed.

\end{sphinxuseclass}
\begin{figure}[htbp]
\centering
\capstart

\noindent\sphinxincludegraphics[scale=0.2]{{Laser-1_control_module_armed}.jpg}
\caption{\sphinxstylestrong{Figure 10:} This is the control module in Laser\sphinxhyphen{}1 airlock in a safe condition}\label{\detokenize{testing_documentation/Laser-1:id7}}\end{figure}

\begin{figure}[htbp]
\centering
\capstart

\noindent\sphinxincludegraphics[scale=0.2]{{Laser-1_push_to_exit}.jpg}
\caption{\sphinxstylestrong{Figure 11:} THis is the push\sphinxhyphen{}to\sphinxhyphen{}exit module in Laser\sphinxhyphen{}1 airlock in a safe condition.}\label{\detokenize{testing_documentation/Laser-1:id8}}\end{figure}

\begin{figure}[htbp]
\centering
\capstart

\noindent\sphinxincludegraphics[scale=0.2]{{Laser-1_entry_armed}.jpg}
\caption{\sphinxstylestrong{Figure 12:} This is the entry keypad in Laser\sphinxhyphen{}1 in a safe condition.}\label{\detokenize{testing_documentation/Laser-1:id9}}\end{figure}

\begin{figure}[htbp]
\centering
\capstart

\noindent\sphinxincludegraphics[scale=0.2]{{e-stop_on}.jpg}
\caption{\sphinxstylestrong{Figure 13:} This is the e\sphinxhyphen{}stop module in Laser\sphinxhyphen{}1 in a safe condition.}\label{\detokenize{testing_documentation/Laser-1:id10}}\end{figure}


\subsection{Arming the Dira}
\label{\detokenize{testing_documentation/Laser-1:arming-the-dira}}\begin{enumerate}
\sphinxsetlistlabels{\arabic}{enumi}{enumii}{}{.}%
\item {} 
\sphinxAtStartPar
Arm the Dira local interlock module.
Dira local interlock module has LED on for \DUrole{orange}{LOCAL CONTACTS ARMED}.

\item {} 
\sphinxAtStartPar
The VIEWMARQ display in Laser\sphinxhyphen{}1 airlock \DUrole{red}{DANGER LASER ON \sphinxhyphen{} IR EYE PROTECTION REQUIRED}.

\item {} 
\sphinxAtStartPar
The VIEWMARQ display in the Laser\sphinxhyphen{}1 entrance shows \DUrole{red}{DANGER LASER ON \sphinxhyphen{} IR HAZARD}.

\item {} 
\sphinxAtStartPar
The VIEWMARQ display in Vault\sphinxhyphen{}1 Control shows \DUrole{green}{LASER SAFE} \sphinxhyphen{} \DUrole{red}{DIRA ARMED}.

\item {} 
\sphinxAtStartPar
The Vault\sphinxhyphen{}1 beacon stacks should show \DUrole{green}{green} and white LEDs on.
\begin{itemize}
\item {} 
\sphinxAtStartPar
Vault\sphinxhyphen{}1 control.

\item {} 
\sphinxAtStartPar
Vault\sphinxhyphen{}1 east wall.

\item {} 
\sphinxAtStartPar
Dira LASER ENCLOSURE INTERLOCK protocase.

\end{itemize}

\item {} 
\sphinxAtStartPar
Beacon stack on the Pharos LASER ENCLOSURE INTERLOCK protocase will only have a \DUrole{green}{green} LED on.

\item {} 
\sphinxAtStartPar
Dira LASER ENCLOSURE INTERLOCK protocase laser warning module shows \DUrole{red}{DANGER LASER ON}.

\end{enumerate}


\begin{savenotes}\sphinxattablestart
\centering
\begin{tabulary}{\linewidth}[t]{|T|T|T|}
\hline

\noindent{\hspace*{\fill}\sphinxincludegraphics[scale=0.2]{{Laser-1_VIEWMARQ_entry_IR}.jpg}\hspace*{\fill}}
&
\noindent{\hspace*{\fill}\sphinxincludegraphics[scale=0.56]{{Laser-1_VIEWMARQ_airlock_IR}.gif}\hspace*{\fill}}
&
\noindent{\hspace*{\fill}\sphinxincludegraphics[scale=0.2]{{Vault-1_Control_VIEWMARQ_dira_armed}.jpg}\hspace*{\fill}}
\\
\hline
\sphinxAtStartPar
This is the Laser\sphinxhyphen{}1 entrance VIEWMARQ when the Dira is armed. \DUrole{white-cell}{===============================}
&
\sphinxAtStartPar
This is the Laser\sphinxhyphen{}1 airlock VIEWMARQ when the Dira is armed. \DUrole{white-cell}{================================}
&
\sphinxAtStartPar
This is the Vault\sphinxhyphen{}1 Control VIEWMARQ when the Dira is armed. \DUrole{white-cell}{================================}
\\
\hline
\end{tabulary}
\par
\sphinxattableend\end{savenotes}

\begin{sphinxuseclass}{tight-table-caption-container}
\sphinxAtStartPar
\sphinxstylestrong{Figure 14:} These are all the VIEWMARQ displays that will update when the Dira is armed.

\end{sphinxuseclass}

\subsection{Safe Dira E\sphinxhyphen{}Stop Test}
\label{\detokenize{testing_documentation/Laser-1:safe-dira-e-stop-test}}\begin{enumerate}
\sphinxsetlistlabels{\arabic}{enumi}{enumii}{}{.}%
\item {} 
\sphinxAtStartPar
Put the Dira into a powered down state.

\item {} 
\sphinxAtStartPar
Arm Laser\sphinxhyphen{}1 and the Dira.

\item {} 
\sphinxAtStartPar
Press one of the Dira enclosure laser e\sphinxhyphen{}stops.

\item {} 
\sphinxAtStartPar
Verify that the Dira power supply is cut off.

\end{enumerate}


\subsection{RF\sphinxhyphen{}1 Door}
\label{\detokenize{testing_documentation/Laser-1:rf-1-door}}\begin{enumerate}
\sphinxsetlistlabels{\arabic}{enumi}{enumii}{}{.}%
\item {} 
\sphinxAtStartPar
With the Dira not armed, open the door between RF\sphinxhyphen{}1 and Laser\sphinxhyphen{}1.
The door monitor module should display nothing.

\end{enumerate}

\begin{figure}[htbp]
\centering
\capstart

\noindent\sphinxincludegraphics[scale=0.2]{{door_monitor_open}.jpg}
\caption{\sphinxstylestrong{Figure 15:} This is the Laser\sphinxhyphen{}1 to RF\sphinxhyphen{}1 door monitor when the door is open.}\label{\detokenize{testing_documentation/Laser-1:id11}}\end{figure}


\subsection{Crashing the Dira}
\label{\detokenize{testing_documentation/Laser-1:crashing-the-dira}}\begin{enumerate}
\sphinxsetlistlabels{\arabic}{enumi}{enumii}{}{.}%
\item {} 
\sphinxAtStartPar
Once every 6 months, the Dira laser emergency stop buttons are testing that they can successfully cut power to the Dira from a fully armed state.
Verify if the last testing date was 6 months ago.

\item {} 
\sphinxAtStartPar
If 6 months have passed, arm the Dira laser and use one of the Dira laser e\sphinxhyphen{}stops to cut power from the Dira in an armed state.

\end{enumerate}

\sphinxstepscope


\section{Vault\sphinxhyphen{}1 Laser Interlock System Testing Protocol}
\label{\detokenize{testing_documentation/Vault-1_laser:vault-1-laser-interlock-system-testing-protocol}}\label{\detokenize{testing_documentation/Vault-1_laser::doc}}
\sphinxAtStartPar
The purpose of this testing procedure if the Verify the functionality of the Vault\sphinxhyphen{}1 laser interlock system including arming Vault\sphinxhyphen{}1 as a laser lab, and arming the Pharos system.
Additionally, the administrative override on the Pharos and Dira enclosures will be tested.


\subsection{Starting Conditions}
\label{\detokenize{testing_documentation/Vault-1_laser:starting-conditions}}\begin{enumerate}
\sphinxsetlistlabels{\arabic}{enumi}{enumii}{}{.}%
\item {} 
\sphinxAtStartPar
VIEWMARQ in Vault\sphinxhyphen{}1 Control displays \DUrole{green}{LASER SAFE}.

\item {} 
\sphinxAtStartPar
Vault\sphinxhyphen{}1 beacon stacks show \DUrole{green}{green} LED on.
\begin{itemize}
\item {} 
\sphinxAtStartPar
Vault\sphinxhyphen{}1 Control.

\item {} 
\sphinxAtStartPar
Vault\sphinxhyphen{}1 east wall.

\item {} 
\sphinxAtStartPar
Pharos LASER ENCLOSURE INTERLOCK protocase.

\item {} 
\sphinxAtStartPar
Dira LASER ENCLOSURE INTERLOCK protoacse.

\end{itemize}

\item {} 
\sphinxAtStartPar
Vault\sphinxhyphen{}1 entry keypad LED for \DUrole{green}{RELEASED} is on.

\item {} 
\sphinxAtStartPar
Vault\sphinxhyphen{}1 laser warning modules show \DUrole{green}{LASER SAFE}.
\begin{itemize}
\item {} 
\sphinxAtStartPar
Vault\sphinxhyphen{}1 Control.

\item {} 
\sphinxAtStartPar
Pharos enclosure south wall.

\item {} 
\sphinxAtStartPar
Pharos enclosure north wall.

\item {} 
\sphinxAtStartPar
Dira LASER ENCLOSURE INTERLOCK protoacse.

\end{itemize}

\item {} 
\sphinxAtStartPar
Laser control module shows \DUrole{green}{LASER SAFE}.
When the Vault\sphinxhyphen{}1 door is open, the module shows \DUrole{orange}{ACCESS}.

\item {} 
\sphinxAtStartPar
Laser warning module on the Pharos LASER ENCLOSURE INTERLOCK protocase shows \DUrole{red}{DANGER LASER ON}.

\item {} 
\sphinxAtStartPar
None of the laser E\sphinxhyphen{}stops are engaged or glowing.
\begin{itemize}
\item {} 
\sphinxAtStartPar
Pharos enclosure east wall.

\item {} 
\sphinxAtStartPar
Pharos enclosure north wall.

\item {} 
\sphinxAtStartPar
Dira enclosure east wall.

\item {} 
\sphinxAtStartPar
Dira enclosure west wall.

\end{itemize}

\item {} 
\sphinxAtStartPar
All room interlock modules LED for \DUrole{green}{ROOM DISABLED (READY TO ARM)} is on.
\begin{itemize}
\item {} 
\sphinxAtStartPar
Vault\sphinxhyphen{}1.

\item {} 
\sphinxAtStartPar
Pharos enclosure.

\end{itemize}

\item {} 
\sphinxAtStartPar
Local interlock modules LEDs for \DUrole{green}{LOCAL CONTACTS DISARMED}, and \DUrole{green}{ROOM NOT ARMED \sphinxhyphen{} LOCAL CONTACTS CANNOT ARM} is on.
\begin{itemize}
\item {} 
\sphinxAtStartPar
Pharos enclosure.

\item {} 
\sphinxAtStartPar
Dira LASER ENCLOSURE INTERLOCK protocase AUX \#1.

\item {} 
\sphinxAtStartPar
Dira LASER ENCLOSURE INTERLOCK protocase AUX \#2.

\end{itemize}

\item {} 
\sphinxAtStartPar
Local interlock modules LEDs for \DUrole{green}{LOCAL CONTACTS DISARMED} are on.
\begin{itemize}
\item {} 
\sphinxAtStartPar
Pharos LASER ENCLOSURE INTERLOCK protocase CONTROL \#1.

\item {} 
\sphinxAtStartPar
Pharos LASER ENCLOSURE INTERLOCK protocase CONTROL \#2.

\end{itemize}

\item {} 
\sphinxAtStartPar
Administrative override keys are set to INTERLOCK, with STATUS LED \DUrole{green}{green}.
\begin{itemize}
\item {} 
\sphinxAtStartPar
Pharos LASER ENCLOSURE INTERLOCK protocase INTERLOCK OVERRIDE.

\item {} 
\sphinxAtStartPar
Dira LASER ENCLOSURE INTERLOCK protocase INTERLOCK OVERRIDE.

\end{itemize}

\item {} 
\sphinxAtStartPar
Both enclosures are closed, and the Door / Curtain Monitors should show \DUrole{green}{CLOSED}.
\begin{itemize}
\item {} 
\sphinxAtStartPar
Pharos enclosure.

\item {} 
\sphinxAtStartPar
Dira enclosure.

\end{itemize}

\item {} 
\sphinxAtStartPar
Push before exit buttons is not on.

\end{enumerate}

\begin{figure}[htbp]
\centering
\capstart

\noindent\sphinxincludegraphics[scale=0.2]{{Vault-1_VIEWMARQ_safe}.jpg}
\caption{\sphinxstylestrong{Figure 1:} This is the Vault\sphinxhyphen{}1 Control VIEWMARQ display when the system is in a safe state.}\label{\detokenize{testing_documentation/Vault-1_laser:id1}}\end{figure}


\begin{savenotes}\sphinxattablestart
\centering
\begin{tabulary}{\linewidth}[t]{|T|T|T|T|}
\hline

\noindent{\hspace*{\fill}\sphinxincludegraphics[scale=0.76]{{Vault-1_Control_beacons}.jpg}\hspace*{\fill}}
&
\noindent{\hspace*{\fill}\sphinxincludegraphics[scale=0.2]{{Vault-1_beacons}.jpg}\hspace*{\fill}}
&
\noindent{\hspace*{\fill}\sphinxincludegraphics[scale=0.43]{{Pharos_beacons}.jpg}\hspace*{\fill}}
&
\noindent{\hspace*{\fill}\sphinxincludegraphics[scale=0.53]{{Dira_beacons}.jpg}\hspace*{\fill}}
\\
\hline
\sphinxAtStartPar
Vault\sphinxhyphen{}1 Control beacon stack. \DUrole{white-cell}{=========================================================}
&
\sphinxAtStartPar
Vault\sphinxhyphen{}1 beacon stack. \DUrole{white-cell}{=================================================================}
&
\sphinxAtStartPar
Pharos LASER ENCLOSURE INTERLOCK protocase beacon stack. \DUrole{white-cell}{==============================}
&
\sphinxAtStartPar
Dira LASER ENCLOSURE INTERLOCK protocase beacon stack. \DUrole{white-cell}{================================}
\\
\hline
\end{tabulary}
\par
\sphinxattableend\end{savenotes}

\begin{sphinxuseclass}{tight-table-caption-container}
\sphinxAtStartPar
\sphinxstylestrong{Table 2:} These are the Vault\sphinxhyphen{}1 laser interlock system beacon stacks.

\end{sphinxuseclass}
\begin{figure}[htbp]
\centering
\capstart

\noindent\sphinxincludegraphics[scale=0.2]{{Vault-1_entry_unarmed}.jpg}
\caption{\sphinxstylestrong{Figure 3:} This is the Vault\sphinxhyphen{}1 entry modules when the system is in a safe state.}\label{\detokenize{testing_documentation/Vault-1_laser:id2}}\end{figure}

\begin{figure}[htbp]
\centering
\capstart

\noindent\sphinxincludegraphics[scale=0.2]{{Vault-1_unarmed}.jpg}
\caption{\sphinxstylestrong{Figure 4:} This is the Vault\sphinxhyphen{}1 laser control module when the system is in a safe state.}\label{\detokenize{testing_documentation/Vault-1_laser:id3}}\end{figure}

\begin{figure}[htbp]
\centering
\capstart

\noindent\sphinxincludegraphics[scale=0.2]{{Pharos_protocase}.jpg}
\caption{\sphinxstylestrong{Figure 5:} This is the Pharos enclosure laser warning module when the system is in a safe state.}\label{\detokenize{testing_documentation/Vault-1_laser:id4}}\end{figure}

\begin{figure}[htbp]
\centering
\capstart

\noindent\sphinxincludegraphics[scale=0.2]{{Dira_protocase}.jpg}
\caption{\sphinxstylestrong{Figure 6:} This is the Dira enclosure laser warning module when the system is in a safe state.}\label{\detokenize{testing_documentation/Vault-1_laser:id5}}\end{figure}

\begin{figure}[htbp]
\centering
\capstart

\noindent\sphinxincludegraphics[scale=0.2]{{laser_e-stop_off1}.jpg}
\caption{\sphinxstylestrong{Figure 7:} This is the laser e\sphinxhyphen{}stop button when the system is in a safe state.}\label{\detokenize{testing_documentation/Vault-1_laser:id6}}\end{figure}


\subsection{Arming Vault\sphinxhyphen{}1 as a Laser Lab}
\label{\detokenize{testing_documentation/Vault-1_laser:arming-vault-1-as-a-laser-lab}}\begin{enumerate}
\sphinxsetlistlabels{\arabic}{enumi}{enumii}{}{.}%
\item {} 
\sphinxAtStartPar
While inside of Vault\sphinxhyphen{}1 with the vault door latched, press ARM on the room interlock arming module.
It should light the LED for \DUrole{orange}{ROOM ARMED}, and there will be an audible chime.

\item {} 
\sphinxAtStartPar
The laser control module shows \DUrole{red}{DANGER LASER ON}.

\item {} 
\sphinxAtStartPar
The push to exit button is on.

\item {} 
\sphinxAtStartPar
The Vault\sphinxhyphen{}1 door is magnetically locked.

\item {} 
\sphinxAtStartPar
The VIEWMARQ display in Vault\sphinxhyphen{}1 Control displays \DUrole{red}{DANGER LASER ON}.

\item {} 
\sphinxAtStartPar
Vault\sphinxhyphen{}1 laser warning modules display \DUrole{red}{DANGER LASER ON}.

\item {} 
\sphinxAtStartPar
Entry keypad LED for \DUrole{red}{INTERLOCKED} is on.

\item {} 
\sphinxAtStartPar
They in a random pin.
The Vault\sphinxhyphen{}1 door will not unlock.

\item {} 
\sphinxAtStartPar
Type in the correct pin and open the Vault\sphinxhyphen{}1 door.

\item {} 
\sphinxAtStartPar
The entry keypad LED for \DUrole{green}{RELEASED} is on.

\item {} 
\sphinxAtStartPar
Vault\sphinxhyphen{}1 beacon stacks show no LEDs on.
\begin{itemize}
\item {} 
\sphinxAtStartPar
Vault\sphinxhyphen{}1 Control.

\item {} 
\sphinxAtStartPar
Vault\sphinxhyphen{}1 east wall.

\item {} 
\sphinxAtStartPar
Pharos LASER ENCLOSURE INTERLOCK protocase.

\item {} 
\sphinxAtStartPar
Dira LASER ENCLOSURE INTERLOCK protocase.

\end{itemize}

\item {} 
\sphinxAtStartPar
Leave the vault door open for \DUrole{red}{x seconds} and allow the system to trip.
\begin{itemize}
\item {} 
\sphinxAtStartPar
The Vault\sphinxhyphen{}1 laser interlock system should return to its initial conditions.

\item {} 
\sphinxAtStartPar
The Vault\sphinxhyphen{}1 room arming module should show \DUrole{orange}{ROOM CRASHED (CANNOT ARM)}, then \DUrole{green}{ROOM DISABLED (READY TO ARM)} once the door is closed.

\end{itemize}

\end{enumerate}

\begin{figure}[htbp]
\centering
\capstart

\noindent\sphinxincludegraphics[scale=0.2]{{Vault-1_armed}.jpg}
\caption{\sphinxstylestrong{Figure 8:} This is the Vault\sphinxhyphen{}1 laser control module when the system is armed.}\label{\detokenize{testing_documentation/Vault-1_laser:id7}}\end{figure}

\begin{figure}[htbp]
\centering
\capstart

\noindent\sphinxincludegraphics[scale=0.2]{{Vault-1_VIEWMARQ_laser_hazard}.jpg}
\caption{\sphinxstylestrong{Figure 9:} This is the Vault\sphinxhyphen{}1 Control VIEWMARQ display when the system is armed.}\label{\detokenize{testing_documentation/Vault-1_laser:id8}}\end{figure}

\begin{figure}[htbp]
\centering
\capstart

\noindent\sphinxincludegraphics[scale=0.2]{{Vault-1_entry_armed}.jpg}
\caption{\sphinxstylestrong{Figure 10:} This is the Vault\sphinxhyphen{}1 entry modules when the system is armed.}\label{\detokenize{testing_documentation/Vault-1_laser:id9}}\end{figure}


\subsection{Arming the Pharos Enclosure and Laser}
\label{\detokenize{testing_documentation/Vault-1_laser:arming-the-pharos-enclosure-and-laser}}\begin{enumerate}
\sphinxsetlistlabels{\arabic}{enumi}{enumii}{}{.}%
\item {} 
\sphinxAtStartPar
With the Vault\sphinxhyphen{}1 unarmed, arm the room interlock module on the Pharos enclosure.
\begin{itemize}
\item {} 
\sphinxAtStartPar
The room interlock module only lights the LED for \DUrole{orange}{ROOM ARMED}.

\item {} 
\sphinxAtStartPar
The local interlock module will auto\sphinxhyphen{}arm only lights the LED for \DUrole{orange}{LOCAL CONTACTS ARMED}.

\item {} 
\sphinxAtStartPar
The laser warning control module shows \DUrole{red}{DANGER LASER ON}.

\end{itemize}

\item {} 
\sphinxAtStartPar
Laser E\sphinxhyphen{}stops buttons are on.
\begin{itemize}
\item {} 
\sphinxAtStartPar
Pharos enclosure west wall

\item {} 
\sphinxAtStartPar
Pharos enclosure north wall

\end{itemize}

\item {} 
\sphinxAtStartPar
The VIEWMARQ in Vault\sphinxhyphen{}1 Control will display \DUrole{green}{LASER SAFE} \sphinxhyphen{} \DUrole{red}{PHAROS ARMED}.

\item {} 
\sphinxAtStartPar
Beacon stacks show \DUrole{green}{green} and \DUrole{blue}{blue} LEDs activated.
\begin{itemize}
\item {} 
\sphinxAtStartPar
Vault\sphinxhyphen{}1 Control

\item {} 
\sphinxAtStartPar
Vault\sphinxhyphen{}1 east wall

\item {} 
\sphinxAtStartPar
Pharos LASER ENCLOSURE INTERLOCK protocase

\item {} 
\sphinxAtStartPar
Dira LASER ENCLOSURE INTERLOCK protocase

\end{itemize}

\item {} 
\sphinxAtStartPar
Change the Pharos LASER ENCLOSURE INTERLOCK protocase INTERLOCK OVERRIDE key from INTERLOCK to OVERRIDE.
The STATUS LED remains \DUrole{green}{green}. Change back to INTERLOCK.

\item {} 
\sphinxAtStartPar
Rearm Vault\sphinxhyphen{}1 as a laser lab.

\item {} 
\sphinxAtStartPar
The VIEWMARQ in Vault\sphinxhyphen{}1 Control displays \DUrole{red}{DANGER LASER HAZARD \sphinxhyphen{} PHAROS ARMED}.

\item {} 
\sphinxAtStartPar
Beacon stacks show \DUrole{blue}{blue} LED activated.
\begin{itemize}
\item {} 
\sphinxAtStartPar
Vault\sphinxhyphen{}1 Control

\item {} 
\sphinxAtStartPar
Vault\sphinxhyphen{}1 east wall

\item {} 
\sphinxAtStartPar
Pharos LASER ENCLOSURE INTERLOCK protocase

\item {} 
\sphinxAtStartPar
Dira LASER ENCLOSURE INTERLOCK protocase

\end{itemize}

\end{enumerate}


\begin{savenotes}\sphinxattablestart
\centering
\begin{tabulary}{\linewidth}[t]{|T|T|}
\hline

\noindent{\hspace*{\fill}\sphinxincludegraphics[scale=0.2]{{Pharos_enclosure_unarmed}.jpg}\hspace*{\fill}}
&
\noindent{\hspace*{\fill}\sphinxincludegraphics[scale=0.2]{{Pharos_enclosure_armed}.jpg}\hspace*{\fill}}
\\
\hline
\sphinxAtStartPar
Pharos enclosure when unarmed. \DUrole{white-cell}{======================================================}
&
\sphinxAtStartPar
Pharos enclosure when armed. \DUrole{white-cell}{========================================================}
\\
\hline
\end{tabulary}
\par
\sphinxattableend\end{savenotes}

\begin{sphinxuseclass}{tight-table-caption-container}
\sphinxAtStartPar
\sphinxstylestrong{Figure 11:} These are the Pharos enclosure laser warning modules when the system is unarmed and armed.

\end{sphinxuseclass}
\begin{figure}[htbp]
\centering
\capstart

\noindent\sphinxincludegraphics[scale=0.2]{{Vault-1_Control_VIEWMARQ_Pharos_armed_hazard}.jpg}
\caption{\sphinxstylestrong{Figure 12:} This is the Vault\sphinxhyphen{}1 Control VIEWMARQ display when the system is armed.}\label{\detokenize{testing_documentation/Vault-1_laser:id10}}\end{figure}

\begin{figure}[htbp]
\centering
\capstart

\noindent\sphinxincludegraphics[scale=0.2]{{laser_e-stop_on}.jpg}
\caption{\sphinxstylestrong{Figure 13:} This is the laser e\sphinxhyphen{}stop button when the system is armed.}\label{\detokenize{testing_documentation/Vault-1_laser:id11}}\end{figure}


\subsection{Safe Pharos E\sphinxhyphen{}Stop Test}
\label{\detokenize{testing_documentation/Vault-1_laser:safe-pharos-e-stop-test}}\begin{enumerate}
\sphinxsetlistlabels{\arabic}{enumi}{enumii}{}{.}%
\item {} 
\sphinxAtStartPar
Put the Pharos into a powered down state.

\item {} 
\sphinxAtStartPar
Arm the Pharos and the Pharos enclosure.

\item {} 
\sphinxAtStartPar
Press one of the Pharos enclosure laser e\sphinxhyphen{}stops.

\item {} 
\sphinxAtStartPar
Verify that the Pharos power supply is cut off.

\end{enumerate}


\subsection{Administrative Override on the Pharos Enclosure}
\label{\detokenize{testing_documentation/Vault-1_laser:administrative-override-on-the-pharos-enclosure}}\begin{enumerate}
\sphinxsetlistlabels{\arabic}{enumi}{enumii}{}{.}%
\item {} 
\sphinxAtStartPar
With the Pharos and Vault\sphinxhyphen{}1 armed, arm the LOCAL INTERLOCK CONTACT CONTROL modules on the Pharos LASER ENCLOSURE INTERLOCK protocase.
\begin{itemize}
\item {} 
\sphinxAtStartPar
CONTROL \#1

\item {} 
\sphinxAtStartPar
CONTROL \#2

\end{itemize}

\item {} 
\sphinxAtStartPar
Open the Pharos enclosure rolling doors. In response:
\begin{itemize}
\item {} 
\sphinxAtStartPar
Pharos LASER ENCLOSURE INTERLOCK protocase laser warning module will display \DUrole{green}{LASER SAFE}.

\item {} 
\sphinxAtStartPar
The LOCAL INTERLOCK CONTACT CONTROL modules will disarm and display \DUrole{orange}{LOCAL CONTACTS DISARMED}.

\item {} 
\sphinxAtStartPar
Pharos LASER ENCLOSURE INTERLOCK protocase door monitor will display nothing.

\item {} 
\sphinxAtStartPar
Pharos UV and IR shutters will close.

\item {} 
\sphinxAtStartPar
The Pharos power supply is cut off.

\end{itemize}

\item {} 
\sphinxAtStartPar
Rearm the contact controls, and repeat step 2 for all enclosure doors.
\begin{itemize}
\item {} 
\sphinxAtStartPar
East door

\item {} 
\sphinxAtStartPar
North door

\item {} 
\sphinxAtStartPar
South door

\end{itemize}

\item {} 
\sphinxAtStartPar
Turn the Pharos LASER ENCLOSURE INTERLOCK protocase INTERLOCK OVERRIDE key from \DUrole{red}{INTERLOCK} to \DUrole{red}{OVERRIDE}.
The STATUS LED will change to \DUrole{red}{red}.

\item {} 
\sphinxAtStartPar
The VIEWMARQ in Vault\sphinxhyphen{}1 Control will display \DUrole{red}{DANGER LASER HAZARD\sphinxhyphen{}PHAROS ARMED\sphinxhyphen{}PHAROS ADMIN OVERRIDE}.

\item {} 
\sphinxAtStartPar
Beacon stacks show \DUrole{orange}{orange} and \DUrole{blue}{blue} LEDs on.
\begin{itemize}
\item {} 
\sphinxAtStartPar
Vault\sphinxhyphen{}1 Control

\item {} 
\sphinxAtStartPar
Vault\sphinxhyphen{}1 east wall

\item {} 
\sphinxAtStartPar
Pharos LASER ENCLOSURE INTERLOCK protocase

\end{itemize}

\item {} 
\sphinxAtStartPar
Beacon stack on Dira LASER ENCLOSURE INTERLOCK protocase only shows \DUrole{blue}{blue} LED on.

\item {} 
\sphinxAtStartPar
Arm the LOCAL INTERLOCK CONTACT CONTROL modules on the Pharos LASER ENCLOSURE INTERLOCK protocase.
\begin{itemize}
\item {} 
\sphinxAtStartPar
CONTROL \#1

\item {} 
\sphinxAtStartPar
CONTROL \#2

\end{itemize}

\item {} 
\sphinxAtStartPar
With the Pharos, Vault\sphinxhyphen{}1, and LOCAL INTERLOCK CONTACT CONTROL modules armed and the Pharos enclosure set to override, open one of the Pharos enclosure rolling doors. In response:
\begin{itemize}
\item {} 
\sphinxAtStartPar
Pharos LASER ENCLOSURE INTERLOCK protocase laser warning module will display \DUrole{red}{LASER ON}.

\item {} 
\sphinxAtStartPar
The LOCAL INTERLOCK CONTACT CONTROL modules will stay armed.

\item {} 
\sphinxAtStartPar
Pharos LASER ENCLOSURE INTERLOCK protocase door monitor will display \DUrole{red}{CLOSED}.

\item {} 
\sphinxAtStartPar
Pharos UV and IR shutters will not close.

\item {} 
\sphinxAtStartPar
The Pharos power supply is cut off.

\end{itemize}

\item {} 
\sphinxAtStartPar
Turn the Pharos LASER ENCLOSURE INTERLOCK protocase INTERLOCK OVERRIDE key from \DUrole{red}{OVERRIDE} to \DUrole{red}{INTERLOCK}.
The STATUS LED changed to \DUrole{green}{green}.
The VIEWMARQ display and beacon stacks show a non\sphinxhyphen{}override status.

\end{enumerate}

\begin{figure}[htbp]
\centering
\capstart

\noindent\sphinxincludegraphics[scale=0.2]{{Vault-1_Control_VIEWMARQ_Pharos_override}.jpg}
\caption{\sphinxstylestrong{Figure 14:} This is the Vault\sphinxhyphen{}1 Control VIEWMARQ display when the system is overridden.}\label{\detokenize{testing_documentation/Vault-1_laser:id12}}\end{figure}

\begin{figure}[htbp]
\centering
\capstart

\noindent\sphinxincludegraphics[scale=0.2]{{Pharos_protocase_override}.jpg}
\caption{\sphinxstylestrong{Figure 15:} This is the Pharos LASER ENCLOSURE INTERLOCK protocase when the system is overridden.}\label{\detokenize{testing_documentation/Vault-1_laser:id13}}\end{figure}


\subsection{Arming the Dira Enclosure and Laser}
\label{\detokenize{testing_documentation/Vault-1_laser:arming-the-dira-enclosure-and-laser}}\begin{enumerate}
\sphinxsetlistlabels{\arabic}{enumi}{enumii}{}{.}%
\item {} 
\sphinxAtStartPar
Disarm Vault\sphinxhyphen{}1 and the Pharos.

\item {} 
\sphinxAtStartPar
See Laser Lab testing procedure for arming the Dira.
The laser warning module on Dira enclosure displays \DUrole{red}{DANGER LASER ON}.

\item {} 
\sphinxAtStartPar
Change the Dira LASER ENCLOSURE INTERLOCK protocase INTERLOCK OVERRIDE key from INTERLOCK to OVERRIDE.
The STATUS LED remains \DUrole{green}{green}. Change back to INTERLOCK.

\end{enumerate}


\subsection{Administrative Override on the Dira Enclosure}
\label{\detokenize{testing_documentation/Vault-1_laser:administrative-override-on-the-dira-enclosure}}\begin{enumerate}
\sphinxsetlistlabels{\arabic}{enumi}{enumii}{}{.}%
\item {} 
\sphinxAtStartPar
Open the Dira enclosure rolling doors.
In response:
\begin{itemize}
\item {} 
\sphinxAtStartPar
Dira and Pharos LASER ENCLOSURE INTERLOCK protocase laser warning module will display \DUrole{green}{LASER SAFE}

\item {} 
\sphinxAtStartPar
The LOCAL INTERLOCK CONTACT CONTROL modules will disarm and display \DUrole{orange}{LOCAL CONTACTS DISARMED} on the Dira and Pharos protocases.

\item {} 
\sphinxAtStartPar
Dira and Pharos LASER ENCLOSURE INTERLOCK protocase door monitor will display nothing.

\item {} 
\sphinxAtStartPar
Pharos UV and IR shutters will close.

\item {} 
\sphinxAtStartPar
Dira will lose power.

\item {} 
\sphinxAtStartPar
The Pharos power supply is cut off.

\end{itemize}

\item {} 
\sphinxAtStartPar
Rearm the Dira.

\item {} 
\sphinxAtStartPar
With Vault\sphinxhyphen{}1, the Dira armed, and the Pharos armed turn the INTERLOCK OVERRIDE key on the Dira LASER ENCLOSURE INTERLOCK protocase from \DUrole{red}{INTERLOCK} to \DUrole{red}{OVERRIDE}.
The STATUS LED will change to \DUrole{red}{red}.

\item {} 
\sphinxAtStartPar
The VIEWMARQ displays \DUrole{red}{DANGER LASER HAZARD\sphinxhyphen{}PHAROS ARMED\sphinxhyphen{}DIRA ARMED\sphinxhyphen{}DIRA ADMIN OVERRIDE}.

\item {} 
\sphinxAtStartPar
Beacon stacks show \DUrole{orange}{orange}, white, and \DUrole{blue}{blue} LEDs on.
\begin{itemize}
\item {} 
\sphinxAtStartPar
Vault\sphinxhyphen{}1 Control

\item {} 
\sphinxAtStartPar
Vault\sphinxhyphen{}1 east wall

\item {} 
\sphinxAtStartPar
Dira LASER ENCLOSURE INTERLOCK protocase.

\end{itemize}

\item {} 
\sphinxAtStartPar
Beacon stack on the Pharos LASER ENCLOSURE INTERLOCK protocase will show \DUrole{blue}{blue} LEDs on.

\item {} 
\sphinxAtStartPar
Turn the INTERLOCK OVERRIDE key on the Pharos LASER ENCLOSURE INTERLOCK protocol case from \DUrole{red}{INTERLOCK} to \DUrole{red}{OVERRIDE}. The STATUS LED will change to \DUrole{red}{red}.

\item {} 
\sphinxAtStartPar
The VIEWMARQ in Vault\sphinxhyphen{}1 Control will display \DUrole{red}{DANGER LASER ON \sphinxhyphen{} PHAROS ARMED \sphinxhyphen{} DIRA ARMED \sphinxhyphen{} PHAROS ADMIN OVERRIDE \sphinxhyphen{} DIRA ADMIN OVERRIDE}.

\item {} 
\sphinxAtStartPar
The beacon stack on the Pharos LASER ENCLOSURE INTERLOCK protocase will show \DUrole{orange}{orange} and \DUrole{blue}{blue} LEDs on.

\item {} 
\sphinxAtStartPar
With the Pharos, Dira, Vault\sphinxhyphen{}1, and LOCAL INTERLOCK CONTACT CONTROL armed and the Dira and Pharos enclosures set to override, open one of the Dira enclosure rolling doors.
In response:
\begin{itemize}
\item {} 
\sphinxAtStartPar
Pharos LASER ENCLOSURE INTERLOCK protocase laser warning module will display \DUrole{red}{DANGER LASER ON}

\item {} 
\sphinxAtStartPar
The LOCAL INTERLOCK CONTACT CONTROL modules will disarm on the Pharos.

\item {} 
\sphinxAtStartPar
Pharos UV and IR shutters will not close.

\item {} 
\sphinxAtStartPar
The Pharos power supply is cut off.

\end{itemize}

\item {} 
\sphinxAtStartPar
With the Pharos, Dira, and Vault\sphinxhyphen{}1 armed and the Dira and Pharos enclosures set to override, open one of the Dira enclosure rolling doors.
In response:
\begin{itemize}
\item {} 
\sphinxAtStartPar
Dira LASER ENCLOSURE INTERLOCK protocase warning module will display \DUrole{red}{DANGER LASER ON}.

\item {} 
\sphinxAtStartPar
Dira LASER ENCLOSURE INTERLOCK protocase door monitor will display \DUrole{red}{CLOSED}.

\item {} 
\sphinxAtStartPar
Pharos LASER ENCLOSURE INTERLOCK protocase door monitor will display nothing.

\item {} 
\sphinxAtStartPar
Pharos UV and IR shutters will close.

\end{itemize}

\end{enumerate}

\begin{figure}[htbp]
\centering
\capstart

\noindent\sphinxincludegraphics[scale=0.2]{{Vault-1_VIEWMARQ_all_armed}.jpg}
\caption{\sphinxstylestrong{Figure 16:} This is the Vault\sphinxhyphen{}1 Control VIEWMARQ display when the system is overridden.}\label{\detokenize{testing_documentation/Vault-1_laser:id14}}\end{figure}

\begin{figure}[htbp]
\centering
\capstart

\noindent\sphinxincludegraphics[scale=0.2]{{Dira_protocase_override}.jpg}
\caption{\sphinxstylestrong{Figure 17:} This is the Dira LASER ENCLOSURE INTERLOCK protocase when the system is overridden.}\label{\detokenize{testing_documentation/Vault-1_laser:id15}}\end{figure}


\subsection{Crashing the Dira Laser}
\label{\detokenize{testing_documentation/Vault-1_laser:crashing-the-dira-laser}}\begin{enumerate}
\sphinxsetlistlabels{\arabic}{enumi}{enumii}{}{.}%
\item {} 
\sphinxAtStartPar
Once every 6 months, the Pharos laser emergency stop buttons are tested that they can successfully cut power to the Pharos from a functional state.
Verify if the last testing date was 6 months ago.

\item {} 
\sphinxAtStartPar
If 6 months have passed, arm the Pharos laser, and use one of the laser e\sphinxhyphen{}stops to crash the laser and verify that power has been cut.

\end{enumerate}


\subsection{Return to Starting Conditions}
\label{\detokenize{testing_documentation/Vault-1_laser:return-to-starting-conditions}}\begin{enumerate}
\sphinxsetlistlabels{\arabic}{enumi}{enumii}{}{.}%
\item {} 
\sphinxAtStartPar
Return the Vault\sphinxhyphen{}1 laser interlock system back to starting conditions.

\end{enumerate}

\sphinxstepscope


\section{Hutch\sphinxhyphen{}1 Laser Interlock System Testing Protocol}
\label{\detokenize{testing_documentation/Hutch-1_laser:hutch-1-laser-interlock-system-testing-protocol}}\label{\detokenize{testing_documentation/Hutch-1_laser::doc}}
\sphinxAtStartPar
The purpose of this document is to describe the testing protocol for the Hutch\sphinxhyphen{}1 laser interlock system including the interlocks for the Astrella laser and enclosure.


\subsection{Starting Conditions}
\label{\detokenize{testing_documentation/Hutch-1_laser:starting-conditions}}\begin{enumerate}
\sphinxsetlistlabels{\arabic}{enumi}{enumii}{}{.}%
\item {} 
\sphinxAtStartPar
VIEWMARQ in Hutch\sphinxhyphen{}1 Control displays \DUrole{green}{LASER SAFE}.

\item {} 
\sphinxAtStartPar
Beacon Stacks have \DUrole{green}{green} LED on.
\begin{itemize}
\item {} 
\sphinxAtStartPar
Hutch\sphinxhyphen{}1 Control.

\item {} 
\sphinxAtStartPar
Astrella LASER ENCLOSURE INTERLOCK protocase.

\end{itemize}

\item {} 
\sphinxAtStartPar
Hutch\sphinxhyphen{}1 Control Entry keypad LED shows \DUrole{green}{RELEASED}.

\item {} 
\sphinxAtStartPar
Hutch\sphinxhyphen{}1 Control laser warning modules show \DUrole{green}{LASER SAFE}.
\begin{itemize}
\item {} 
\sphinxAtStartPar
Hutch\sphinxhyphen{}1 Control.

\item {} 
\sphinxAtStartPar
Astrella enclosure south wall.

\item {} 
\sphinxAtStartPar
Astrella enclosure west wall.

\end{itemize}

\item {} 
\sphinxAtStartPar
Hutch\sphinxhyphen{}1 Laser control module shows \DUrole{green}{LASER SAFE}.
The module will show \DUrole{orange}{ACCESS} when the door is open.

\item {} 
\sphinxAtStartPar
The laser warning module on the Astrella LASER ENCLOSURE INTERLOCK protocase shows \DUrole{red}{DANGER LASER ON}.

\item {} 
\sphinxAtStartPar
None of the Astrella laser e\sphinxhyphen{}stops are on.
\begin{itemize}
\item {} 
\sphinxAtStartPar
Astrella enclosure outside south wall.

\item {} 
\sphinxAtStartPar
Astrella enclosure outside west wall.

\item {} 
\sphinxAtStartPar
Astrella enclosure inside south wall.

\item {} 
\sphinxAtStartPar
Astrella enclosure inside west wall.

\item {} 
\sphinxAtStartPar
Astrella enclosure inside east wall.

\item {} 
\sphinxAtStartPar
Hutch\sphinxhyphen{}1 experimental chamber.

\end{itemize}

\item {} 
\sphinxAtStartPar
All room interlock modules light the LED for \DUrole{green}{ROOM DISABLED (READY TO ARM)}.
\begin{itemize}
\item {} 
\sphinxAtStartPar
Hutch\sphinxhyphen{}1.

\item {} 
\sphinxAtStartPar
Astrella enclosure.

\end{itemize}

\item {} 
\sphinxAtStartPar
Local interlock modules light the LEDs for \DUrole{green}{LOCAL CONTACTS DISARMED}, and \DUrole{green}{ROOM NOT ARMED \sphinxhyphen{} LOCAL CONTACTS CANNOT ARM}.
\begin{itemize}
\item {} 
\sphinxAtStartPar
Astrella enclosure.

\item {} 
\sphinxAtStartPar
Astrella LASER ENCLOSURE INTERLOCK protocase CONTROL \#1.

\item {} 
\sphinxAtStartPar
Astrella LASER ENCLOSURE INTERLOCK protocase CONTROL \#2.

\end{itemize}

\item {} 
\sphinxAtStartPar
Astrella LASER ENCLOSURE INTERLOCK protocase INTERLOCK OVERRIDE panel is set to interlock.
\begin{itemize}
\item {} 
\sphinxAtStartPar
Administrative override key is set to INTERLOCK.

\item {} 
\sphinxAtStartPar
STATUS lamp is \DUrole{green}{green}.

\end{itemize}

\item {} 
\sphinxAtStartPar
All Astrella enclosure curtain doors are closed, and the Door / Curtain Monitor should show \DUrole{green}{CLOSED}.
\begin{itemize}
\item {} 
\sphinxAtStartPar
West enclosure door 1.

\item {} 
\sphinxAtStartPar
West enclosure door 2.

\item {} 
\sphinxAtStartPar
North enclosure door 1.

\item {} 
\sphinxAtStartPar
North enclosure door 2.

\item {} 
\sphinxAtStartPar
East enclosure door 1.

\item {} 
\sphinxAtStartPar
East enclosure door 2.

\end{itemize}

\item {} 
\sphinxAtStartPar
Push before exit button is not on.

\end{enumerate}

\begin{figure}[htbp]
\centering
\capstart

\noindent\sphinxincludegraphics[scale=0.2]{{Hutch-1_VIEWMARQ_laser_safe}.jpg}
\caption{\sphinxstylestrong{Figure 1:} Hutch\sphinxhyphen{}1 Control VIEWMARQ showing LASER SAFE.}\label{\detokenize{testing_documentation/Hutch-1_laser:id1}}\end{figure}

\begin{figure}[htbp]
\centering
\capstart

\noindent\sphinxincludegraphics[scale=0.2]{{Hutch-1_entry_disarmed}.jpg}
\caption{\sphinxstylestrong{Figure 2:} Hutch\sphinxhyphen{}1 Control Entry Keypad showing RELEASED.}\label{\detokenize{testing_documentation/Hutch-1_laser:id2}}\end{figure}

\begin{figure}[htbp]
\centering
\capstart

\noindent\sphinxincludegraphics[scale=0.2]{{Hutch-1_unarmed}.jpg}
\caption{\sphinxstylestrong{Figure 3:} Hutch\sphinxhyphen{}1 Control showing all room interlock modules disabled.}\label{\detokenize{testing_documentation/Hutch-1_laser:id3}}\end{figure}

\begin{figure}[htbp]
\centering
\capstart

\noindent\sphinxincludegraphics[scale=0.2]{{Astrella_enclosure_unarmed}.jpg}
\caption{\sphinxstylestrong{Figure 4:} Astrella enclosure showing all local interlock modules disarmed.}\label{\detokenize{testing_documentation/Hutch-1_laser:id4}}\end{figure}

\begin{figure}[htbp]
\centering
\capstart

\noindent\sphinxincludegraphics[scale=0.2]{{Astrella_protocase}.jpg}
\caption{\sphinxstylestrong{Figure 5:} Astrella LASER ENCLOSURE INTERLOCK protocase showing the INTERLOCK OVERRIDE panel set to INTERLOCK.}\label{\detokenize{testing_documentation/Hutch-1_laser:id5}}\end{figure}

\begin{figure}[htbp]
\centering
\capstart

\noindent\sphinxincludegraphics[scale=0.2]{{laser_e-stop_off}.jpg}
\caption{\sphinxstylestrong{Figure 6:} Astrella enclosure showing all e\sphinxhyphen{}stops off.}\label{\detokenize{testing_documentation/Hutch-1_laser:id6}}\end{figure}


\subsection{Arming Hutch\sphinxhyphen{}1 as a Laser Lab}
\label{\detokenize{testing_documentation/Hutch-1_laser:arming-hutch-1-as-a-laser-lab}}\begin{enumerate}
\sphinxsetlistlabels{\arabic}{enumi}{enumii}{}{.}%
\item {} 
\sphinxAtStartPar
While inside of Hutch\sphinxhyphen{}1, close the first curtain and press ARM on the Hutch\sphinxhyphen{}1 room interlock control module. It lights the LED for \DUrole{orange}{ROOM ARMED}.

\item {} 
\sphinxAtStartPar
The laser control module shows DANGER LASER ON.

\item {} 
\sphinxAtStartPar
The Push Before Exit button is on.

\item {} 
\sphinxAtStartPar
The curtain door is locked magnetically locked.

\item {} 
\sphinxAtStartPar
Close the second curtain door. Verify that the second curtain door chimes when opened.

\item {} 
\sphinxAtStartPar
Use the Push Before Exit button to leave the Hutch, as you leave there is a chime sounding.

\item {} 
\sphinxAtStartPar
VIEWMARQ in Hutch\sphinxhyphen{}1 Control displays \DUrole{red}{DANGER LASER HAZARD}.

\item {} 
\sphinxAtStartPar
Hutch\sphinxhyphen{}1 Control laser warning modules display \DUrole{red}{DANGER LASER ON}.

\item {} 
\sphinxAtStartPar
Beacon stacks show no LEDs on.
\begin{itemize}
\item {} 
\sphinxAtStartPar
Hutch\sphinxhyphen{}1 Control.

\item {} 
\sphinxAtStartPar
Astrella LASER ENCLOSURE INTERLOCK protocase.

\end{itemize}

\item {} 
\sphinxAtStartPar
Entry keypad lights the LED for \DUrole{red}{INTERLOCKED}.

\item {} 
\sphinxAtStartPar
Type a random pin into the entry keypad.
The curtain door remains locked.

\item {} 
\sphinxAtStartPar
Type in the pin to open the Hutch curtain door.

\item {} 
\sphinxAtStartPar
Keypad module lights the LED for \DUrole{green}{RELEASED}.

\item {} 
\sphinxAtStartPar
Leave the curtain door open and allow the system to trip. It should trip in \DUrole{red}{x seconds/minutes}.
\begin{itemize}
\item {} 
\sphinxAtStartPar
The Hutch\sphinxhyphen{}1 room arm module shows \DUrole{orange}{ROOM CRASHED (CANNOT ARM)}, then \DUrole{green}{ROOM DISARMED (READY TO ARM)}.

\item {} 
\sphinxAtStartPar
Hutch\sphinxhyphen{}1 is now in starting conditions.

\end{itemize}

\end{enumerate}

\begin{figure}[htbp]
\centering
\capstart

\noindent\sphinxincludegraphics[scale=0.2]{{Hutch-1_VIEWMARQ_laser_hazard}.jpg}
\caption{\sphinxstylestrong{Figure 7:} Hutch\sphinxhyphen{}1 Control VIEWMARQ showing DANGER LASER HAZARD.}\label{\detokenize{testing_documentation/Hutch-1_laser:id7}}\end{figure}

\begin{figure}[htbp]
\centering
\capstart

\noindent\sphinxincludegraphics[scale=0.2]{{Hutch-1_armed}.jpg}
\caption{\sphinxstylestrong{Figure 8:} Hutch\sphinxhyphen{}1 Control showing the room interlock module armed.}\label{\detokenize{testing_documentation/Hutch-1_laser:id8}}\end{figure}

\begin{figure}[htbp]
\centering
\capstart

\noindent\sphinxincludegraphics[scale=0.2]{{Hutch-1_entry_armed}.jpg}
\caption{\sphinxstylestrong{Figure 9:} Hutch\sphinxhyphen{}1 Control Entry Keypad showing INTERLOCKED.}\label{\detokenize{testing_documentation/Hutch-1_laser:id9}}\end{figure}


\subsection{Arming the Astrella Enclosure and Laser}
\label{\detokenize{testing_documentation/Hutch-1_laser:arming-the-astrella-enclosure-and-laser}}\begin{enumerate}
\sphinxsetlistlabels{\arabic}{enumi}{enumii}{}{.}%
\item {} 
\sphinxAtStartPar
Attempt to arm the Astrella by arming local room interlock module on the Astrella enclosure before arming the room module.
Astrella does not arm.

\item {} 
\sphinxAtStartPar
With Hutch\sphinxhyphen{}1 unarmed, arm the room interlock module on the Astrella enclosure.

\item {} 
\sphinxAtStartPar
The room interlock module only lights the LED for \DUrole{orange}{ROOM ARMED}.

\item {} 
\sphinxAtStartPar
The local interlock module only lights the LED for \DUrole{green}{LOCAL CONTACT DISARMED}.

\item {} 
\sphinxAtStartPar
The laser warning control module shows \DUrole{red}{LASER ON}.

\item {} 
\sphinxAtStartPar
Laser E\sphinxhyphen{}stops buttons should be on.
\begin{itemize}
\item {} 
\sphinxAtStartPar
Outside enclosure south

\item {} 
\sphinxAtStartPar
Outside enclosure west

\item {} 
\sphinxAtStartPar
Experimental Chamber

\item {} 
\sphinxAtStartPar
Inside enclosure south

\item {} 
\sphinxAtStartPar
Inside enclosure east

\item {} 
\sphinxAtStartPar
Inside enclosure west

\end{itemize}

\item {} 
\sphinxAtStartPar
Arm the local room interlock module on the Astrella enclosure.
The local interlock module only lights the LED for \DUrole{green}{LOCAL CONTACTS ARMED}.

\item {} 
\sphinxAtStartPar
VIEWMARQ in Hutch\sphinxhyphen{}1 Control displays \DUrole{green}{LASER SAFE \sphinxhyphen{} ASTRELLA ARMED}.

\item {} \begin{description}
\sphinxlineitem{Beacon stacks show green and white LEDs on.}\begin{itemize}
\item {} 
\sphinxAtStartPar
Hutch\sphinxhyphen{}1 Control

\item {} 
\sphinxAtStartPar
Astrella LASER ENCLOSURE INTERLOCK protocase

\end{itemize}

\end{description}

\item {} 
\sphinxAtStartPar
Go through all Astrella enclosure curtain doors and open them one at a time.
In response:
\begin{itemize}
\item {} 
\sphinxAtStartPar
The Astrella enclosure laser warning module displays \DUrole{green}{LASER SAFE}.

\item {} 
\sphinxAtStartPar
The Door/Curtain Monitor shows nothing.

\item {} 
\sphinxAtStartPar
“MANUAL INTERLOCK OPEN” is flashing on the shutter controller.
\begin{itemize}
\item {} 
\sphinxAtStartPar
West enclosure door 1

\item {} 
\sphinxAtStartPar
West enclosure door 2

\item {} 
\sphinxAtStartPar
North enclosure door 1

\item {} 
\sphinxAtStartPar
North enclosure door 2

\item {} 
\sphinxAtStartPar
East enclosure door 1

\item {} 
\sphinxAtStartPar
East enclosure door 2

\end{itemize}

\end{itemize}

\item {} 
\sphinxAtStartPar
Arm CONTROL \#1 local interlock module.
It lights the LED for \DUrole{green}{LOCAL CONTACTS ARMED}.

\item {} 
\sphinxAtStartPar
Open the Astrella enclosure, the system trips, and the local interlock module lights the LEDs for \DUrole{green}{LOCAL CONTACTS DISARMED}, and \DUrole{green}{ROOM NOT ARMED \sphinxhyphen{} LOCAL CONTACT CANNOT ARM}.

\end{enumerate}
\begin{enumerate}
\sphinxsetlistlabels{\arabic}{enumi}{enumii}{}{.}%
\item {} 
\sphinxAtStartPar
Rearm Hutch\sphinxhyphen{}1 as a laser lab and rearm the Astrella.
All laser warning modules display \DUrole{red}{DANGER LASER ON}.

\item {} 
\sphinxAtStartPar
VIEWMARQ in Hutch\sphinxhyphen{}1 Control displays \DUrole{red}{DANGER LASER HAZARD \sphinxhyphen{} ASTRELLA ARMED}.

\item {} 
\sphinxAtStartPar
Beacon stacks show white LED on.
\begin{itemize}
\item {} 
\sphinxAtStartPar
Hutch\sphinxhyphen{}1 Control

\item {} 
\sphinxAtStartPar
Astrella LASER ENCLOSURE INTERLOCK protocase

\end{itemize}

\end{enumerate}

\begin{figure}[htbp]
\centering
\capstart

\noindent\sphinxincludegraphics[scale=0.2]{{Hutch-1_VIEWMARQ_laser_hazard_armed}.jpg}
\caption{\sphinxstylestrong{Figure 10:} Hutch\sphinxhyphen{}1 Control VIEWMARQ showing DANGER LASER HAZARD \sphinxhyphen{} ASTRELLA ARMED.}\label{\detokenize{testing_documentation/Hutch-1_laser:id10}}\end{figure}

\begin{figure}[htbp]
\centering
\capstart

\noindent\sphinxincludegraphics[scale=0.2]{{Astrella_enclosure_armed}.jpg}
\caption{\sphinxstylestrong{Figure 11:} Astrella enclosure showing the room interlock module armed.}\label{\detokenize{testing_documentation/Hutch-1_laser:id11}}\end{figure}


\subsection{Administrative Override on the Astrella Enclosure}
\label{\detokenize{testing_documentation/Hutch-1_laser:administrative-override-on-the-astrella-enclosure}}\begin{enumerate}
\sphinxsetlistlabels{\arabic}{enumi}{enumii}{}{.}%
\item {} 
\sphinxAtStartPar
Turn the Astrella LASER ENCLOSURE INTERLOCK protocase INTERLOCK OVERRIDE key from INTERLOCK to OVERRIDE.
The STATUS lamp will not change.

\item {} 
\sphinxAtStartPar
With Hutch and Astrella armed, turn the Astrella LASER ENCLOSURE INTERLOCK protocase INTERLOCK OVERRIDE key from INTERLOCK to OVERRIDE.
The STATUS lamp turns \DUrole{red}{red}.

\item {} 
\sphinxAtStartPar
VIEWMARQ in Hutch\sphinxhyphen{}1 Control displays \DUrole{red}{DANGER LASER HAZARD \sphinxhyphen{} ASTRELLA ARMED \sphinxhyphen{} ASTRELLA ADMIN OVERRIDE}.

\item {} 
\sphinxAtStartPar
Beacon stacks show white and \DUrole{orange}{orange} LEDs activated.
\begin{itemize}
\item {} \begin{enumerate}
\sphinxsetlistlabels{\alph}{enumii}{enumiii}{}{.}%
\item {} 
\sphinxAtStartPar
Hutch\sphinxhyphen{}1 Control

\end{enumerate}

\item {} \begin{enumerate}
\sphinxsetlistlabels{\alph}{enumii}{enumiii}{}{.}%
\setcounter{enumii}{1}
\item {} 
\sphinxAtStartPar
Astrella LASER ENCLOSURE INTERLOCK protocase

\end{enumerate}

\end{itemize}

\item {} 
\sphinxAtStartPar
Open the Astrella enclosure door.
The laser warning module displays \DUrole{red}{DANGER LASER ON}, and the Door/Curtain Monitor shows \DUrole{green}{CLOSED}.

\item {} 
\sphinxAtStartPar
Close the Astrella enclosure and arm CONTROL \#1 local interlock module, it lights the LED for \DUrole{orange}{LOCAL CONTACTS ARMED}.

\item {} 
\sphinxAtStartPar
Open the Astrella enclosure. “MANUAL\_” is flashing on the shutter controller.

\end{enumerate}

\begin{figure}[htbp]
\centering
\capstart

\noindent\sphinxincludegraphics[scale=0.2]{{Hutch-1_VIEWMARQ_override}.jpg}
\caption{\sphinxstylestrong{Figure 12:} Hutch\sphinxhyphen{}1 Control VIEWMARQ showing DANGER LASER HAZARD \sphinxhyphen{} ASTRELLA ARMED \sphinxhyphen{} ASTRELLA ADMIN OVERRIDE.}\label{\detokenize{testing_documentation/Hutch-1_laser:id12}}\end{figure}

\begin{figure}[htbp]
\centering
\capstart

\noindent\sphinxincludegraphics[scale=0.2]{{Astrella_override}.jpg}
\caption{\sphinxstylestrong{Figure 13:} Astrella enclosure showing the INTERLOCK OVERRIDE panel set to OVERRIDE.}\label{\detokenize{testing_documentation/Hutch-1_laser:id13}}\end{figure}


\subsection{Safe Astrella E\sphinxhyphen{}Stop Test}
\label{\detokenize{testing_documentation/Hutch-1_laser:safe-astrella-e-stop-test}}\begin{enumerate}
\sphinxsetlistlabels{\arabic}{enumi}{enumii}{}{.}%
\item {} 
\sphinxAtStartPar
Put the Astrella into a powered down state.

\item {} 
\sphinxAtStartPar
Arm the Astrella and Astrella enclosure.

\item {} 
\sphinxAtStartPar
Press one of the Astrella enclosure e\sphinxhyphen{}stops.

\item {} 
\sphinxAtStartPar
Verify that the Astrella power supply is cut off.

\end{enumerate}


\subsection{Crashing the Astrella}
\label{\detokenize{testing_documentation/Hutch-1_laser:crashing-the-astrella}}\begin{enumerate}
\sphinxsetlistlabels{\arabic}{enumi}{enumii}{}{.}%
\item {} 
\sphinxAtStartPar
Once every 6 months, the Astrella laser emergency stop buttons are testing that they can successfully cut power to the Astrella from a functional state.
Verify if the last testing date was 6 months ago.

\item {} 
\sphinxAtStartPar
If 6 months have passed, arm the Astrella laser and use one the the Astrella laser e\sphinxhyphen{}stop to crash the laser and verify that power has been cut.

\end{enumerate}


\subsection{Return to Starting Conditions}
\label{\detokenize{testing_documentation/Hutch-1_laser:return-to-starting-conditions}}\begin{enumerate}
\sphinxsetlistlabels{\arabic}{enumi}{enumii}{}{.}%
\item {} 
\sphinxAtStartPar
Return the Hutch\sphinxhyphen{}1 laser interlock system back to starting conditions.

\end{enumerate}

\sphinxstepscope


\section{Radiation Survey Equipment Calibration Check}
\label{\detokenize{testing_documentation/radiation_detection:radiation-survey-equipment-calibration-check}}\label{\detokenize{testing_documentation/radiation_detection::doc}}

\subsection{Apantec Monitors}
\label{\detokenize{testing_documentation/radiation_detection:apantec-monitors}}

\begin{savenotes}\sphinxattablestart
\centering
\begin{tabulary}{\linewidth}[t]{|T|T|T|T|T|}
\hline

\sphinxAtStartPar
Model
&
\sphinxAtStartPar
Serial Number
&
\sphinxAtStartPar
Calibration Date
&
\sphinxAtStartPar
Location
&
\sphinxAtStartPar
Description
\\
\hline
\sphinxAtStartPar
NH1X
&
\sphinxAtStartPar
2127NH01
&
\sphinxAtStartPar
08/18/24
&
\sphinxAtStartPar
Hutch\sphinxhyphen{}1
&
\sphinxAtStartPar
Neutron Detector
\\
\hline
\sphinxAtStartPar
NH1X
&
\sphinxAtStartPar
2127NH02
&
\sphinxAtStartPar
08/18/24
&
\sphinxAtStartPar
Vault\sphinxhyphen{}1 Control
&
\sphinxAtStartPar
Neutron Detector
\\
\hline
\sphinxAtStartPar
NH1X
&
\sphinxAtStartPar
2127NH03
&
\sphinxAtStartPar
08/18/24
&
\sphinxAtStartPar
Accelerator Lab
&
\sphinxAtStartPar
Neutron Detector
\\
\hline
\sphinxAtStartPar
NB1XV2
&
\sphinxAtStartPar
2139NB01
&
\sphinxAtStartPar
08/18/24
&
\sphinxAtStartPar
Vault\sphinxhyphen{}1
&
\sphinxAtStartPar
Neutron Detector
\\
\hline
\sphinxAtStartPar
GG2W
&
\sphinxAtStartPar
2139GG01
&
\sphinxAtStartPar
11/10/24
&
\sphinxAtStartPar
Hutch\sphinxhyphen{}1
&
\sphinxAtStartPar
Gamma Detector
\\
\hline
\sphinxAtStartPar
GG2W
&
\sphinxAtStartPar
2139GG02
&
\sphinxAtStartPar
08/18/24
&
\sphinxAtStartPar
Accelerator Lab
&
\sphinxAtStartPar
Gamma Detector
\\
\hline
\sphinxAtStartPar
GG1W
&
\sphinxAtStartPar
2139GG03
&
\sphinxAtStartPar
08/18/24
&
\sphinxAtStartPar
RF\sphinxhyphen{}1
&
\sphinxAtStartPar
Gamma Detector
\\
\hline
\sphinxAtStartPar
GG1W
&
\sphinxAtStartPar
2139GG04
&
\sphinxAtStartPar
08/18/24
&
\sphinxAtStartPar
Laser\sphinxhyphen{}1
&
\sphinxAtStartPar
Gamma Detector
\\
\hline
\sphinxAtStartPar
GG1W
&
\sphinxAtStartPar
2139GG05
&
\sphinxAtStartPar
08/18/24
&
\sphinxAtStartPar
Vault\sphinxhyphen{}1 Control
&
\sphinxAtStartPar
Gamma Detector
\\
\hline
\sphinxAtStartPar
Gl1V1
&
\sphinxAtStartPar
2139Gl02
&
\sphinxAtStartPar
08/18/24
&
\sphinxAtStartPar
Vault\sphinxhyphen{}1
&
\sphinxAtStartPar
Gamma Detector
\\
\hline
\end{tabulary}
\par
\sphinxattableend\end{savenotes}


\subsection{Ludlum Monitors}
\label{\detokenize{testing_documentation/radiation_detection:ludlum-monitors}}

\begin{savenotes}\sphinxattablestart
\centering
\begin{tabulary}{\linewidth}[t]{|T|T|T|T|T|}
\hline

\sphinxAtStartPar
Model
&
\sphinxAtStartPar
Serial Number
&
\sphinxAtStartPar
Calibration Date
&
\sphinxAtStartPar
Location
&
\sphinxAtStartPar
Description
\\
\hline
\sphinxAtStartPar
9DP
&
\sphinxAtStartPar
25019647
&
\sphinxAtStartPar
08/18/24
&
\sphinxAtStartPar
Accelerator Lab
&
\sphinxAtStartPar
Portable Ion Chamber
\\
\hline
\sphinxAtStartPar
9DP
&
\sphinxAtStartPar
25024335
&
\sphinxAtStartPar
08/18/24
&
\sphinxAtStartPar
Accelerator Lab
&
\sphinxAtStartPar
Portable Ion Chamber
\\
\hline
\sphinxAtStartPar
2363
&
\sphinxAtStartPar
332393
&
\sphinxAtStartPar
08/23/24
&
\sphinxAtStartPar
Accelerator Lab
&
\sphinxAtStartPar
Geiger Muller Counter
\\
\hline
\sphinxAtStartPar
42\sphinxhyphen{}41L
&
\sphinxAtStartPar
PR385342
&
\sphinxAtStartPar
08/23/24
&
\sphinxAtStartPar
Accelerator Lab
&
\sphinxAtStartPar
Proportional Neutron Counter on 2363
\\
\hline
\sphinxAtStartPar
23
&
\sphinxAtStartPar
D191938
&
\sphinxAtStartPar
08/30/24
&
\sphinxAtStartPar
Accelerator Lab
&
\sphinxAtStartPar
Personal Electric Dosimeter
\\
\hline
\sphinxAtStartPar
23
&
\sphinxAtStartPar
D191997
&
\sphinxAtStartPar
08/30/24
&
\sphinxAtStartPar
Accelerator Lab
&
\sphinxAtStartPar
Personal Electric Dosimeter
\\
\hline
\sphinxAtStartPar
23
&
\sphinxAtStartPar
D191943
&
\sphinxAtStartPar
08/30/24
&
\sphinxAtStartPar
Accelerator Lab
&
\sphinxAtStartPar
Personal Electric Dosimeter
\\
\hline
\sphinxAtStartPar
375DUAL 44\sphinxhyphen{}2
&
\sphinxAtStartPar
PR393130
&
\sphinxAtStartPar
12/09/24
&
\sphinxAtStartPar
Accelerator Lab
&
\sphinxAtStartPar
Area Gamma Detector
\\
\hline
\sphinxAtStartPar
375DUAL 42\sphinxhyphen{}30H
&
\sphinxAtStartPar
PR392238
&
\sphinxAtStartPar
12/09/24
&
\sphinxAtStartPar
Accelerator Lab
&
\sphinxAtStartPar
Area Neutron Detector
\\
\hline
\end{tabulary}
\par
\sphinxattableend\end{savenotes}


\subsection{O2 Monitors}
\label{\detokenize{testing_documentation/radiation_detection:o2-monitors}}

\subsection{Microwave Monitors}
\label{\detokenize{testing_documentation/radiation_detection:microwave-monitors}}
\sphinxstepscope


\section{CXLS Interlock System Testing Log}
\label{\detokenize{testing_documentation/testing_log:cxls-interlock-system-testing-log}}\label{\detokenize{testing_documentation/testing_log::doc}}

\begin{savenotes}\sphinxattablestart
\centering
\begin{tabulary}{\linewidth}[t]{|T|T|T|}
\hline
\sphinxstyletheadfamily 
\sphinxAtStartPar
Tester
&\sphinxstyletheadfamily 
\sphinxAtStartPar
Date
&\sphinxstyletheadfamily 
\sphinxAtStartPar
Notes
\\
\hline
\sphinxAtStartPar
Eric Everett
&
\sphinxAtStartPar
10/03/2023
&
\sphinxAtStartPar
First test run through of the testing documentation.
\\
\hline
\end{tabulary}
\par
\sphinxattableend\end{savenotes}

\sphinxstepscope


\section{IDEM Safety Relay Troubleshooting Guide}
\label{\detokenize{troubleshooting_documentation/IDEM_relay:idem-safety-relay-troubleshooting-guide}}\label{\detokenize{troubleshooting_documentation/IDEM_relay::doc}}
\sphinxAtStartPar
In the CXLS interlocks system, we use two modules of IDEM safety relays, SCR\sphinxhyphen{}21\sphinxhyphen{}i and SCR\sphinxhyphen{}31\sphinxhyphen{}i.

\noindent{\hspace*{\fill}\sphinxincludegraphics{{IDEM}.png}\hspace*{\fill}}

\sphinxAtStartPar
The face of the IDEM relays have 6 diagnostics LEDs:
\begin{enumerate}
\sphinxsetlistlabels{\arabic}{enumi}{enumii}{}{.}%
\item {} 
\sphinxAtStartPar
POWER: Power to the safety relay.

\item {} 
\sphinxAtStartPar
RESET: Reset loop S11\sphinxhyphen{}S21 or S11\sphinxhyphen{}S22 is closed (see manual).

\item {} 
\sphinxAtStartPar
CH1: Channel 1 control loop S11\sphinxhyphen{}S21 is closed.

\item {} 
\sphinxAtStartPar
CH2: Channel 2 control loop S13\sphinxhyphen{}S10 is closed.

\item {} 
\sphinxAtStartPar
K1: Power to internal relay K1.

\item {} 
\sphinxAtStartPar
K2: Power to internal relay K2.

\end{enumerate}

\sphinxAtStartPar
During normal operations all 6 of these lights should be on if the relay is supposed to be receiving power.
If any of the LEDs are not on, there is an error occurring.




\subsection{Resolved Issues}
\label{\detokenize{troubleshooting_documentation/IDEM_relay:resolved-issues}}\begin{enumerate}
\sphinxsetlistlabels{\arabic}{enumi}{enumii}{}{.}%
\item {} 
\sphinxAtStartPar
A relay was not properly actuating when receiving power, and would show either the K1 or K2 off.
It was discovered that the relay was receiving too voltage to function properly due to voltage drops from Apantec breakout boards.
The operating voltage for these units is 24VDC, with a tolerance of 20.4\sphinxhyphen{}26.4VDC.

\end{enumerate}

\sphinxstepscope


\section{Apantec Ionizing Radiation Sensor No Count Fail Troubleshooting Guide}
\label{\detokenize{troubleshooting_documentation/apantec_ncf:apantec-ionizing-radiation-sensor-no-count-fail-troubleshooting-guide}}\label{\detokenize{troubleshooting_documentation/apantec_ncf::doc}}
\sphinxAtStartPar
The each gamma and neutron probe for the Apantec area monitoring system has a no count fail time (NCFT) setting.
This set point tells the probe that if radiation is not detected over a certain period of time, then the probe if failing.

\sphinxAtStartPar
This NCFT has caused false fails because of the during of time we go without running the beam.
The CXLS interlock system is currently in an override state, bypassing the fail signal so that the accelerator can be armed.
Once RF is enabled in to the accelerator structures and the probes see radiation, the fail is cleared.

\begin{sphinxadmonition}{note}{Note:}
\sphinxAtStartPar
We are currently exploring ways to work with the NCFT.
This could mean changing the setting, or working in a manual reset in the radiation monitoring software.
\end{sphinxadmonition}

\sphinxstepscope


\section{VIEWMARQ Marquee Display Interlock Error Troubleshooting Guide}
\label{\detokenize{troubleshooting_documentation/viewmarq:viewmarq-marquee-display-interlock-error-troubleshooting-guide}}\label{\detokenize{troubleshooting_documentation/viewmarq::doc}}
\sphinxAtStartPar
The VIEWMARQ Marquee displays show the status of the room the display oversees.
It knows the status of the room by receiving an 8\sphinxhyphen{}bit signal from the possible hazards, where each hazard corresponds to one of the 8 channels.


\begin{savenotes}\sphinxattablestart
\centering
\sphinxcapstartof{table}
\sphinxthecaptionisattop
\sphinxcaption{Vault\sphinxhyphen{}1 Control VIEWMARQ Display Bit Channels}\label{\detokenize{troubleshooting_documentation/viewmarq:id1}}
\sphinxaftertopcaption
\begin{tabulary}{\linewidth}[t]{|T|T|}
\hline
\sphinxstyletheadfamily 
\sphinxAtStartPar
Bit Channel
&\sphinxstyletheadfamily 
\sphinxAtStartPar
Status
\\
\hline
\sphinxAtStartPar
1
&
\sphinxAtStartPar
Ionizing Radiation E\sphinxhyphen{}stop Active
\\
\hline
\sphinxAtStartPar
2
&
\sphinxAtStartPar
RF Armed
\\
\hline
\sphinxAtStartPar
3
&
\sphinxAtStartPar
Pharos Admin Override
\\
\hline
\sphinxAtStartPar
4
&
\sphinxAtStartPar
Dira Admin Override
\\
\hline
\sphinxAtStartPar
5
&
\sphinxAtStartPar
Laser Safety
\\
\hline
\sphinxAtStartPar
6
&
\sphinxAtStartPar
Pharos Armed
\\
\hline
\sphinxAtStartPar
7
&
\sphinxAtStartPar
Dira Armed
\\
\hline
\sphinxAtStartPar
8
&
\sphinxAtStartPar
Unused Bit Channel
\\
\hline
\end{tabulary}
\par
\sphinxattableend\end{savenotes}


\begin{savenotes}\sphinxattablestart
\centering
\sphinxcapstartof{table}
\sphinxthecaptionisattop
\sphinxcaption{Hutch\sphinxhyphen{}1 Control VIEWMARQ Display Bit Channels}\label{\detokenize{troubleshooting_documentation/viewmarq:id2}}
\sphinxaftertopcaption
\begin{tabulary}{\linewidth}[t]{|T|T|}
\hline
\sphinxstyletheadfamily 
\sphinxAtStartPar
Bit Channel
&\sphinxstyletheadfamily 
\sphinxAtStartPar
Status
\\
\hline
\sphinxAtStartPar
1
&
\sphinxAtStartPar
Unused Bit Channel
\\
\hline
\sphinxAtStartPar
2
&
\sphinxAtStartPar
Unused Bit Channel
\\
\hline
\sphinxAtStartPar
3
&
\sphinxAtStartPar
Unused Bit Channel
\\
\hline
\sphinxAtStartPar
4
&
\sphinxAtStartPar
Astrella Admin Override
\\
\hline
\sphinxAtStartPar
5
&
\sphinxAtStartPar
Unused Bit Channel
\\
\hline
\sphinxAtStartPar
6
&
\sphinxAtStartPar
Laser Safety
\\
\hline
\sphinxAtStartPar
7
&
\sphinxAtStartPar
Unused Bit Channel
\\
\hline
\sphinxAtStartPar
8
&
\sphinxAtStartPar
Astrella Armed
\\
\hline
\end{tabulary}
\par
\sphinxattableend\end{savenotes}


\begin{savenotes}\sphinxattablestart
\centering
\sphinxcapstartof{table}
\sphinxthecaptionisattop
\sphinxcaption{Accelerator Lab VIEWMARQ Display Bit Channels}\label{\detokenize{troubleshooting_documentation/viewmarq:id3}}
\sphinxaftertopcaption
\begin{tabulary}{\linewidth}[t]{|T|T|}
\hline
\sphinxstyletheadfamily 
\sphinxAtStartPar
Bit Channel
&\sphinxstyletheadfamily 
\sphinxAtStartPar
Status
\\
\hline
\sphinxAtStartPar
1
&
\sphinxAtStartPar
Ionizing Radiation E\sphinxhyphen{}stop Active
\\
\hline
\sphinxAtStartPar
2
&
\sphinxAtStartPar
RF Armed
\\
\hline
\sphinxAtStartPar
3
&
\sphinxAtStartPar
Unused Bit Channel
\\
\hline
\sphinxAtStartPar
4
&
\sphinxAtStartPar
Unused Bit Channel
\\
\hline
\sphinxAtStartPar
5
&
\sphinxAtStartPar
Unused Bit Channel
\\
\hline
\sphinxAtStartPar
6
&
\sphinxAtStartPar
Unused Bit Channel
\\
\hline
\sphinxAtStartPar
7
&
\sphinxAtStartPar
Unused Bit Channel
\\
\hline
\sphinxAtStartPar
8
&
\sphinxAtStartPar
Unused Bit Channel
\\
\hline
\end{tabulary}
\par
\sphinxattableend\end{savenotes}


\subsection{Resolved Issues}
\label{\detokenize{troubleshooting_documentation/viewmarq:resolved-issues}}\begin{enumerate}
\sphinxsetlistlabels{\arabic}{enumi}{enumii}{}{.}%
\item {} 
\sphinxAtStartPar
The a VIEWMARQ displays shows \DUrole{red}{Interlock System Error: xxxxxxxx}, where the xxxxxxxx is an 8\sphinxhyphen{}bit binary number, if there is a state that should not exist.
The error was that the specific 8\sphinxhyphen{}bit combination was not programmed into the VIEWMARQ display.
All 255 possible combinations of the 8\sphinxhyphen{}bit signal must be programmed individually, therefore is a combination was not considered or overlooked, a false error will occur.

\item {} 
\sphinxAtStartPar
The VIEWMARQ display either shows the wrong status or an error status.
This error was caused by incorrect wiring in the interlock system aggregator panel between the relays.
This could also be caused by a faulty relay, bad power supply, or other downstream issues.

\end{enumerate}

\sphinxstepscope


\section{Laser Safety Systems Troubleshooting Guide}
\label{\detokenize{troubleshooting_documentation/laser_safety_systems:laser-safety-systems-troubleshooting-guide}}\label{\detokenize{troubleshooting_documentation/laser_safety_systems::doc}}
\sphinxAtStartPar
For the CXLS laser interlock systems, Laser Safety Systems modules are used to perform the majority of the interlocking function.




\subsection{Resolved Issues}
\label{\detokenize{troubleshooting_documentation/laser_safety_systems:resolved-issues}}\begin{enumerate}
\sphinxsetlistlabels{\arabic}{enumi}{enumii}{}{.}%
\item {} 
\sphinxAtStartPar
The Laser\sphinxhyphen{}1 VIEWMARQ display was not updating when Laser\sphinxhyphen{}1 was armed.
This was traced back to the the corresponding arming module relay was not being energized.
This was due to loose contact terminals, barely holding onto the PCB, causing intermittent contact.
Replacing the arming module fixed this issue.

\item {} 
\sphinxAtStartPar
Again, the Laser\sphinxhyphen{}1 VIEWMARQ display was not updating correctly.
As well, the Laser\sphinxhyphen{}1 control module was showing dim lighting and the keypad was working, but with no LED or audible feedback.
This was traced back to the Laser\sphinxhyphen{}1 control module.
The relay contact that sends a signal to the Laser\sphinxhyphen{}1 VIEWMARQ display was being energized by the push to exit button, not the arming button.
We are not sure why, nor why the keypad was affect, but replacing the Laser\sphinxhyphen{}1 control module fixed this issue.

\end{enumerate}

\begin{sphinxadmonition}{note}{Note:}
\sphinxAtStartPar
This documentation is a work in progress.
\end{sphinxadmonition}



\renewcommand{\indexname}{Index}
\printindex
\end{document}